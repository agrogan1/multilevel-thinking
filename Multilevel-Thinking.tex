% Options for packages loaded elsewhere
\PassOptionsToPackage{unicode}{hyperref}
\PassOptionsToPackage{hyphens}{url}
\PassOptionsToPackage{dvipsnames,svgnames,x11names}{xcolor}
%
\documentclass[
  letterpaper,
  DIV=11,
  numbers=noendperiod]{scrreprt}

\usepackage{amsmath,amssymb}
\usepackage{setspace}
\usepackage{iftex}
\ifPDFTeX
  \usepackage[T1]{fontenc}
  \usepackage[utf8]{inputenc}
  \usepackage{textcomp} % provide euro and other symbols
\else % if luatex or xetex
  \usepackage{unicode-math}
  \defaultfontfeatures{Scale=MatchLowercase}
  \defaultfontfeatures[\rmfamily]{Ligatures=TeX,Scale=1}
\fi
\usepackage{lmodern}
\ifPDFTeX\else  
    % xetex/luatex font selection
\fi
% Use upquote if available, for straight quotes in verbatim environments
\IfFileExists{upquote.sty}{\usepackage{upquote}}{}
\IfFileExists{microtype.sty}{% use microtype if available
  \usepackage[]{microtype}
  \UseMicrotypeSet[protrusion]{basicmath} % disable protrusion for tt fonts
}{}
\makeatletter
\@ifundefined{KOMAClassName}{% if non-KOMA class
  \IfFileExists{parskip.sty}{%
    \usepackage{parskip}
  }{% else
    \setlength{\parindent}{0pt}
    \setlength{\parskip}{6pt plus 2pt minus 1pt}}
}{% if KOMA class
  \KOMAoptions{parskip=half}}
\makeatother
\usepackage{xcolor}
\setlength{\emergencystretch}{3em} % prevent overfull lines
\setcounter{secnumdepth}{5}
% Make \paragraph and \subparagraph free-standing
\makeatletter
\ifx\paragraph\undefined\else
  \let\oldparagraph\paragraph
  \renewcommand{\paragraph}{
    \@ifstar
      \xxxParagraphStar
      \xxxParagraphNoStar
  }
  \newcommand{\xxxParagraphStar}[1]{\oldparagraph*{#1}\mbox{}}
  \newcommand{\xxxParagraphNoStar}[1]{\oldparagraph{#1}\mbox{}}
\fi
\ifx\subparagraph\undefined\else
  \let\oldsubparagraph\subparagraph
  \renewcommand{\subparagraph}{
    \@ifstar
      \xxxSubParagraphStar
      \xxxSubParagraphNoStar
  }
  \newcommand{\xxxSubParagraphStar}[1]{\oldsubparagraph*{#1}\mbox{}}
  \newcommand{\xxxSubParagraphNoStar}[1]{\oldsubparagraph{#1}\mbox{}}
\fi
\makeatother

\usepackage{color}
\usepackage{fancyvrb}
\newcommand{\VerbBar}{|}
\newcommand{\VERB}{\Verb[commandchars=\\\{\}]}
\DefineVerbatimEnvironment{Highlighting}{Verbatim}{commandchars=\\\{\}}
% Add ',fontsize=\small' for more characters per line
\usepackage{framed}
\definecolor{shadecolor}{RGB}{241,243,245}
\newenvironment{Shaded}{\begin{snugshade}}{\end{snugshade}}
\newcommand{\AlertTok}[1]{\textcolor[rgb]{0.68,0.00,0.00}{#1}}
\newcommand{\AnnotationTok}[1]{\textcolor[rgb]{0.37,0.37,0.37}{#1}}
\newcommand{\AttributeTok}[1]{\textcolor[rgb]{0.40,0.45,0.13}{#1}}
\newcommand{\BaseNTok}[1]{\textcolor[rgb]{0.68,0.00,0.00}{#1}}
\newcommand{\BuiltInTok}[1]{\textcolor[rgb]{0.00,0.23,0.31}{#1}}
\newcommand{\CharTok}[1]{\textcolor[rgb]{0.13,0.47,0.30}{#1}}
\newcommand{\CommentTok}[1]{\textcolor[rgb]{0.37,0.37,0.37}{#1}}
\newcommand{\CommentVarTok}[1]{\textcolor[rgb]{0.37,0.37,0.37}{\textit{#1}}}
\newcommand{\ConstantTok}[1]{\textcolor[rgb]{0.56,0.35,0.01}{#1}}
\newcommand{\ControlFlowTok}[1]{\textcolor[rgb]{0.00,0.23,0.31}{\textbf{#1}}}
\newcommand{\DataTypeTok}[1]{\textcolor[rgb]{0.68,0.00,0.00}{#1}}
\newcommand{\DecValTok}[1]{\textcolor[rgb]{0.68,0.00,0.00}{#1}}
\newcommand{\DocumentationTok}[1]{\textcolor[rgb]{0.37,0.37,0.37}{\textit{#1}}}
\newcommand{\ErrorTok}[1]{\textcolor[rgb]{0.68,0.00,0.00}{#1}}
\newcommand{\ExtensionTok}[1]{\textcolor[rgb]{0.00,0.23,0.31}{#1}}
\newcommand{\FloatTok}[1]{\textcolor[rgb]{0.68,0.00,0.00}{#1}}
\newcommand{\FunctionTok}[1]{\textcolor[rgb]{0.28,0.35,0.67}{#1}}
\newcommand{\ImportTok}[1]{\textcolor[rgb]{0.00,0.46,0.62}{#1}}
\newcommand{\InformationTok}[1]{\textcolor[rgb]{0.37,0.37,0.37}{#1}}
\newcommand{\KeywordTok}[1]{\textcolor[rgb]{0.00,0.23,0.31}{\textbf{#1}}}
\newcommand{\NormalTok}[1]{\textcolor[rgb]{0.00,0.23,0.31}{#1}}
\newcommand{\OperatorTok}[1]{\textcolor[rgb]{0.37,0.37,0.37}{#1}}
\newcommand{\OtherTok}[1]{\textcolor[rgb]{0.00,0.23,0.31}{#1}}
\newcommand{\PreprocessorTok}[1]{\textcolor[rgb]{0.68,0.00,0.00}{#1}}
\newcommand{\RegionMarkerTok}[1]{\textcolor[rgb]{0.00,0.23,0.31}{#1}}
\newcommand{\SpecialCharTok}[1]{\textcolor[rgb]{0.37,0.37,0.37}{#1}}
\newcommand{\SpecialStringTok}[1]{\textcolor[rgb]{0.13,0.47,0.30}{#1}}
\newcommand{\StringTok}[1]{\textcolor[rgb]{0.13,0.47,0.30}{#1}}
\newcommand{\VariableTok}[1]{\textcolor[rgb]{0.07,0.07,0.07}{#1}}
\newcommand{\VerbatimStringTok}[1]{\textcolor[rgb]{0.13,0.47,0.30}{#1}}
\newcommand{\WarningTok}[1]{\textcolor[rgb]{0.37,0.37,0.37}{\textit{#1}}}

\providecommand{\tightlist}{%
  \setlength{\itemsep}{0pt}\setlength{\parskip}{0pt}}\usepackage{longtable,booktabs,array}
\usepackage{calc} % for calculating minipage widths
% Correct order of tables after \paragraph or \subparagraph
\usepackage{etoolbox}
\makeatletter
\patchcmd\longtable{\par}{\if@noskipsec\mbox{}\fi\par}{}{}
\makeatother
% Allow footnotes in longtable head/foot
\IfFileExists{footnotehyper.sty}{\usepackage{footnotehyper}}{\usepackage{footnote}}
\makesavenoteenv{longtable}
\usepackage{graphicx}
\makeatletter
\def\maxwidth{\ifdim\Gin@nat@width>\linewidth\linewidth\else\Gin@nat@width\fi}
\def\maxheight{\ifdim\Gin@nat@height>\textheight\textheight\else\Gin@nat@height\fi}
\makeatother
% Scale images if necessary, so that they will not overflow the page
% margins by default, and it is still possible to overwrite the defaults
% using explicit options in \includegraphics[width, height, ...]{}
\setkeys{Gin}{width=\maxwidth,height=\maxheight,keepaspectratio}
% Set default figure placement to htbp
\makeatletter
\def\fps@figure{htbp}
\makeatother
% definitions for citeproc citations
\NewDocumentCommand\citeproctext{}{}
\NewDocumentCommand\citeproc{mm}{%
  \begingroup\def\citeproctext{#2}\cite{#1}\endgroup}
\makeatletter
 % allow citations to break across lines
 \let\@cite@ofmt\@firstofone
 % avoid brackets around text for \cite:
 \def\@biblabel#1{}
 \def\@cite#1#2{{#1\if@tempswa , #2\fi}}
\makeatother
\newlength{\cslhangindent}
\setlength{\cslhangindent}{1.5em}
\newlength{\csllabelwidth}
\setlength{\csllabelwidth}{3em}
\newenvironment{CSLReferences}[2] % #1 hanging-indent, #2 entry-spacing
 {\begin{list}{}{%
  \setlength{\itemindent}{0pt}
  \setlength{\leftmargin}{0pt}
  \setlength{\parsep}{0pt}
  % turn on hanging indent if param 1 is 1
  \ifodd #1
   \setlength{\leftmargin}{\cslhangindent}
   \setlength{\itemindent}{-1\cslhangindent}
  \fi
  % set entry spacing
  \setlength{\itemsep}{#2\baselineskip}}}
 {\end{list}}
\usepackage{calc}
\newcommand{\CSLBlock}[1]{\hfill\break\parbox[t]{\linewidth}{\strut\ignorespaces#1\strut}}
\newcommand{\CSLLeftMargin}[1]{\parbox[t]{\csllabelwidth}{\strut#1\strut}}
\newcommand{\CSLRightInline}[1]{\parbox[t]{\linewidth - \csllabelwidth}{\strut#1\strut}}
\newcommand{\CSLIndent}[1]{\hspace{\cslhangindent}#1}

\usepackage{makeidx}
\makeindex
\KOMAoption{captions}{tableheading}
\makeatletter
\@ifpackageloaded{tcolorbox}{}{\usepackage[skins,breakable]{tcolorbox}}
\@ifpackageloaded{fontawesome5}{}{\usepackage{fontawesome5}}
\definecolor{quarto-callout-color}{HTML}{909090}
\definecolor{quarto-callout-note-color}{HTML}{0758E5}
\definecolor{quarto-callout-important-color}{HTML}{CC1914}
\definecolor{quarto-callout-warning-color}{HTML}{EB9113}
\definecolor{quarto-callout-tip-color}{HTML}{00A047}
\definecolor{quarto-callout-caution-color}{HTML}{FC5300}
\definecolor{quarto-callout-color-frame}{HTML}{acacac}
\definecolor{quarto-callout-note-color-frame}{HTML}{4582ec}
\definecolor{quarto-callout-important-color-frame}{HTML}{d9534f}
\definecolor{quarto-callout-warning-color-frame}{HTML}{f0ad4e}
\definecolor{quarto-callout-tip-color-frame}{HTML}{02b875}
\definecolor{quarto-callout-caution-color-frame}{HTML}{fd7e14}
\makeatother
\makeatletter
\@ifpackageloaded{bookmark}{}{\usepackage{bookmark}}
\makeatother
\makeatletter
\@ifpackageloaded{caption}{}{\usepackage{caption}}
\AtBeginDocument{%
\ifdefined\contentsname
  \renewcommand*\contentsname{Table of contents}
\else
  \newcommand\contentsname{Table of contents}
\fi
\ifdefined\listfigurename
  \renewcommand*\listfigurename{List of Figures}
\else
  \newcommand\listfigurename{List of Figures}
\fi
\ifdefined\listtablename
  \renewcommand*\listtablename{List of Tables}
\else
  \newcommand\listtablename{List of Tables}
\fi
\ifdefined\figurename
  \renewcommand*\figurename{Figure}
\else
  \newcommand\figurename{Figure}
\fi
\ifdefined\tablename
  \renewcommand*\tablename{Table}
\else
  \newcommand\tablename{Table}
\fi
}
\@ifpackageloaded{float}{}{\usepackage{float}}
\floatstyle{ruled}
\@ifundefined{c@chapter}{\newfloat{codelisting}{h}{lop}}{\newfloat{codelisting}{h}{lop}[chapter]}
\floatname{codelisting}{Listing}
\newcommand*\listoflistings{\listof{codelisting}{List of Listings}}
\makeatother
\makeatletter
\makeatother
\makeatletter
\@ifpackageloaded{caption}{}{\usepackage{caption}}
\@ifpackageloaded{subcaption}{}{\usepackage{subcaption}}
\makeatother

\ifLuaTeX
  \usepackage{selnolig}  % disable illegal ligatures
\fi
\usepackage{bookmark}

\IfFileExists{xurl.sty}{\usepackage{xurl}}{} % add URL line breaks if available
\urlstyle{same} % disable monospaced font for URLs
\hypersetup{
  pdftitle={Multilevel Thinking},
  pdfauthor={Andrew Grogan-Kaylor},
  colorlinks=true,
  linkcolor={blue},
  filecolor={Maroon},
  citecolor={Blue},
  urlcolor={Blue},
  pdfcreator={LaTeX via pandoc}}


\title{Multilevel Thinking}
\usepackage{etoolbox}
\makeatletter
\providecommand{\subtitle}[1]{% add subtitle to \maketitle
  \apptocmd{\@title}{\par {\large #1 \par}}{}{}
}
\makeatother
\subtitle{Discovering Diversity, Universals, and Particulars in
Cross-Cultural Research}
\author{Andrew Grogan-Kaylor}
\date{2024-10-08}

\begin{document}
\maketitle

\renewcommand*\contentsname{Table of contents}
{
\hypersetup{linkcolor=}
\setcounter{tocdepth}{2}
\tableofcontents
}
\listoffigures
\listoftables

\setstretch{2}
\bookmarksetup{startatroot}

\chapter{The Usefulness of Multilevel Modeling and Multilevel
Thinking}\label{sec-multilevel-thinking}

\begin{quote}
``I am because we are; and since we are, therefore I am.'' (Mbiti, 1970)
\end{quote}

For decades now, multilevel models have been an important quantitative
tool for social research. While multilevel models have become very
common in social research, there are aspects of these models that are
explored less frequently in published articles that appear in academic
journals. This book arises from my experiences of teaching a course
entitled \emph{Multilevel and Longitudinal Modeling} that I have taught
for over a decade in the \emph{Joint Doctoral Program in Social Work and
Social Science} at the University of Michigan.

The book started out as a set of notes on \emph{advanced things I only
get to discuss during breaks, or after class, or during office hours} in
my class on \emph{Multilevel and Longitudinal Modeling}, and grew from
that set of notes into a more general introduction to multilevel
modeling. My aim is to provide an intuitive introduction to multilevel
modeling.

My contention is that \emph{multilevel modeling} offers powerful tools
for understanding the \emph{multilevel data} that social researchers
often confront. For example, researchers are often interested in
studying outcomes for diverse groups of children in different schools,
residents of diverse and different neighborhoods, or individuals or
families living in diverse and different countries. Such inherently
multilevel data lead to analytic complexities, some of which appear to
me to be well understood, while others seem to be much less often
appreciated.

The point that I wish to make about multilevel data is that when
presented with complex multilevel data, failure to use the appropriate
multilevel model may lead to conclusions that are demonstrably
incorrect. Fortunately, many of these difficulties can be avoided with
applications of simple and straightforward multilevel models.
\index{correct answers} \index{less wrong}

I start by presenting some initial ideas about multilevel modeling.
First, as is relatively commonly understood, \emph{multilevel models
allow for the correct estimation of p values in the presence of data
clustering}. \index{p values} Second, as is less commonly appreciated,
when data are clustered, \emph{multilevel models correctly estimate
\(\beta\) regression coefficients and may avoid estimating a regression
coefficient that is too large, too small, or even has the wrong sign}.
\index{sign of coefficients}

I go on to explore some more complex ideas about multilevel models that
I see less often in the published empirical literature. I focus
especially on two ideas: \emph{multilevel models as the exploration of
diversity and variation across countries and cultures}; and
\emph{multilevel models as a foundation for models that let us think
more rigorously about causality}. I argue that multilevel models provide
a foundation for engaging with cross-cultural diversity in a
quantitatively rigorous fashion. \index{diversity}

Certainly, none of the statistical ideas contained in this book are
unique to me. There are thorough--and often much more mathematically
rigorous--presentations of many of the ideas contained in this book in
some of the excellent foundational texts on multilevel modeling such as
the early book by Raudenbush \& Bryk (2002), the excellent book on
longitudinal models by Singer \& Willett (2003), excellent books by
Snijders \& Bosker (2012) and Hox et al. (2018), and Rabe-Hesketh \&
Skrondal (2022)'s more recent and extremely comprehensive two volume
text. Luke (2004), and Kreft \& de Leeuw (1998), offer shorter, less
mathematically rigorous, but still excellent introductions to the topic
of multilevel modeling. Gelman et al. (2007) introduced me to the ideas
that in this book I describe as ``multilevel structure'' using an
example with voting patterns.

My intent in this book is to offer a kind of accessible tutorial for
applied researchers, including especially those who see their research
having some advocacy based component. My approach, while offering up
some equations, is less mathematically rigorous than some of the above
mentioned texts, and written with the intent of providing a clear and
practically focused guide for the applied researcher who is attempting
to carry out better research with diverse populations, particularly
research directed toward advocacy or social change.

\begin{quote}
``-- And so, why can't my numbers be beautiful to me? Why the scorn, the
doubt in your face? Do you think I am brittle and dusty as old paper?
Look again. See the numbers shine in my eyes.'' (Pye, 2011)
\index{Eveline Pye}
\end{quote}

\section{Complex Answers Hidden Inside Simple
Questions}\label{complex-answers-hidden-inside-simple-questions}

My history of teaching statistical methods has over time convinced me
that statistical questions are often seemingly simple questions, that
contain complex answers, or answers that have complex
operationalizations.

One thing that I often say in teaching is that because so many of the
outcomes we study are so important--and are often so unequally
allocated--we want to make sure our answers are as precise, and as close
to correct, as we can make them. As I detail elsewhere in this book, it
turns out that failing to understand some of the hidden complexities of
statistical thinking may lead to providing very wrong answers to
important questions. I explore these complexities of multilevel thinking
in this book.

Two images that I carry with me, and that help me convey this idea are
the nautilus shell, and the Mandelbrot set. \index{nautilus shell}
\index{Mandelbrot set} \index{complex answers to simple questions}

\begin{figure}

\begin{minipage}{0.50\linewidth}

\begin{figure}[H]

{\centering \includegraphics[width=\textwidth,height=3in]{nautilus-OUP.jpg}

}

\subcaption{The Nautilus Shell}

\end{figure}%

\end{minipage}%
%
\begin{minipage}{0.50\linewidth}

\begin{figure}[H]

{\centering \includegraphics[width=\textwidth,height=3in]{mandelbrot.png}

}

\subcaption{Image of the Mandelbrot Set, produced with
\texttt{mandelbrot} by Moore \& dos Reis (2017)}

\end{figure}%

\end{minipage}%
\newline
\begin{minipage}{0.50\linewidth}
Complex structures produced with simple rules\end{minipage}%

\end{figure}%

Just as each of these images is a complex outcome that derives from a
simple set of rules, complex situations emerge from applying statistical
ideas, that are initially simple in principle. The nautilus shell is
based upon a logarithmic spiral, a spiral that as it grows outward
maintains a constant angle with the center (Livio, 2002). The Mandlebrot
set is a complex outcome that results from iterating a simple rule over
and over (Moore \& dos Reis, 2017).

\bookmarksetup{startatroot}

\chapter*{Acknowledgements}\label{acknowledgements}
\addcontentsline{toc}{chapter}{Acknowledgements}

\markboth{Acknowledgements}{Acknowledgements}

No good learning happens without community. At least, that has always
been true for me. I am grateful for many creative and energizing
discussions with the other members of the MICS (UNICEF data) research
team: Professor Julie Ma, Dr.~Kaitlin Ward, Professor Garrett Pace and
Professor Shawna Lee. I'm thankful for their collegiality, their
friendship, their patience with me, and their dedication to good
science. Working with these colleagues has greatly deepened my
understanding of multilevel models. I'm also very grateful to one of my
mentors, Professor Sandra Danziger, who has taught me so much about the
\emph{what} and the \emph{why} of mentoring, teaching, and doing
research. My colleagues Professor Karen Staller, and Professor Julie
Ribaudo have also helped me clarify my thinking about the \emph{what}
and the \emph{why} of research and teaching, and have encouraged me
along the way. Don Deutsch showed ongoing interest in the development of
this book, and has asked some hard questions that have improved its
logic. Thank you to the entire \emph{Qungasvik} team for their
inspiration and support. Thank you to Professor Jim Allen for our
discussions about how quantitative methods can better listen to lived
experience. Thank you to Guillermo Flores for frequently inquiring about
the progress of this work. Importantly, I'd like to express gratitude to
the many students in my class on \emph{Multilevel and Longitudinal
Modeling} who over the years have helped me think more deeply about
statistical and substantive issues, including Dr.~Kaitlin Ward,
Professor Garrett Pace, Professor Julie Ma, Professor Yoonsun Han,
Professor Berenice Castillo, Professor Maria Galano, Dr.~Sara Stein,
Madhur Singh, and Tong Suo. Thank you for the many valuable discussions
during class breaks, and after class. These discussions have also
greatly extended my understanding of multilevel models. I'd like to
thank Marty Betts and Greg Knollmeyer for their help and support. Most
importantly, I have two final people to thank: I'd like to thank Ross
Grogan-Kaylor for continued interest in the progress of this book,
scrupulous attention to detail, and many probing thoughtful questions
that helped the logic of the presentation. Shari Grogan-Kaylor had many
important and challenging questions about the \emph{what} and the
\emph{why} of this book, and most importantly to me, \emph{believed} in
it. While I'm thankful for the inspiration and support provided by
others, any remaining errors and omissions in this book are of course my
responsibility.

\bookmarksetup{startatroot}

\chapter*{License and Citation}\label{license-and-citation}
\addcontentsline{toc}{chapter}{License and Citation}

\markboth{License and Citation}{License and Citation}

\section*{License}\label{license}
\addcontentsline{toc}{section}{License}

\markright{License}

\includegraphics[width=0.29in,height=\textheight]{88x31.png}

\emph{Multilevel Thinking} by \href{https://agrogan1.github.io/}{Andrew
Grogan-Kaylor} is licensed under a
\href{http://creativecommons.org/licenses/by/4.0/}{Creative Commons
Attribution 4.0 International License}

\section*{Citation}\label{citation}
\addcontentsline{toc}{section}{Citation}

\markright{Citation}

For attribution, please cite as:

Grogan-Kaylor (2024). \emph{Multilevel Thinking}. Retrieved from
\url{https://agrogan1.github.io/multilevel-thinking}

\section*{BibTeX Citation}\label{bibtex-citation}
\addcontentsline{toc}{section}{BibTeX Citation}

\markright{BibTeX Citation}

\begin{Shaded}
\begin{Highlighting}[]
\NormalTok{@book\{,}
\NormalTok{   author = \{Andrew Grogan{-}Kaylor\},}
\NormalTok{   city = \{Ann Arbor, MI\},}
\NormalTok{   title = \{Multilevel Thinking\},}
\NormalTok{   url = \{https://agrogan1.github.io/multilevel{-}thinking/\},}
\NormalTok{   year = \{2024\},}
\NormalTok{\}}
\end{Highlighting}
\end{Shaded}

\bookmarksetup{startatroot}

\chapter*{Some Preliminary Thoughts}\label{some-preliminary-thoughts}
\addcontentsline{toc}{chapter}{Some Preliminary Thoughts}

\markboth{Some Preliminary Thoughts}{Some Preliminary Thoughts}

\begin{quote}
``Like you I

Love love, life, the sweet smell of things, the sky-blue landscape of
January days.

\ldots{}

I believe the world is beautiful.

And that poetry like bread, is for everyone.

And that my veins don't end in me.

But in the unanimous blood.

Of those who struggle for life,

Love, little things,

Landscape and bread, the poetry of everyone.'' \index{universalism}
\index{variation and diversity} \index{Roque Dalton}
\end{quote}

--- (Dalton, 2000) (translated By Jack Hirschman)

\(~\)

\begin{quote}
``A lifetime is too narrow to understand it all, beginning with the huge
rockshelves that underlie all that life.

No one ever told us we had to study our lives, make of our lives a
study, as if learning natural history or music, that we should begin
with the simple exercises first and slowly go on trying the hard ones,
practicing till strength and accuracy became one with the daring
\ldots{}

But there come times---perhaps this is one of them---when we have to
take ourselves more seriously or die, when we have to pull back from the
incantations, rhythms we've moved to thoughtlessly, and disenthrall
ourselves, bestow ourselves to silence, or a severer listening \ldots''
\index{listening} \index{Adrienne Rich}
\end{quote}

--- (Rich, 1984)

\(~\)

\begin{quote}
``Research is formalized curiosity. It is poking and prying with a
purpose.'' \index{Zora Neale Hurston}
\end{quote}

--- (Hurston, 1942)

\bookmarksetup{startatroot}

\chapter{Introduction}\label{introduction}

\begin{quote}
I don't take your words

Merely as words.

Far from it.

I listen

to what makes you talk--

Whatever that is--

And me listen. (Takahashi, 2000) \index{Shinkichi Takahashi}
\end{quote}

\begin{quote}
``Listening to the world. Well, I did that, and I still do it. I still
do it.'' (Mary Oliver in Oliver \& Tippett, 2015) \index{listening}
\index{Mary Oliver}
\end{quote}

\section{Quantitative Methods and Social
Justice}\label{sec-socialjustice}

There is clearly need for both qualitative and quantitative methods.
\index{quantitative and qualitative methods} Central to the argument of
this book is the idea that advanced quantitative methods can be core
contributors to the agenda of understanding issues of diversity and
social justice more fully and thoroughly (Cokley \& Awad, 2013;
Grogan-Kaylor et al., 2018). Quantitative methods, particularly in
discussions comparing qualitative and quantitative methodologies, are
sometimes labelled as inherently \emph{positivist} methods.
\index{positivist} My argument regarding this point is twofold. First,
there is nothing within the mathematics of quantitative methods that
requires a positivist epistemology. Quantitative methodologies could as
easily be conducted using a critical epistemology--that is aware of
dynamics of power and privilege--as any other methodology (Scharrer \&
Ramasubramanian, 2021; Stage \& Wells, 2014).
\index{quantitative literacy} \index{quantitative criticalism} I note
that one of the pioneers of liberation psychology, Martín-Baró (Aron \&
Corne, 1994), used both qualitative and quantitative methods
(Martin-Baro, 1994a), including in the latter case, relatively
sophisticated arguments about patterns of missing data across a survey
data set (Aron \& Corne, 1994). \index{liberation psychology}
\index{Ignacio Martín-Baró}

Second, when we have samples of a hundred, several hundred, several
thousand, or even hundreds of thousands of study participants
distributed across multiple and diverse social contexts, it is difficult
to imagine a methodology other than a \emph{quantitative} methodology
that could accomplish the following:

\begin{enumerate}
\def\labelenumi{\arabic{enumi}.}
\tightlist
\item
  Sift through thousands of responses, and determine the \emph{overall,
  or average, pattern of relationships} between risk factors, protective
  factors, and outcomes.
\item
  Determine whether there is evidence that the relationships observed
  within the data are more than \emph{statistical noise}.
\item
  Adjudicate the \emph{complex multivariate relationships} of risk
  factors, protective factors and outcomes, while controlling for
  possible confounding variables, contextual variables, or background
  variables.
\end{enumerate}

Additionally, it is difficult to imagine a methodology other than a
\emph{quantitative multilevel} methodology that could accomplish the
above 3 goals, and could additionally:

\begin{enumerate}
\def\labelenumi{\arabic{enumi}.}
\setcounter{enumi}{3}
\tightlist
\item
  Explore the \emph{diversity and variation and commonalities in these
  relationships} across social contexts. \index{diversity}
\end{enumerate}

Therefore, I consider multilevel modeling to be a principled
quantitative method for \emph{listening} to the voices of large numbers
of study participants across social contexts. \index{listening} Since
graduate school, I have been inspired by the idea of ``hearing the other
into speech'' (Morton, 1972). I believe that when conducted with a
critical lens, appropriate multilevel methods can be a way of ``hearing
others into speech'', even with very large data sets.

Liberation psychology often contains a focus on analyzing society from
the \emph{bottom up} (Montero \& Sonn, 2009). According to the
liberation psychology perspective, social change in particular is often
thought to be most effective when it comes from the bottom up (Montero
\& Sonn, 2009). I have not seen this connection made elsewhere, but I
have come to believe that by aggregating hundreds, thousands, or even
hundreds of thousands of individual responses, and by accounting for,
and modeling, the diversity and variation in these responses, that
multilevel modeling can be seen as congruent with the liberatory impulse
of working from the bottom up. \index{liberation psychology}

In Section~\ref{sec-Simpsons}, I consider the way that issues of omitted
variables--an often under-appreciated issue--contribute to the
difficulty of obtaining \emph{correct} answers in all quantitative work.
Dichotomous dependent variables, discussed in
Chapter~\ref{sec-logistic}, also add additional complications to the
process of obtaining \emph{correct} answers in any kind of regression
modeling. In Section~\ref{sec-pvalues}, where I consider the estimation
of p values, and Section~\ref{sec-multilevelstructure}, where I consider
the signs of regression coefficients, I explore the ways that
\emph{multilevel data} can contribute substantially to the complexity of
the analysis of data. This complexity of \emph{multilevel data} means
that unless one employs appropriately sophisticated \emph{multilevel
analysis} methods with \emph{multilevel data}, one runs a relatively
high risk of making \emph{incorrect substantive conclusions}. I thus
argue that advanced quantitative methods, like multilevel modeling, can
play an important role in helping one to be progressively \emph{less
wrong} over time, and in contributing to liberatory ideas and to social
justice. \index{less wrong}

There is an ethical argument that is embedded in this book. Many of us
do research with the hope of better understanding the relationship of
risk and protective factors with outcomes in diverse, and often
disadvantaged or marginalized, populations. Many of us further hope that
our work might be part of conversations about appropriate policies,
programs, treatments or interventions. Given the frequent vulnerability
and marginalization of the people with whom we work, when using
quantitative methods, it is incumbent upon us to employ methods that
adequately address the complexities of the data, that offer an
appreciation of the variability and diversity within the data, that
provide the most accurate and unbiased estimates possible, and that
increase the probability of obtaining \emph{correct} answers to
important substantive questions.

\begin{quote}
``It is hard to imagine that anyone with a humanitarian worldview would
argue against the need for a more quantitatively literate citizenry.
Informed political decision-making, retirement planning, active
parenting, and the vast majority of choices we make in our personal,
occupational, and civic lives can be better served by improved
quantitative understanding and reasoning, as well as accompanying
action-oriented dispositions.'' (Wiest et al., 2007)
\end{quote}

The idea of this book is that a deeper study of multilevel modeling can
result in an advanced ``quantitative literacy'' (Wiest et al., 2007),
\index{quantitative literacy} ``quantitative criticalism'' (Scharrer \&
Ramasubramanian, 2021), \index{quantitative criticalism} ``critical
quantitative inquiry'' (Stage \& Wells, 2014),
\index{critical quantitative inquiry} or ``principled argument''
(Abelson, 1995), \index{principled argument} that is appropriate for
drawing accurate conclusions from multilevel data.

\section{Some Philosophy of Science}\label{sec-science}

I am not much of a philosopher of science. \index{philosophy of science}
However, I am very persuaded by Strevens' (2020) minimalist criterion of
the ``iron rule''. In essence, this rule specifies that to count as
``science'', investigations must engage in ``performing an experiment or
making an observation that generates relevant empirical evidence''
against which competing hypotheses can be tested. A similar perspective
is offered by Goldacre (2011) who argues that ideas about interventions
should be scrutinized with a ``fair test''. That is to say, they should
be tested against evidence that can support or refute those ideas. I
would argue all ideas about promoting human well-being should be able to
be subjected to such a ``fair test''.

I believe that our work---whether qualitative, or quantitative---should
strive to be both critical \emph{and} scientific, in the sense that: our
research should gather evidence; that evidence should be assessed in
order to support, refute, or modify our initial beliefs; and that
evidence should be used to think critically about human wellbeing,
including dynamics of power and privilege and disparities. With regard
to this idea, Shrader-Frechette (2014) suggests that a ``practical
philosophy of science'' can contribute both to ``speaking truth to
power'' and to ``seeking justice''.

Scientific and quantitative thinking are sometimes construed as a
\emph{Western}, or even \emph{colonial} activities.
\index{philosophy of science} In contrast, I believe and argue that
science and quantitative thinking are ways of thinking that are useful
for all of us, whenever we encounter concrete realities, particularly
when we desire to confront or address inequities or challenges of those
concrete realities. In this light, science and quantitative reasoning
might be seen as pan-human cross-cultural activities, allowing human
beings to fruitfully engage with the physical world. In a set of rich
descriptions, Sagan (1995) describes the thinking of the !Kung San,
\index{"!"Kung San} who use detailed careful and repeated observation of
the natural world to construct empirical knowledge that allows them to
navigate and survive in the harsh environment of the Kalahari desert.
Far from being the domain exclusively of one group of people, Sagan
(1995) argues that science is a quintessentially human
activity\footnote{I'm certainly aware that science--like \emph{any}
  domain of human thinking--can be misused, and has been misused. The
  only way to prevent this, I believe, is to couple scientific thinking
  with rigorous moral reasoning. But that is a subject for a different
  book!}.

Similarly, the Islamic scientist Alhazen (2023), often seen as one of
the first exponents of scientific, thinking, described the scientific
method in a text from 1000 A.D. The language is unfortunately gendered,
which can perhaps be understood given the era of the writing.
\index{Alhazen}

\begin{quote}
``The seeker after the truth\ldots{} is not he who studies the writings
of the ancients and\ldots{} puts his trust in them, but rather the one
who suspects his faith in them and questions what he gathers from them,
the one who submits to argument and demonstration, and not to the
sayings of a human being whose nature is fraught with all kinds of
imperfection and deficiency.'' (Alhazen, 2023)
\end{quote}

In the approach that I am taking in this book, I believe that there are
definite \emph{material realities} of suffering, exploitation, violence,
discrimination, and other associated problems that we are trying to
understand.

\begin{quote}
``What we see and how we see is of course determined by our perspective,
by the place from which we begin our examination of history; but it is
determined also by reality itself.'' (Martin-Baro, 1994b)
\index{Ignacio Martín-Baró}
\end{quote}

We hope that our understandings will inform efforts for social change.
Yet, at the same time we must recognize that our understandings are at
best \emph{iterative} and \emph{contingent.} While we will never have a
perfect understanding of social reality, we can always improve our
understandings, and move in the direction of being \emph{less wrong}.
\index{less wrong}

\begin{quote}
``\ldots{} there is no way to know when our observations about complex
events in nature are complete. Our knowledge is finite, Karl Popper
emphasised, but our ignorance is infinite. \ldots{} {[}W{]}e can never
be certain about the consequences of our interventions, we can only
narrow the area of uncertainty. This admission is not as pessimistic as
it sounds: claims that resist repeated energetic challenges often turn
out to be quite reliable. Such `working truths' are the building blocks
for the reasonably solid structures that support our everyday
actions\ldots{}'' (Silverman, 1998) \index{Karl Popper}
\end{quote}

This recalls the famous saying by the statistician George Box about
statistical models, reported in many places, and well captured in the
passage below by Hand (2014):

\begin{quote}
``In general, when building statistical models, we must not forget that
the aim is to understand something about the real world. Or predict,
choose an action, make a decision, summarize evidence, and so on, but
always about the real world, not an abstract mathematical world: our
models are not the reality---a point well made by George Box in his
oft-cited remark that `all models are wrong, but some are useful' (Box,
1979 in Launer \& Wilkinson (1979)).'' (Hand, 2014) \index{George Box}
\end{quote}

A key task then, of using advanced quantitative methods such as
multilevel modeling, is to use them to try to be progressively
\emph{less wrong} about the answers we are finding to important
questions about improving human wellbeing. \index{less wrong}

To reiterate a point made earlier in this chapter, there are many
potential complications in data analysis. Omitted variables are an issue
that trouble all quantitative research (Section~\ref{sec-Simpsons}).
Multilevel data provides additional complications in that statistical
significance (p values) becomes more difficult to estimate correctly
(making false positives more likely) (Section~\ref{sec-pvalues}), and
without use of appropriate methods, the signs and magnitudes of
regression coefficients (\(\beta\)'s) may be wrong
(Section~\ref{sec-multilevelstructure}).

Correct application of multilevel models can help us to be \emph{less
wrong} about all of these issues, and to come closer to providing
helpful answers to the people we are working with. \index{less wrong}

\section{A Pragmatic Approach}\label{a-pragmatic-approach}

This book will discuss the ways in which a multilevel statistical
perspective not only allows one to appropriately analyze cross cultural
or international data, but also the ways in which a multilevel
perspective affords the opportunity for more precise quantitative
thinking about cross cultural phenomena. The book takes a very pragmatic
and very advocacy oriented approach to improving research.
\index{pragmatic approach} \index{theory}

\begin{quote}
``It shouldn't be theories that define the problems of our situation,
but rather the problems that demand, and so to speak, select, their own
theorisation.'' (Martin-Baro (1998) in Burton \& Kagan (2005)).
\end{quote}

\begin{quote}
``\ldots{} philosophy can only carry out its critical and creative
function in favor of an effective praxis of liberation if it is
adequately situated within that liberating praxis, which is independent
of philosophy.'' (Ellacuria, 2013) \index{Ignacio Ellacuría}
\end{quote}

Following from this pragmatic and advocacy oriented emphasis, the book
is largely oriented to the \emph{doing} of quantitative social research
with multilevel (or multi-country) data, and is therefore mostly
statistical in nature.

The book moves quickly into detailed statistical arguments. Some of
these statistical discussions may seem very technical, or even overly
technical. However, an overarching theme of the book is that multilevel
data contains hidden complexities. A lack of awareness of the
complexities of multilevel data---e.g.~complexities of multi-country
data---might lead to statistical analyses that point in the wrong
direction: yielding false positives; false negatives; or substantively
wrong conclusions.

\section{Are Answers from Social Science
``Obvious''?}\label{sec-obvious}

Closely related, I think to the idea that quantitative research can
advance issues of social justice, is the question of whether answers
from social science are ``obvious''. If social science answers are
obvious, then social science has limited abilities to make new
discoveries, and to build scientific foundations for evidence. In
contrast, if answers from social science are sometimes not obvious, then
social science has a greater ability to make new discoveries and build
new foundations for evidence.

I have been thinking a lot about the idea that \emph{Everything Is
Obvious, Once You Know The Answer}, as detailed in the book with this
title by Duncan Watts (2011). \index{everything is obvious}

This seems to me especially true in social research. Arguably, some
conclusions of social research may indeed be obvious. For example, it
may be obvious that \emph{Adverse Childhood Experiences} (ACEs) are
associated with long term decreases in mental health. However, even
obvious conclusions may need to be quantitatively documented, in order
to legitimate programs and interventions, and to secure funding. I also
observe that I think that there is often a \emph{historical} dimension
to what is considered ``obvious'': conclusions that are at first
considered to be unlikely to be true, or even counter-intuitive, require
the weight of accumulating evidence over time for these connections to
become ``obvious''. It is likely that the ``obviousness'' of the
relationship between ACEs and later physical and mental health problems
did not become apparent until research began to document these
relationships (e.g. Felitti et al. (1998)).

As another example, Proctor (2012) documents the way which smoking was
first considered to be an \emph{unlikely} cause of lung cancer; only
over the course of several decades of research and discussion to become
an \emph{obvious} cause of lung cancer. A similar \emph{historical}
dynamic seems to be playing out in some research on parenting and child
development. Despite decades of evidence indicating that corporal
punishment has undesirable consequences for children (Gershoff \&
Grogan-Kaylor, 2016b), corporal punishment remains a disciplinary
strategy endorsed by the majority of the American population (Hines et
al., 2022).

In contrast sometimes the conclusions of social research may not always
be obvious. For example:

\begin{enumerate}
\def\labelenumi{\arabic{enumi}.}
\tightlist
\item
  There has been an ongoing debate about whether corporal punishment is
  more or less harmful when used by parents in social contexts, or
  communities where it is more common, or normative, or in contexts that
  are disadvantaged. Eamon (2001) suggested that ``when environmental
  risk is high, parenting practices that are firmer and higher in
  control result in lower levels of young adolescent antisocial
  behavior.'' This echoes similar research by (Deater-Deckard et al.,
  1996) suggesting that physical punishment was harmful for
  European-American children, but not for African-American children.
  Later, larger sample research has found that this appears not to be
  the case: physical punishment is harmful for children in \emph{all}
  groups (Gershoff \& Grogan-Kaylor, 2016b, 2016a; Pace et al., 2019).
\item
  Using MICS Data (UNICEF, 2024), we conducted a study of the link
  between gender inequality and physical child abuse (Ma et al., 2022).
  We expected to find that higher levels of gender inequality led to
  higher levels of physical abuse for female children, but not for male
  children. Instead, we found that higher levels of gender inequality
  were associated with higher levels of physical abuse for \emph{both}
  male and female children. Additionally, there was some slight evidence
  that male children were at higher risk of being abused than female
  children. Equally interesting was that we found that gender inequality
  was predictive of levels of child abuse, while country level GDP was
  not.
\item
  In a study of parenting during Covid-19 (Lee et al., 2022), we
  expected to find that households with children would experience
  \emph{higher} levels of anxiety and depression than households without
  children. Instead, we found the opposite. Being in a household with
  children was generally \emph{protective} against anxiety and
  depression.
\end{enumerate}

In Section~\ref{sec-studyvariation}, Section~\ref{sec-pvalues} and
Section~\ref{sec-multilevelstructure}, I provide specific examples of
how multilevel data provides even more opportunity to present answers
that are \emph{not} obvious.

\section{Presenting Advanced Statistical
Ideas}\label{presenting-advanced-statistical-ideas}

In presenting advanced, statistical concepts, one is faced with a
quandary. One can present statistical concepts in the most general
terms, in terms of \emph{x} and \emph{y}. While perhaps the
mathematically most general way to present ideas, a highly general (and
abstract) presentation risks not being a good way of teaching the ideas,
as it is sometimes difficult to apply abstract ideas to one's own
specific area of research.

Alternatively, one can present statistical ideas in terms of specific
substantive concepts. The risk of making use of a specific substantive
concept is that while concrete examples are always helpful, it may be
difficult for the reader to generalize from a specific example to their
own area of research.

I ground this presentation in research that we have conducted on
parenting and child development in international context (Grogan-Kaylor
et al., 2021; Ma et al., 2022; Pace et al., 2019; Ward, Grogan-Kaylor,
Pace, et al., 2021; Ward et al., 2022; Ward et al., 2023). For the
presentation in this book, I use simulated data on these issues.

Using the simulated data, I refer to \emph{predictors} and
\emph{outcomes}, and explore the ways that the multilevel model can
contribute to understanding how relationships between predictors and
outcomes might be similar, or might be different, across \emph{social
contexts}. In the examples presented below, I focus on two predictors,
parental \emph{warmth}, and parental use of \emph{physical punishment}
and focus on the \emph{outcome} of \emph{improved} mental health. I use
the social context of different \emph{countries} in our example.

It is my belief that while I use this specific set of examples, that the
idea of studying \emph{families in different countries} is generalizable
enough to a multiplicity of diverse contexts, such that the reader can
apply these ideas to their own area of interest, whether that be
\emph{children in schools}; \emph{residents in neighborhoods}; or
\emph{people in different countries}.

\section{Research on Parenting and Child Development in International
Context}\label{research-on-parenting-and-child-development-in-international-context}

Research on parenting and child development has identified robust
associations between parenting behaviors and child developmental
outcomes. Broadly speaking, physical punishment is associated with
increases in child aggression, child anxiety and child mental health
problems (Gershoff \& Grogan-Kaylor, 2016b), while warm and supportive
parenting is associated with decreases in these outcomes (Khaleque \&
Rohner, 2002; Rothenberg et al., 2022). However, much of this research
is conducted on North American samples (Draper et al., 2022; Henrich et
al., 2010).

Barth \& Olsen (2020) have argued that children constitute a class of
oppressed persons. If children are oppressed, then it is imperative to
empirically determine what factors are promotive of children's
well-being, and what factors constitute risk factors that contribute to
decreases in children's well-being. Equally imperative--given the North
American focus of so much research on parenting and child development
(Draper et al., 2022; Henrich et al., 2010)--would be efforts to extend
the study of parenting and child development to a broader, more global
context. As part of such a research agenda, it is necessary to have
quantitative tools that are able to determine the consistency of
relationships in parenting and child development. That is, are the
relationships between certain forms of parenting and child developmental
outcomes, largely consistent across countries, largely different across
countries, or somewhere in between?

\section{Universalism And
Particularity}\label{universalism-and-particularity}

\begin{quote}
``My conception of the universal is that of a universal enriched by all
that is particular, a universal enriched by every particular: the
deepening and coexistence of all particulars.'' (Cesaire, 1956)
\index{Aimé Césaire}
\end{quote}

\begin{quote}
``Despite the incredible diversity existing among and within human
cultures, there are many phenomena that occur regularly in all known
societies. These commonalities, or universals, while deriving in part
from human nature, may also have specific social, cultural, and systemic
sources. We need to develop a working understanding of these universals
so that we might advance legitimate, empirically based human science set
on creating knowledge that is politically relevant to fostering real
solutions to the problems that complicate human co-existence in the Age
of the Anthropocene.'' (Antweiler, 2016)
\end{quote}

The specific domain of cross-cultural research on parenting and child
development raises more general questions in cross-cultural research of
\emph{universalism} and \emph{particularity}. \index{universalism} With
regard to child development it is universal that all children need some
amount of emotional and material care to grow into healthy youth and
healthy adults (Kottak, 2021). Further it is broadly understood that
children should be protected from violence (UNICEF, 2014). This broad
consensus is manifested in such documents as the Convention on the
Rights of the Child (United Nations General Assembly, 1989) and the
United Nations Sustainable Development Goals (United Nations, 2022),
representing global efforts to ensure the children are cared for, and
are protected against violence.

At the same time, broad international efforts to improve children's
well-being must engage with important considerations of cultural
uniqueness. Put simply, parenting practices may vary widely between
cultural groups (Gottlieb, 2002). Further, what is considered to be
beneficial for children in one country or culture may not be considered
to be beneficial in all countries or cultures. Similarly, what is
considered to be detrimental in one country or culture may not equally
be considered to be detrimental in all. Within the area of parenting and
child development, most of the debate has focused around the question of
whether physical punishment is equally detrimental in all settings,
particularly whether physical punishment is detrimental in countries
where it is especially common, or normative (Gershoff et al., 2010).
Much less attention has been focused on the study of positive parenting
internationally, and the degree to which the outcomes of positive
parenting are consistent across countries remains understudied (Ward,
Grogan-Kaylor, Ma, et al., 2021).

However, as global initiatives to improve child well-being and family
life move forward, it becomes increasingly important to continue to
collect internationally relevant data about parenting and child
outcomes. If recommendations are to be made for policies, interventions,
or treatments, such recommendations must be based on accurate balancing
of that which is universal against that which is unique to particular
cultural contexts. Thus it is necessary to employ statistical methods
that are able to adequately and accurately analyze data across
countries.

As I will outline below--and is evident in the literature (Hox et al.,
2018; Kreft \& de Leeuw, 1998; Luke, 2004; Rabe-Hesketh \& Skrondal,
2022; Raudenbush \& Bryk, 2002; Singer \& Willett, 2003; Snijders \&
Bosker, 2012)--multilevel models are eminently suited for cross-cultural
research in that they are not only able to \emph{control for} the
clustering of study participants within countries, but are also able to
\emph{explore the variation}--or \emph{consistency}--of patterns of
social life across countries.

\bookmarksetup{startatroot}

\chapter{Simulated Multi-Country (Multilevel)
Data}\label{sec-simulateddata}

\begin{figure}

\centering{

\includegraphics[width=0.5\textwidth,height=\textheight]{world.png}

}

\caption{\label{fig-world}Countries of the World}

\end{figure}%

\begin{quote}
``\ldots{} the particular and the universal are not to be seen as
opposites, \ldots{} the universal is not the negation of the particular
but is reached by a deeper exploration of the particular.'' (Cesaire in
UNESCO (1997)) \index{universalism} \index{Aimé Césaire}
\end{quote}

I use simulated data in this example. Data come from 30 hypothetical
countries. \texttt{country} is a numeric variable ranging from 1 to
30.\footnote{The \texttt{country} variable demonstrates an important
  point. Data for a grouping, nesting, or clustering variable does
  \emph{not} need to identify the actual groups by name, but only needs
  to distinguish the groups from one another. Put another way, the data
  for the grouping variable can be anonymous about the actual identities
  of the grouping variable.} Data contain measures of a few key aspects
of parenting\footnote{I use the term parenting throughout this book, but
  am aware that such parenting may come from biological parents, or from
  other caregivers.} or caregiving that have proven salient in the
empirical literature on parenting to date: parental \texttt{warmth}, and
\texttt{physical\ punishment}. Both parenting measures are normally
distributed variables, and are considered to be \emph{Level 1}, or
\emph{individual level} variables.

\texttt{identity} is a hypothetical designation of an identity, such as
a race, ethnicity, or gender identity. In this simulated data, identity
has two categories--for ease of presentation--but could easily be a more
than two category variable. \texttt{identity} is also a \emph{Level 1}
variable.

Many readers will be interested in using multilevel models to evaluate
interventions. The variable \texttt{intervention} represents a program,
treatment or intervention to which study participants have been
assigned. \texttt{intervention} is a \emph{Level 1} variable. Assignment
to interventions may or may not be random, a topic which is considered
in more detail in Section~\ref{sec-causality}.

\texttt{HDI} is a measure of the \emph{Human Development Index} (United
Nations Development Program, 2022), and is measured at the \emph{country
level}, or \emph{Level 2}. (I discuss more in depth thinking about
levels of the data in Chapter~\ref{sec-conceptualframework}.)

Our \texttt{outcome} is conceptualized as a positive mental health
outcome or behavioral outcome, and higher levels of \texttt{outcome} are
considered to be better. Statistically, the data are clustered within
countries.

\begin{tcolorbox}[enhanced jigsaw, breakable, colbacktitle=quarto-callout-note-color!10!white, leftrule=.75mm, bottomrule=.15mm, opacityback=0, colframe=quarto-callout-note-color-frame, coltitle=black, rightrule=.15mm, title=\textcolor{quarto-callout-note-color}{\faInfo}\hspace{0.5em}{Download The Data}, colback=white, opacitybacktitle=0.6, toprule=.15mm, titlerule=0mm, bottomtitle=1mm, left=2mm, toptitle=1mm, arc=.35mm]

Data are presented in Stata format. The Appendix considers the analysis
of multilevel models using multiple software packages: Stata, R \&
Julia, but Stata format is used to store the data as it can be read by
each of these software packages.

\begin{itemize}
\tightlist
\item
  \href{https://github.com/agrogan1/multilevel-thinking/raw/main/simulate-and-analyze-multilevel-data/simulated_multilevel_data.dta}{Cross-Sectional
  Data}
\item
  \href{https://github.com/agrogan1/multilevel-thinking/raw/main/simulate-and-analyze-multilevel-data/simulated_multilevel_longitudinal_data.dta}{Longitudinal
  Data}
\end{itemize}

\end{tcolorbox}

In this simulation, I construct the data so that \texttt{warmth} is
positively related to the \texttt{outcome}, while
\texttt{physical\ punishment} is negatively related to the
\texttt{outcome}.

\begin{longtable}[]{@{}
  >{\centering\arraybackslash}p{(\columnwidth - 12\tabcolsep) * \real{0.0759}}
  >{\centering\arraybackslash}p{(\columnwidth - 12\tabcolsep) * \real{0.1266}}
  >{\centering\arraybackslash}p{(\columnwidth - 12\tabcolsep) * \real{0.1139}}
  >{\centering\arraybackslash}p{(\columnwidth - 12\tabcolsep) * \real{0.2785}}
  >{\centering\arraybackslash}p{(\columnwidth - 12\tabcolsep) * \real{0.1392}}
  >{\centering\arraybackslash}p{(\columnwidth - 12\tabcolsep) * \real{0.1899}}
  >{\centering\arraybackslash}p{(\columnwidth - 12\tabcolsep) * \real{0.0759}}@{}}

\caption{\label{tbl-simulateddata}Simulated Multilevel Data}

\tabularnewline

\caption{Table continues below}\tabularnewline
\toprule\noalign{}
\begin{minipage}[b]{\linewidth}\centering
id
\end{minipage} & \begin{minipage}[b]{\linewidth}\centering
country
\end{minipage} & \begin{minipage}[b]{\linewidth}\centering
warmth
\end{minipage} & \begin{minipage}[b]{\linewidth}\centering
physical\_punishment
\end{minipage} & \begin{minipage}[b]{\linewidth}\centering
identity
\end{minipage} & \begin{minipage}[b]{\linewidth}\centering
intervention
\end{minipage} & \begin{minipage}[b]{\linewidth}\centering
HDI
\end{minipage} \\
\midrule\noalign{}
\endfirsthead
\toprule\noalign{}
\begin{minipage}[b]{\linewidth}\centering
id
\end{minipage} & \begin{minipage}[b]{\linewidth}\centering
country
\end{minipage} & \begin{minipage}[b]{\linewidth}\centering
warmth
\end{minipage} & \begin{minipage}[b]{\linewidth}\centering
physical\_punishment
\end{minipage} & \begin{minipage}[b]{\linewidth}\centering
identity
\end{minipage} & \begin{minipage}[b]{\linewidth}\centering
intervention
\end{minipage} & \begin{minipage}[b]{\linewidth}\centering
HDI
\end{minipage} \\
\midrule\noalign{}
\endhead
\bottomrule\noalign{}
\endlastfoot
1.1 & 1 & 3 & 3 & 1 & 0 & 69 \\
1.2 & 1 & 1 & 2 & 1 & 1 & 69 \\
1.3 & 1 & 2 & 3 & 0 & 1 & 69 \\
1.4 & 1 & 5 & 0 & 1 & 0 & 69 \\
1.5 & 1 & 4 & 4 & 1 & 0 & 69 \\
1.6 & 1 & 3 & 5 & 0 & 1 & 69 \\

\end{longtable}

\begin{longtable}[]{@{}
  >{\centering\arraybackslash}p{(\columnwidth - 0\tabcolsep) * \real{0.1389}}@{}}

\caption{\label{tbl-simulateddata}Simulated Multilevel Data}

\tabularnewline

\toprule\noalign{}
\begin{minipage}[b]{\linewidth}\centering
outcome
\end{minipage} \\
\midrule\noalign{}
\endhead
\bottomrule\noalign{}
\endlastfoot
57.47 \\
50.1 \\
52.92 \\
60.17 \\
55.05 \\
49.81 \\

\end{longtable}

\begin{figure}

\centering{

\includegraphics{simulated-multi-country-data_files/figure-pdf/fig-data-1.pdf}

}

\caption{\label{fig-data}Graph of Simulated Data}

\end{figure}%

\bookmarksetup{startatroot}

\chapter{Conceptual Framework}\label{sec-conceptualframework}

\begin{quote}
``Sure, it's hard to get started; remember learning to use knife and
fork? Dig in: you'll never reach bottom. It's not like it's the end of
the world--just the world as you think you know it.'' (Dove, 1999)
\end{quote}

\begin{quote}
``Ubuntu'' defined as: ``A person is a person through other people.''
e.g.~in (Mangharam, 2017)
\end{quote}

\section{Units of Analysis and Processes at Multiple
Levels}\label{units-of-analysis-and-processes-at-multiple-levels}

When confronted with multilevel data, one has a number of choices about
the units of analysis: one could consider individuals to be the units of
analysis; or, one could consider the larger social units to be the units
of analyses. With multilevel analytic methods, one is able to avoid this
false dichotomy, and to conceptualize the data from a multilevel
perspective, wherein both individuals and social units are different
levels of the same analysis Raudenbush \& Bryk (2002). I discuss some of
the statistical implications of different ideas about the units of
analysis in Section~\ref{sec-wrongapproaches}. \index{units of analysis}

Further, with multilevel models, we are not only able to consider the
idea of units of analysis at multiple levels of the data, but to
consider how variables at both Level 2 and Level 1 may affect an
individual level (Level 1) outcome.

\begin{figure}

\centering{

\includegraphics[width=0.5\textwidth,height=\textheight]{fig-conceptual.png}

}

\caption{\label{fig-conceptual}Conceptual Framework}

\end{figure}%

\section{Variables at Multiple Levels}\label{sec-levels}

In this book, I distinguish between \emph{conceptual} and
\emph{statistical} levels of variables.
\index{variables at multiple levels}

By \emph{conceptual} level, I refer to whether a variable is
\emph{conceptualized} to be measure of an \emph{individual} level
characteristic, such as parenting or mental health, or a
\emph{community} level construct, such as community collective efficacy,
or community safety.

By \emph{statistical} level, I refer to whether a variable measures an
\emph{individual} response, or an \emph{aggregated} response.

\begin{longtable}[]{@{}
  >{\raggedright\arraybackslash}p{(\columnwidth - 4\tabcolsep) * \real{0.2625}}
  >{\centering\arraybackslash}p{(\columnwidth - 4\tabcolsep) * \real{0.3625}}
  >{\centering\arraybackslash}p{(\columnwidth - 4\tabcolsep) * \real{0.3750}}@{}}

\caption{\label{tbl-variablelevel}Multiple Levels of Variables}

\tabularnewline

\toprule\noalign{}
\begin{minipage}[b]{\linewidth}\raggedright
\end{minipage} & \begin{minipage}[b]{\linewidth}\centering
statistical level 1
\end{minipage} & \begin{minipage}[b]{\linewidth}\centering
statistical level 2
\end{minipage} \\
\midrule\noalign{}
\endhead
\bottomrule\noalign{}
\endlastfoot
conceptual level 1 & Individual response about parenting or mental
health & Aggregated responses about parenting or mental health \\
conceptual level 2 & Individual response about community & Aggregated
response about community \\
conceptual level 2 & N/A & Administrative indicator of social unit \\

\end{longtable}

\begin{itemize}
\item
  Thus, \(\text{mental health}_{ij}\) or \(\text{parenting}_{ij}\) would
  be considered in the terminology that I am using to be a variable both
  \emph{conceptually} and \emph{statistically} at Level 1.
\item
  \(\overline{\text{mental health}_{.j}}\) or
  \(\overline{\text{parenting}_{.j}}\) would be variables that
  \emph{conceptually} come from Level 1 responses, but are
  \emph{statistically} aggregated to Level 2.
\end{itemize}

\begin{tcolorbox}[enhanced jigsaw, breakable, colbacktitle=quarto-callout-tip-color!10!white, leftrule=.75mm, bottomrule=.15mm, opacityback=0, colframe=quarto-callout-tip-color-frame, coltitle=black, rightrule=.15mm, title=\textcolor{quarto-callout-tip-color}{\faLightbulb}\hspace{0.5em}{Contextual Variables}, colback=white, opacitybacktitle=0.6, toprule=.15mm, titlerule=0mm, bottomtitle=1mm, left=2mm, toptitle=1mm, arc=.35mm]

Such aggregated variables represent the \emph{average} level of a
response across each Level 2 unit, and are sometimes called ``contextual
variables'' (Diez Roux, 2002; Hox et al., 2018; Snijders \& Bosker,
2012). \index{contextual variables} These aggregated variables could be
included in the model alongside the individual level, or Level 1,
predictors. Consider:

\begin{equation}\phantomsection\label{eq-contextual}{y_{ij} = \beta_0 + \beta_1 \text{parenting}_{ij} + \beta_2 \overline{\text{parenting}_{.j}} + u_{0j} + e_{ij}}\end{equation}

Equation~\ref{eq-contextual} is be a model that includes parenting as a
predictor of the outcome at two different levels.
\(\text{parenting}_{ij}\) is a parenting behavior--whether discipline or
warmth--at an \emph{individual} or \emph{family} level, while
\(\overline{\text{parenting}_{.j}}\) is a \emph{contextual variable}
representing the average level of parenting--whether discipline or
warmth--at the \emph{country} level. Equation~\ref{eq-contextual} is
thus testing whether parenting at the country level--sometimes called a
country-level effect--has an association with the outcome over and above
individual level parenting behavior.

Discussion of ways to create variables that are the average of a
predictor is contained in the Appendix.

\end{tcolorbox}

\begin{itemize}
\item
  Using my terminology, \(\text{community collective efficacy}_{ij}\) or
  \(\text{community safety}_{ij}\) would be considered to be a variable
  that was \emph{conceptually} at Level 2, but \emph{statistically} at
  Level 1.
\item
  \(\overline{\text{community collective efficacy}_{.j}}\) or
  \(\overline{\text{community safety}_{.j}}\) would be variables that
  \emph{conceptually} refer to Level 2 concepts that are
  \emph{statistically} aggregated to Level 2.
\end{itemize}

Some variables only exist at Level 2, and their Level 1 counterparts are
undefined. For example, the size of a school, neighborhood, or country,
is inherently a Level 2 variable, with no easily definable Level 1
counterpart. Similarly, some administrative indicators, such as the Gini
level of inequality, while developed by calculating across Level 1
responses, have no easily definable Level 1 counterpart.

\section{Multilevel Models As The Exploration Of Variation and
Diversity}\label{sec-studyvariation}

Multilevel models are sometimes seen as an analytic technique that
\emph{controls for} the clustering or nesting of individuals inside
larger social units such as schools, neighborhoods, or countries. I will
describe below how this ability to \emph{control for} clustering is
indeed an important and crucial aspect of multilevel models.
\index{variation and diversity}

However, my argument here is that multilevel models are better seen as a
method to \emph{explore} the variation and diversity inherent within
nested or clustered data. Again, while these issues are well understood
within the statistical literature (Hox et al., 2018; Kreft \& de Leeuw,
1998; Luke, 2004; Rabe-Hesketh \& Skrondal, 2012; Raudenbush \& Bryk,
2002; Singer \& Willett, 2003; Snijders \& Bosker, 2012), they are less
often noted in applied research.

\subsection{A First Example: A Study Of Parenting And Child
Development}\label{a-first-example-a-study-of-parenting-and-child-development}

In the graph below, imagine that physical punishment, or some other risk
factor, is associated with detrimental mental health outcomes. Each
country in the data has its own \emph{country specific regression line}.

\begin{figure}

\centering{

\includegraphics{conceptual-framework_files/figure-pdf/fig-variation1-1.pdf}

}

\caption{\label{fig-variation1}Plausible Alternative Patterns of Between
Country Variation}

\end{figure}%

In Panel A, there is some variation in the \emph{intercept}, which is
equivalent to saying that there is some variation in the average level
of psychological well-being across countries. When we look at the slope
of the country-specific regression lines in Panel A, we notice that
there is little variation in these \emph{slopes}. Put another way, there
is a great amount of consistency in the slopes of the country-specific
regression lines: parental use of physical punishment is consistently
associated with decreases in child psychological wellbeing across
countries.

In Panel B, the situation is different. There is more variation in the
\emph{intercept}, that is, more variation between countries in the
initial or average amount of psychological well-being. There is also
more variation in the \emph{slopes} of the country-specific regression
lines. While the average association between physical punishment and
psychological well-being is very similar to that in Panel A, there is
more variation across countries, in the relationship of physical
punishment and child psychological wellbeing, which would likely merit
exploration were one considering developing programs, policies or
interventions for different countries.

Lastly, the pattern of variation in Panel C is considerably different
from either Panel A or Panel B. The average association of physical
punishment with psychological well-being in the hypothetical scenario
represented by Panel C is approximately 0. There is some variation in
the \emph{intercepts} of the country-specific regression lines.
Additionally, there is considerable variation in the \emph{slopes} of
the country-specific regression line, suggesting that the use of
physical punishment might be beneficial in some countries, and
detrimental in others.

Empirically, data generally suggest a scenario somewhere between Panel A
and Panel B, but these different hypothetical scenarios afford us the
opportunity to think about possible patterns of variation.

\subsection{A Second Example: A Study Of A Treatment Or
Intervention}\label{a-second-example-a-study-of-a-treatment-or-intervention}

A second pedagogically helpful example might be obtained if we flip the
slopes in the diagram, and consider a different set of independent
variables, perhaps some kind of treatment or intervention designed to
improve psychological well-being.

\begin{figure}

\centering{

\includegraphics{conceptual-framework_files/figure-pdf/fig-variation2-1.pdf}

}

\caption{\label{fig-variation2}Considering an Intervention or Treatment
Across Countries}

\end{figure}%

We see a similar pattern as before, but the use of a different
substantive example may be illustrative.

In Panel A, there is relative consistency in the initial levels of
psychological well-being across countries, as well as consistency in the
degree to which the intervention is associated with improvements in
psychological well-being across countries.

In Panel B, we see more variation in both initial levels of
psychological well-being, but also more variation in the association of
the intervention with improvements in psychological well-being.

Lastly, in Panel C, we note an overall association of the intervention
with psychological well-being that is close to zero. However
associations vary widely by countries. In some countries there appears
to be evidence that the intervention is beneficial, while in other
countries there appears to be evidence that the intervention is not
beneficial, or even possibly harmful.

\subsection{Exploring Variation}\label{exploring-variation}

Thus, I emphasize an approach to multilevel modeling that sees
multilevel modeling as the \emph{study of variation} or an
\emph{exploration of variation}, not simply \emph{accounting for
variation}, or \emph{controlling for variation}.
\index{variation and diversity} \index{universalism}

\begin{quote}
``\ldots{} universal theorizing requires adequately sampled (i.e.,
diverse) data and better appreciation of issues of comparability and the
most powerful theories ought to predict and explain variation, not sweep
variation under the rug.'' (Blasi et al., 2022)
\end{quote}

As I discuss these ideas in more statistical depth, later in the book, I
develop more statistically based ideas about the study of diversity and
variation in Section~\ref{sec-ICC}, Section~\ref{sec-withinbetween}, and
Section~\ref{sec-studyvariation2}.

Again, statistically sophisticated treatments of all of the ideas are
available in one form or another across the excellent textbooks on
multilevel modeling (Hox et al., 2018; Kreft \& de Leeuw, 1998; Luke,
2004; Rabe-Hesketh \& Skrondal, 2012; Raudenbush \& Bryk, 2002; Singer
\& Willett, 2003; Snijders \& Bosker, 2012). However, some of these
ideas appear less often in applied research, and my intention here is to
make the application of these ideas to applied research, and to concerns
of variation and diversity, more clear.

\bookmarksetup{startatroot}

\chapter{The Cross Sectional Multilevel Model}\label{sec-crosssectional}

\begin{quote}
``Mathematical Science shows us what is. It is the language of unseen
relations between things. But to use \& apply that language we must be
able fully to appreciate, to feel, to seize, the unseen, the
unconscious. Imagination too shows us what is, the is that is beyond the
senses.'' (Lovelace, 1992) \index{Ada Lovelace}
\end{quote}

\begin{quote}
``I'm often asked if there is something I think writers ought to do, and
recently in an interview I heard myself say: `Several things. Love
words, agonize over sentences. And pay attention to the world.'\,''
(Sontag, 2007) \index{Susan Sontag}
\end{quote}

\section{Introduction}\label{introduction-1}

I begin this chapter with some introductory thinking about multilevel
modeling, by starting with two key ideas: multilevel models can improve
our estimation of p values; multilevel models can improve our estimation
of \(\beta\) coefficients. \index{cross sectional models}

After introducing these two key concepts of multilevel modeling, I then
begin a more in depth exploration of the equations and concepts and
statistical syntax of the cross sectional multilevel model.

\section{Some First Ideas About Multilevel
Modeling}\label{some-first-ideas-about-multilevel-modeling}

\subsection{Estimating Standard Errors And p Values}\label{sec-pvalues}

\subsubsection{Introducing the Idea}\label{introducing-the-idea}

If the data are grouped, nested, or clustered, then this aspect of the
structure of the data needs to be accounted for. Bland \& Altman (1994)
describe a simulation in which grouped data are artificially generated
according to the following procedure. \index{p values}
\index{advantages of the multilevel model}

\begin{quote}
``The data were generated from random numbers, and there is no relation
between X and Y at all. Firstly, values of X and Y were generated for
each `subject,' then a further random number was added to make the
individual observation.'' (Bland \& Altman, 1994)
\end{quote}

The graph below illustrates the process of simulating the data.

\begin{figure}

\centering{

\includegraphics{cross-sectional_files/figure-pdf/fig-simulatedclustereddata-1.pdf}

}

\caption{\label{fig-simulatedclustereddata}Simulated Clustered Data}

\end{figure}%

\subsubsection{Compare OLS and MLM}\label{compare-ols-and-mlm}

An analysis that is not aware of the grouped nature of the data will
give biased results, will mis-estimate standard errors, and importantly,
will often attribute statistical significance to some of the independent
variables when this is not appropriate (Bland \& Altman, 1994;
Raudenbush \& Bryk, 2002). \index{p values}

In the example below, we compare a simple ordinary least squares
analysis of the data with a multilevel model that accounts for the
clustered nature of the data.

\begin{longtable}[]{@{}lllll@{}}
\toprule\noalign{}
& OLS & & MLM & \\
\midrule\noalign{}
\endhead
\bottomrule\noalign{}
\endlastfoot
x & 1.046 & ** & 0.039 & \\
Intercept & 4.488 & & 97.005 & ** \\
var(\_cons) & & & 74.523 & \\
var(e) & & & 0.594 & \\
Number of observations & 25 & & & \\
\end{longtable}

** p\textless.01, * p\textless.05

We see that in the ordinary least squares analysis, the independent
variable is judged to have a statistically significant association with
the dependent variable. The more appropriate multilevel model finds that
in fact the independent variable \(x\) is \emph{not} associated with
\(y\). Thus, the multilevel model provides more accurate results than
OLS in the presence of clustered data.

\subsection{Multilevel Structure}\label{sec-multilevelstructure}

Associations between two variables can be \emph{very different} (or even
\emph{reversed}) depending upon whether or not the analysis is ``aware''
of the grouped, nested, or clustered nature of the data (Gelman et al.,
2007; Nieuwenhuis, 2015). In the example presented here, the groups are
countries, but could as easily be neighborhoods, communities, or
schools.

\begin{quote}
For teaching purposes, I use an example with very few clusters, although
it would be more appropriate to apply multilevel analysis to an example
with many more clusters e.g.~(\(N_\text{clusters} >= 30\))
\end{quote}

A model that is ``aware'' of the clustered nature of the data may
provide very different--likely better--substantive conclusions than a
model that is not aware of the clustered nature of the data.
\index{sign of coefficients} \index{advantages of the multilevel model}

I use some data simulated for this particular example.

\subsubsection{Graphs}\label{graphs}

\paragraph{A ``Naive'' Graph}\label{a-naive-graph}

This ``naive'' graph is unaware of the grouped nature of the data.
Notice that the overall regression line slopes downward, even though
there is some suggestion that \emph{within each group} the regression
lines may slope upward.

\begin{figure}

\centering{

\includegraphics{cross-sectional_files/figure-pdf/fig-naive-1.pdf}

}

\caption{\label{fig-naive}A `Naive' Graph}

\end{figure}%

\paragraph{An ``Aware'' Graph}\label{an-aware-graph}

This ``aware'' graph is aware of the grouped nature of the data. The
graph is ``aware'' of the grouped or clustered nature of the data, and
provides indication that the regression lines \emph{when accounting for
group} slope upward.

\begin{figure}

\centering{

\includegraphics{cross-sectional_files/figure-pdf/fig-aware-1.pdf}

}

\caption{\label{fig-aware}An `Aware' Graph}

\end{figure}%

\subsubsection{Regressions: A ``Naive'' OLS Analysis vs.~An ``Aware''
MLM
Analysis}\label{regressions-a-naive-ols-analysis-vs.-an-aware-mlm-analysis}

The OLS model with only \emph{x} as a covariate is not aware of the
grouped structure of the data, and the coefficient for \emph{x} in the
OLS model reflects this. The coefficient for \emph{x} in the OLS model
is \emph{negative}, and statistically significant.

The multilevel model is aware of the grouped structure of the data, and
the coefficient for \emph{x} in the multilevel model reflects this. The
coefficient for \emph{x} in the multilevel model is \emph{positive}, and
statistically significant.

\begin{longtable}[]{@{}lllll@{}}
\toprule\noalign{}
& OLS & & MLM & \\
\midrule\noalign{}
\endhead
\bottomrule\noalign{}
\endlastfoot
x & -0.775 & ** & 1.038 & ** \\
Intercept & 57.133 & ** & 29.029 & ** \\
var(\_cons) & & & 276.867 & \\
var(e) & & & 0.916 & \\
Number of observations & 30 & & & \\
\end{longtable}

** p\textless.01, * p\textless.05

\subsubsection{A Thought Experiment}\label{a-thought-experiment}

When might a situation like this arise in practice? This is surprisingly
difficult to think through.

Imagine that \emph{x} is a protective factor, or an intervention or
treatment. Imagine that \emph{y} is a desirable outcome, like improved
mental health or psychological well being.

Now imagine that residents of countries provide more of the protective
factor or more of the intervention in situations where there are lower
levels of the desirable outcome. If one thinks about it, this is a very
plausible situation.

\begin{quote}
A naive analysis that was unaware of the grouped nature of the data
would therefore misconstrue the results, suggesting that the
intervention was harmful, when it was in fact helpful.
\end{quote}

\begin{figure}

\centering{

\includegraphics{cross-sectional_files/figure-pdf/fig-heuristic-example-1.pdf}

}

\caption{\label{fig-heuristic-example}A Heuristic Example}

\end{figure}%

The idea that group level and individual level relationships must be the
same has been termed the ``ecological fallacy'' (Firebaugh, 2001; Hox et
al., 2018; Snijders \& Bosker, 2012). \index{ecological fallacy}

These data are constructed to provide this kind of extreme example, but
it easy to see how multilevel thinking, and multilevel analysis may
provide better answers than one would get if one ignored the grouped
nature of the data.

\section{The Equation}\label{the-equation}

The equation for the multilevel model can be written in several ways: as
multiple levels of equations; or as a single equation. The advantage of
having multiple levels of equations is that these multiple equations
make clear the multiple levels of the data, and thus conform to an
initial understanding of how a multilevel model should be estimated.
However, \emph{results} from multiple levels of equations quickly become
difficult to interpret, and thus, I will not spend a great deal of time
on discussing empirical results of the two level formulation. Whether
multiple levels of equations, or a single equation are employed, the
numerical results are equivalent.

\subsection{Two Levels of Equations}\label{two-levels-of-equations}

I start with two levels of equations: Level 1 at the level of the
individual; and Level 2 at the level of the country.

\subsubsection{Level 1 (Individuals)}\label{level-1-individuals}

\begin{equation}\phantomsection\label{eq-MLM1}{y_{ij} = \beta_{0j} + \beta_{1j} x_{ij} + \beta_{2j} z_{ij} + e_{ij}}\end{equation}

\subsubsection{Level 2 (Countries)}\label{level-2-countries}

\begin{equation}\phantomsection\label{eq-MLM2}{\beta_{0j} = \gamma_{00} +\gamma_{01} w_j + u_{0j}}\end{equation}

\[\beta_{1j} = \gamma_{10} + u_{1j}\]

\[\beta_{2j} = \gamma_{20}\]

\[\beta_{3j} = \gamma_{30}\]

Here \(y_{ij}\) is the dependent variable, or outcome for the model. We
note that the \(ij\) subscripts indicate that this is outcome \(y\) for
individual \(i\) in country \(j\). Note that the outcome is at Level 1,
or the level of individuals. \(\beta_{0j}\) is a regression intercept,
and the other \(\beta\)'s\footnote{Technically, all of these \(\beta\)'s
  could be written as \(\beta_j\) since the multilevel model could be
  said to estimate a regression parameter for each group, in this case
  each country. One could even write \(\beta_{jk}\) to represent the
  regression parameter for the \(k^{th}\) independent variable the for
  the \(j^{th}\) group or country. To keep matters simple, I simply
  write \(\beta\) in most cases.} are regression slope parameters.
\(x_{ij}\) and \(z_{ij}\) are independent variables. I note that in this
discussion I am \emph{not} considering a model in which there are
repeated observations on the same individuals, although the multilevel
model is certainly extensible to such cases
(Chapter~\ref{sec-longitudinal}). \(u_{0j}\) is a \emph{random
intercept} for the \(\beta_{0j}\) term, and \(u_{1j}\) is a \emph{random
slope} for the \(\beta_{1j}\) term, indicating that we are modeling
cross country variation in these parameters. \index{random intercept}
\index{random slope} The other \(\beta\) terms are not modeled as having
random country level variation, although this could certainly be a
possibility in subsequent models.

In this formulation of the multilevel model, each regression parameter
\(\beta\) in the level 1 equation is the outcome of an equation at Level
2. The parameters for the Level 2 equations are represented by
\(\gamma\)'s. \(w\) a Level 2 variable appears in the first Level 2
equation.

\subsection{One Level of Equations}\label{one-level-of-equations}

By simply substituting the values of the Level 2 equations into the
Level 1 equations--and rewriting the \(\gamma\)'s as \(\beta\)'s--we
obtain:

\begin{equation}\phantomsection\label{eq-MLM}{y_{ij} = \beta_0 + \beta_1 x_{ij} + \beta_2 z_{ij} + \beta_3 w_{j} + u_{0j} + u_{1j} \times x_{ij} + e_{ij}}\end{equation}

Here again \(y_{ij}\) is the dependent variable, or outcome for the
model. \(\beta_0\) is a regression intercept, and the \(\beta\)'s are
regression parameters. \(x_{ij}\) and \(z_{ij}\) are independent
variables and \(w\) is a Level 2 variable.

\begin{quote}
Notice that in this \emph{single equation} format all variables--no
matter their \emph{level}--appear in the same equation.
\end{quote}

In this formulation of the equation, the nature of the random effects is
more clear, and merits discussion. Notice that we have included a
\emph{random intercept}, \(u_{0j}\) as well as a \emph{random slope},
\(u_{1j} \times x\). The \emph{random intercept}, \(u_{0j}\), indicates
that there is variation in the \emph{intercept} of the country specific
regression lines, as is true in Figure~\ref{fig-data}. The \emph{random
slope} term associated with \(x_{ij}\), \(u_{1j} \times x_{ij}\),
indicates that we are allowing for the possibility of variation in the
\emph{slope} of the regression lines that is associated with \(x_{ij}\),
in this case, the slope of parental warmth, as is possibly suggested in
Figure~\ref{fig-data}.

To make these ideas more concrete, I rewrite this equation in terms of
the main substantive ideas of this book:

\begin{equation}\phantomsection\label{eq-MLMsubstantive}{\text{outcome}_{ij} = \beta_0 + \beta_1 \text{parental warmth}_{ij} + \beta_2 \text{physical punishment}_{ij} +}\end{equation}

\[\beta_3 \text{identity}_{ij} + \beta_4 \text{intervention}_{ij} + \beta_5 \text{HDI}_{ij} + \]
\[u_{0j} + u_{1j} \times \text{parental warmth}_{ij} + e_{ij}\]

Put substantively, this model indicates that the outcome can be
conceptualized as a function of an intercept term, and contributions of
parental warmth, physical punishment, identity group membership,
participation in the intervention, and country level HDI. The
\emph{random intercept}, \(u_{0j}\) indicates that there is some
unexplained variation in the outcome at the country level. The
\emph{random slope}, \(u_{1j} \times \text{parental warmth}_{ij}\)
indicates that the model is allowing for country level variation in the
association of parental warmth with the outcome. Inspection of
Figure~\ref{fig-data} indicates that it might be possible that there
would be variation across countries in this slope. The model could be
extended to allow for country level variation in other slope terms by
adding other random slopes, eg \(u_{2j}\), \(u_{3j}\), etc.

\section{Regression With Simulated Multi-Country
Data}\label{sec-regression}

After considering some of these broader issues, let's now examine the
results of a multilevel regression with the simulated multi-country
data. I will again imagine that the desirable outcome is an outcome such
as improved psychological wellbeing.

\subsection{Unconditional Model}\label{sec-unconditional}

The unconditional model is a model with no \(x\)'s or covariates
(Raudenbush \& Bryk, 2002). \index{unconditional model}

\begin{equation}\phantomsection\label{eq-unconditional}{\text{outcome}_{ij} = \beta_0 + u_{0j} + e_{ij}}\end{equation}

Here, \(\text{outcome}_{ij}\) is a function of an intercept \(\beta_0\),
a country specific error term, \(u_{0j}\), and an individual level error
term \(e_{ij}\).

Thus, all of the variation in \(\text{outcome}_{ij}\) is--given the
\emph{unconditional} nature of our model--attributable to unmeasured
variation at the country and individual level.

\subsection{Intra-Class Correlation Coefficient}\label{sec-ICC}

I now introduce a measure known as the Intra-Class Correlation
Coefficient, (ICC) that can be computed from this unconditional model
(Raudenbush \& Bryk, 2002). \index{variation and diversity} \index{ICC}

\begin{equation}\phantomsection\label{eq-ICC}{\text{ICC} = \frac{\text{var}(u_{0j})}{\text{var}(u_{0j}) + \text{var}(e_{ij})}}\end{equation}

Heuristically:

\begin{equation}\phantomsection\label{eq-ICCheuristic}{\text{ICC} = \frac{\text{group level variation}}{\text{group level variation} + \text{individual level variation}} = }\end{equation}

\[\frac{\text{group level variation}}{\text{total variation}}\]

The ICC from the \emph{unconditional} model
(Equation~\ref{eq-unconditional}) is the most informative ICC as it
represents the amount of variation in the dependent variable that could
\emph{potentially} be explained by the grouping variable.

Put another way, in a two level model, the ICC provides a quantitative
measure of the amount of variation in a measure that is present at Level
2. Knowing the ICC, we can then easily calculate the percentage of
variation at Level 1:

\[\text{Level 1 variation} = \text{total variation} - \text{Level 2 variation} =\]
\[= 100\% - (100\% \times \text{ICC})\]

Thus, in some broader sense, the ICC might be thought of as a measure of
diversity: the higher the ICC, the higher the clustering of the data in
groups, and more of the diversity in the sample is between countries;
the lower the ICC, the lower the clustering of the data, and less of the
diversity in the sample is between countries.

In practice, I have often found the ICC in cross-sectional models to be
relatively low: often on the order of 5\%, 10\% or 15\%. Phrased in
terms of diversity and variation, this means that in cross-sectional
applications, there is often much more diversity and variation within
each group indicated by the grouping variable (e.g.~neighborhood,
school, country) than there is between the values of the grouping
variable.

\begin{longtable}[]{@{}lll@{}}
\toprule\noalign{}
& 1 & \\
\midrule\noalign{}
\endhead
\bottomrule\noalign{}
\endlastfoot
\_cons & 52.433 & ** \\
var(\_cons) & 3.179 & \\
var(e) & 39.461 & \\
Number of observations & 3000 & \\
\end{longtable}

** p\textless.01, * p\textless.05

From using procedures to estimate the ICC, as detailed in the Appendix,
or calculating by hand, we see that the ICC for this data is .076 or
7.6\%.

As we add covariates, \(x\)'s, to the model the ICC will most often
decrease.

\subsection{Conditional Model}\label{sec-conditional-model}

We next estimate a \emph{conditional} model, \emph{with} independent
variables.

\begin{longtable}[]{@{}lll@{}}
\toprule\noalign{}
& 1 & \\
\midrule\noalign{}
\endhead
\bottomrule\noalign{}
\endlastfoot
warmth & 0.835 & ** \\
physical\_punishment & -0.992 & ** \\
identity & & \\
1 & -0.300 & \\
intervention & & \\
1 & 0.640 & ** \\
HDI & -0.003 & \\
\_cons & 52.000 & ** \\
var(warmth) & 0.023 & \\
var(\_cons) & 2.964 & \\
var(e) & 34.975 & \\
Number of observations & 3000 & \\
\end{longtable}

** p\textless.01, * p\textless.05

The data suggest that parental warmth is positively associated with the
desirable outcome, and that this result is statistically significant.
Parental use of physical punishment is associated with statistically
significant decreases in the desirable outcome. The identity variable is
not associated here with the outcome. In contrast, the application of
the intervention is associated with increases in the outcome.

I note that there is some variation in the \emph{constant} indicating
that there is some variation in the initial or average levels of the
desirable outcome--again improved psychological well-being--that is
attributable to country.

There is--in contrast--no discernible variation in the \emph{slope}
associated with parental warmth that is attributable to country. Thus,
the relationship of parental warmth with child outcomes does not appear
to differ appreciably from country to country. Had there been
statistically significant variation in this \emph{random slope}, it
would have indicated that the association of parental warmth with the
outcome varied across countries.

\texttt{HDI}, the \emph{Human Development Index}, our only country
level, or Level 2, variable in this model is not associated with the
outcome.

\begin{tcolorbox}[enhanced jigsaw, breakable, colbacktitle=quarto-callout-tip-color!10!white, leftrule=.75mm, bottomrule=.15mm, opacityback=0, colframe=quarto-callout-tip-color-frame, coltitle=black, rightrule=.15mm, title=\textcolor{quarto-callout-tip-color}{\faLightbulb}\hspace{0.5em}{Intuitions}, colback=white, opacitybacktitle=0.6, toprule=.15mm, titlerule=0mm, bottomtitle=1mm, left=2mm, toptitle=1mm, arc=.35mm]

\emph{Random intercepts} and \emph{random slopes} are statistically
closely related concepts. In my experience of over 10 years of teaching
multilevel modeling, I have found that \emph{random intercepts} are more
intuitive, while a fuller set of intuitions about what \emph{random
slopes} substantively imply seems to be much more difficult, and to take
much more time to fully understand.

Briefly, \emph{random intercepts} refer to group level
(e.g.~neighborhood level, school level, or country level) variation or
diversity in the initial level---or equivalently the overall level---of
some outcome like behavior or mental health. Put concretely: do, for
example, neighborhoods, or schools, or countries, systematically differ
in the \emph{overall level of some outcome} like behavior or mental
health?

In contrast, \emph{random slopes} refer to group level
(e.g.~neighborhood level, school level, or country level) variation or
diversity in \emph{the relationship of a predictor to an outcome}. Put
concretely, for example:

\begin{enumerate}
\def\labelenumi{\arabic{enumi}.}
\tightlist
\item
  Does the \emph{relationship of parenting with an outcome} differ
  systematically by neighborhoods, or schools, or countries?
\item
  Does the \emph{relationship of some identity with an outcome} differ
  systematically by neighborhoods, or schools, or countries?
\item
  Does the \emph{relationship of an intervention with an outcome} differ
  systematically by neighborhoods, or schools, or countries?
\end{enumerate}

\end{tcolorbox}

\section{Indicator Variables, Random Intercepts and Random Slopes, and
Identities}\label{sec-indicator-variables-random-effects}

\subsection{Variable Types}\label{variable-types}

In thinking about any statistical analysis, it is important to think
about different types of variables. Thinking about different types of
variables may seem to be an arcane or obscure topic. Statisticians can
make very fine distinctions between different types of variables, and it
sometimes seems that there are as many typologies of variables as there
are statisticians. Nonetheless some broad distinctions are useful.
\index{types of variables}

In a classic text, Freedman et al. (1991) offer a short definition of
variable types that provides some useful crucial distinctions:

\begin{quote}
``Some questions are answered by giving a number: the corresponding
variables are \emph{quantitative}. Age, family size, and family income
are examples of quantitative variables. Some questions are answered with
adjectives, and the corresponding variables are \emph{qualitative}:
examples are marital status (single, married, widowed, divorced,
separated) and employment status (employed, unemployed, not in the labor
force).''
\end{quote}

When teaching and learning about statistical software, I find it easiest
to talk about \emph{quantitative} variables as \emph{continuous} or
\emph{numeric} variables, and \emph{qualitative} variables as
\emph{categorical} variables.\footnote{I recognize that there are much
  finer distinctions to be made here, many of which Freedman et al.
  (1991)--as well as other introductory texts--go on to make. However,
  for the purposes of explication in this book, I find the
  \emph{continuous/numeric} versus \emph{categorical} variable
  distinction to be the most useful and the one that most readily
  facilitates communication with faculty colleagues and with students.}

\begin{tcolorbox}[enhanced jigsaw, breakable, colbacktitle=quarto-callout-tip-color!10!white, leftrule=.75mm, bottomrule=.15mm, opacityback=0, colframe=quarto-callout-tip-color-frame, coltitle=black, rightrule=.15mm, title=\textcolor{quarto-callout-tip-color}{\faLightbulb}\hspace{0.5em}{Categorical Variables as Indicators or Random Intercepts}, colback=white, opacitybacktitle=0.6, toprule=.15mm, titlerule=0mm, bottomtitle=1mm, left=2mm, toptitle=1mm, arc=.35mm]

Statistically, there is often a concern about whether a particular
qualitative or categorical variable is most appropriately modeled as an
\emph{indicator variable}, or as a \emph{level} (random intercept).
Substantively, this question intersects with how one should
statistically include and model various measures of identity. Many
identities are often measured as categorical variables.
\index{indicator variables} \index{random intercept}

\end{tcolorbox}

\subsection{The Example of U.S. Census
Data}\label{the-example-of-u.s.-census-data}

The United States Census (United States Census Bureau, 2022) asks about
race using several different categories: White; Black or African
American; American Indian or Alaska Native; Asian; Native Hawaiian and
Pacific Islander; Some Other Race. Followup questions are asked for some
identities, and a separate question is asked about Latino / Hispanic
identity. Thus one might imagine a questionnaire with two measures.
\index{Census}

\begin{longtable}[]{@{}ll@{}}
\caption{Hypothetical Questions Measuring
Race}\label{tbl-race-ethnicity-questions}\tabularnewline
\toprule\noalign{}
Race & Latino \\
\midrule\noalign{}
\endfirsthead
\toprule\noalign{}
Race & Latino \\
\midrule\noalign{}
\endhead
\bottomrule\noalign{}
\endlastfoot
White & Latino or Hispanic \\
Black or African American & Not Latino or Hispanic \\
American Indian or Alaska Native & \\
Asian & \\
Native Hawaiian and Pacific Islander & \\
Some Other Race & \\
\end{longtable}

A table of data might then look something like the following.

\begin{longtable}[]{@{}
  >{\centering\arraybackslash}p{(\columnwidth - 8\tabcolsep) * \real{0.0694}}
  >{\centering\arraybackslash}p{(\columnwidth - 8\tabcolsep) * \real{0.1389}}
  >{\centering\arraybackslash}p{(\columnwidth - 8\tabcolsep) * \real{0.0972}}
  >{\centering\arraybackslash}p{(\columnwidth - 8\tabcolsep) * \real{0.1250}}
  >{\centering\arraybackslash}p{(\columnwidth - 8\tabcolsep) * \real{0.1389}}@{}}

\caption{\label{tbl-race-ethnicity-data}Simulated Data on Race and
Ethnicity}

\tabularnewline

\toprule\noalign{}
\begin{minipage}[b]{\linewidth}\centering
id
\end{minipage} & \begin{minipage}[b]{\linewidth}\centering
outcome
\end{minipage} & \begin{minipage}[b]{\linewidth}\centering
race
\end{minipage} & \begin{minipage}[b]{\linewidth}\centering
latino
\end{minipage} & \begin{minipage}[b]{\linewidth}\centering
country
\end{minipage} \\
\midrule\noalign{}
\endhead
\bottomrule\noalign{}
\endlastfoot
1 & 110.9 & 2 & 1 & 21 \\
2 & 85.56 & 1 & 0 & 6 \\
3 & 104.4 & 5 & 0 & 3 \\
4 & 90.79 & 1 & 1 & 1 \\
5 & 108.7 & 1 & 0 & 4 \\
6 & 114.4 & 1 & 0 & 13 \\

\end{longtable}

More recently, the Census has allowed respondents to select multiple
racial identities. Such a table of data might look like this.

\begin{longtable}[]{@{}
  >{\centering\arraybackslash}p{(\columnwidth - 18\tabcolsep) * \real{0.0610}}
  >{\centering\arraybackslash}p{(\columnwidth - 18\tabcolsep) * \real{0.1220}}
  >{\centering\arraybackslash}p{(\columnwidth - 18\tabcolsep) * \real{0.0976}}
  >{\centering\arraybackslash}p{(\columnwidth - 18\tabcolsep) * \real{0.0976}}
  >{\centering\arraybackslash}p{(\columnwidth - 18\tabcolsep) * \real{0.0976}}
  >{\centering\arraybackslash}p{(\columnwidth - 18\tabcolsep) * \real{0.0976}}
  >{\centering\arraybackslash}p{(\columnwidth - 18\tabcolsep) * \real{0.0976}}
  >{\centering\arraybackslash}p{(\columnwidth - 18\tabcolsep) * \real{0.0976}}
  >{\centering\arraybackslash}p{(\columnwidth - 18\tabcolsep) * \real{0.1098}}
  >{\centering\arraybackslash}p{(\columnwidth - 18\tabcolsep) * \real{0.1220}}@{}}

\caption{\label{tbl-race-ethnicity-2-data}Simulated Data on Race and
Ethnicity With Multiple Identities}

\tabularnewline

\toprule\noalign{}
\begin{minipage}[b]{\linewidth}\centering
id
\end{minipage} & \begin{minipage}[b]{\linewidth}\centering
outcome
\end{minipage} & \begin{minipage}[b]{\linewidth}\centering
race1
\end{minipage} & \begin{minipage}[b]{\linewidth}\centering
race2
\end{minipage} & \begin{minipage}[b]{\linewidth}\centering
race3
\end{minipage} & \begin{minipage}[b]{\linewidth}\centering
race4
\end{minipage} & \begin{minipage}[b]{\linewidth}\centering
race5
\end{minipage} & \begin{minipage}[b]{\linewidth}\centering
race6
\end{minipage} & \begin{minipage}[b]{\linewidth}\centering
latino
\end{minipage} & \begin{minipage}[b]{\linewidth}\centering
country
\end{minipage} \\
\midrule\noalign{}
\endhead
\bottomrule\noalign{}
\endlastfoot
1 & 110.9 & 0 & 0 & 1 & 0 & 0 & 0 & 0 & 19 \\
2 & 85.56 & 0 & 0 & 1 & 0 & 0 & 0 & 0 & 4 \\
3 & 104.4 & 0 & 0 & 1 & 0 & 0 & 1 & 0 & 24 \\
4 & 90.79 & 0 & 1 & 1 & 0 & 0 & 0 & 1 & 30 \\
5 & 108.7 & 1 & 0 & 1 & 1 & 1 & 0 & 0 & 19 \\
6 & 114.4 & 1 & 0 & 0 & 1 & 0 & 1 & 0 & 25 \\

\end{longtable}

\subsection{Identities As Indicator
Variables}\label{identities-as-indicator-variables}

Identities can be modeled as indicator variables. Importantly, whether
only a single identity is selected (e.g.
Table~\ref{tbl-race-ethnicity-data}), or multiple identities are
selected (Table~\ref{tbl-race-ethnicity-2-data}), if identities are
modeled as indicator variables, they should be modeled as a \emph{set}
of indicator variables with an omitted reference category. If there are
\(k\) categories of a variable, there should be \(k-1\) indicator
variables., e.g.

\begin{equation}\phantomsection\label{eq-indicator-variables}{y_{ij} = \beta_0 + \beta_2 \text{race2} + \beta_3 \text{race3} + \beta_4 \text{race4} + \beta_5 \text{race5} + \beta_6 \text{race6} + }\end{equation}

\[\beta_7 \text{latinx} + u_{0j} + e_{ij}\]

In Equation~\ref{eq-indicator-variables} the first group of race,
\texttt{race1} is treated as the reference category, and is therefore
omitted from the model, and all comparisons are made to individuals with
identity \texttt{race1}.

\begin{tcolorbox}[enhanced jigsaw, breakable, colbacktitle=quarto-callout-tip-color!10!white, leftrule=.75mm, bottomrule=.15mm, opacityback=0, colframe=quarto-callout-tip-color-frame, coltitle=black, rightrule=.15mm, title=\textcolor{quarto-callout-tip-color}{\faLightbulb}\hspace{0.5em}{Identities As Indicator Variables in Equations And Syntax}, colback=white, opacitybacktitle=0.6, toprule=.15mm, titlerule=0mm, bottomtitle=1mm, left=2mm, toptitle=1mm, arc=.35mm]

Thus, if an identity, such as race, is used as an \emph{indicator
variable}, it would appear in the equations (e.g.
Equation~\ref{eq-MLMsubstantive}), and in the statistical syntax covered
in the Appendix, in the same place as the \texttt{identity} variable.

\end{tcolorbox}

\begin{tcolorbox}[enhanced jigsaw, breakable, colbacktitle=quarto-callout-tip-color!10!white, leftrule=.75mm, bottomrule=.15mm, opacityback=0, colframe=quarto-callout-tip-color-frame, coltitle=black, rightrule=.15mm, title=\textcolor{quarto-callout-tip-color}{\faLightbulb}\hspace{0.5em}{Reference Categories Need To Be Chosen Carefully}, colback=white, opacitybacktitle=0.6, toprule=.15mm, titlerule=0mm, bottomtitle=1mm, left=2mm, toptitle=1mm, arc=.35mm]

Reference categories need to be chosen carefully. This is especially
true in studies that hope to be attentive to diversity and social
change. The default in many software programs is to exclude the first
category of an identity, as is evident in
Equation~\ref{eq-indicator-variables}. However, in the example given in
Table~\ref{tbl-race-ethnicity-questions}, this would mean that all
comparisons would be made to individuals who identified as \emph{white}.
It often makes sense to choose a different reference category than the
default, and investigators are encouraged to be thoughtful about the
choice of reference category for indicator variables. In most software,
the choice of reference categories can be manually specified.

\end{tcolorbox}

\begin{tcolorbox}[enhanced jigsaw, breakable, colbacktitle=quarto-callout-warning-color!10!white, leftrule=.75mm, bottomrule=.15mm, opacityback=0, colframe=quarto-callout-warning-color-frame, coltitle=black, rightrule=.15mm, title=\textcolor{quarto-callout-warning-color}{\faExclamationTriangle}\hspace{0.5em}{Indicator Variables Should Not Be Treated As Continuous Variables}, colback=white, opacitybacktitle=0.6, toprule=.15mm, titlerule=0mm, bottomtitle=1mm, left=2mm, toptitle=1mm, arc=.35mm]

Identities that are measured in the data with a single categorical
variable with values like \texttt{1}, \texttt{2}, \texttt{3},
\texttt{4}, \texttt{5}, \texttt{6} should \emph{not} be modeled as a
single continuous variable. For example, if the data are similar to
those presented in Table~\ref{tbl-race-ethnicity-data}--where identity
is measured as a single column of data--it is important that the
multiple identities \emph{not} be modeled as a single variable, which
would end up \emph{de facto} treating this multi-category
\emph{categorical} variable as a single \emph{continuous} variable.
Instead, multiple indicator variables (\(k-1\)) for the multiple (\(k\))
identities should be clearly specified in one's equation (as in
Equation~\ref{eq-indicator-variables}) and in the statistical syntax.
Details of how to accomplish this for different software packages are
included in the Appendix.

\end{tcolorbox}

\subsection{Identities as Random Intercepts or
Slopes}\label{identities-as-random-intercepts-or-slopes}

If there are many values of an identity it may be difficult to model
those multiple identities as indicator variables. For example, if there
were 10 possible values of a particular identity, and one was modeling
those identities with indicator variables, one would need to include 9
(\(k-1\)) indicator variables. Those 9 indicator variables could
potentially use a large fraction of the available degrees of freedom,
and interpreting the \(\beta\) regression coefficients for 9 indicator
variables might also be challenging. The problem is even greater when
one considers a larger number of identities (e.g.~25, 30, 50, 100), as
is possible in some data sets. In a global or cross cultural data set,
such as the simulated data that is used as an example throughout this
book, one might easily conceive of data with 20, 30 or 50 different
identities, especially since racial and ethnic identity categories may
differ substantially across countries, around the world (Rocha \&
Aspinall, 2020).

An alternative would be to model large numbers of possible identities as
a random intercept, rather than as a set of indicator variables. Such an
equation might appear as follows:

\begin{equation}\phantomsection\label{eq-identity-random-intercept}{y_{ij} = \beta_0 + \Sigma \beta_m \text{covariates} + v_{0k} + u_{0j} + e_{ij}}\end{equation}

Here, the inclusion of \(v_{0k}\), a random intercept for identity,
would be a way to model the presence of many possible values of an
identity variable.

\begin{tcolorbox}[enhanced jigsaw, breakable, colbacktitle=quarto-callout-tip-color!10!white, leftrule=.75mm, bottomrule=.15mm, opacityback=0, colframe=quarto-callout-tip-color-frame, coltitle=black, rightrule=.15mm, title=\textcolor{quarto-callout-tip-color}{\faLightbulb}\hspace{0.5em}{Identities As Random Intercepts in Equations And Syntax}, colback=white, opacitybacktitle=0.6, toprule=.15mm, titlerule=0mm, bottomtitle=1mm, left=2mm, toptitle=1mm, arc=.35mm]

Thus, if an identity, such as race, is used as a \emph{random
intercept}, it would appear in the equations (e.g.
Equation~\ref{eq-MLMsubstantive}), and in the statistical syntax covered
in the Appendix, in the same place as the \texttt{country} variable.

\end{tcolorbox}

An \emph{advantage} of the approach outlined in
Equation~\ref{eq-identity-random-intercept} is that we could model
multiple identities without including a large--potentially
enormous--number of indicator variables. We would thus avoid using up a
large number of degrees of freedom in our data set. We would also avoid
the conceptual difficulties that might come from including a large
number of indicator variables in our data set. For example, if we had 30
categories of a particular categorical variable, it might be difficult
to interpret the resultant 29 regression coefficients for the 29
indicator variables. At the same time, it would be straightforward to
include a random intercept that had 30, or even 50, or 100, or 1,000
possible values.

A potential \emph{disadvantage} of the approach outlined in
Equation~\ref{eq-identity-random-intercept} is that this is
multi-country data that already employs a random slope, \(u_{0j}\).
Adding \(v_{0k}\) means that we now have a model with \emph{two} random
intercepts, a topic which is discussed more extensively in
Chapter~\ref{sec-morelevels}.

A \emph{caveat} of the approach outlined in
Equation~\ref{eq-identity-random-intercept} is that random intercepts
(e.g.~\(v_{0k}\) and \(u_{0j}\))--which are essentially error terms at
level 2--are assumed to follow a normal distribution (StataCorp, 2023b).
Ten identities, to follow the example above, is generally considered to
be too few groups to satisfy this assumption of a normal distribution.
C. J. M. Maas \& Hox (2004) suggest that at least 50 level 2 units are
necessary to ensure satisfaction of this normality assumption.
Importantly, if one has fewer than 50 units, the bias introduced only
appears to affect the estimation of the level 2 random intercepts and
slopes, not the estimation of the independent variables (C. J. M. Maas
\& Hox, 2004; C. Maas \& Hox, 2005). If one has fewer than this number
one possible procedure is to model the groups both as a \emph{random
intercept} and as \emph{a set of indicator variables}, and to see if the
different approaches lead to different substantive conclusions (Gershoff
et al., 2010).

A second \emph{caveat} of modeling identities as a random intercept is
that these identities would need to be mutually exclusive. For example,
``black'', ``white'' and ``identifies as both black and white'', would
all need to be separately coded identities under this approach.

\section{Correlation of Random Intercept and Random
Slope(s)}\label{correlation-of-random-intercept-and-random-slopes}

To further elaborate the cross-sectional multilevel model that we have
been considering, we could also consider a situation in which a random
slope or slopes were \emph{correlated} with each other, and with the
random intercept. In the equation that we are considering, this would
entail estimation of whether or not, the random intercept, \(u_{0j}\),
was correlated with the random slope for warmth, \(u_{1j}\).
\index{correlation of random effects}

Substantively, this question would be asking whether the association of
warmth and the outcome, was correlated with the initial level or average
level of the outcome. From Figure~\ref{fig-data}, it appears that there
is some slight evidence that the country specific regression slopes are
more steep in countries where the initial level of the outcome is
higher. However, we may wish to investigate this question more
rigorously.

Procedures for estimating models with correlated or uncorrelated random
effects vary across software. I illustrate this issue in
Equation~\ref{eq-varcovar} below, where the diagonal elements are the
\emph{variances} of each of the random effects, and the off diagonals,
which would be the \emph{covariances} of the random effects are
constrained to 0.

\begin{equation}\phantomsection\label{eq-varcovar}{\begin{bmatrix}
var(u_{0j}) & 0 \\
0 & var(u_{1j}) 
\end{bmatrix}}\end{equation}

In contrast, we might wish to estimate a model in which the random
effects are allowed to be correlated.

\begin{equation}\phantomsection\label{eq-varcovaruns}{\begin{bmatrix}
var(u_{0j}) & cov(u_{0j}, u_{1j}) \\
cov(u_{0j}, u_{1j}) & var(u_{1j}) 
\end{bmatrix}}\end{equation}

When we estimate such a model, we get the following information.

\begin{longtable}[]{@{}lll@{}}
\toprule\noalign{}
& 1 & \\
\midrule\noalign{}
\endhead
\bottomrule\noalign{}
\endlastfoot
warmth & 0.833 & ** \\
physical\_punishment & -0.994 & ** \\
identity & & \\
1 & -0.298 & \\
intervention & & \\
1 & 0.644 & ** \\
HDI & -0.008 & \\
\_cons & 52.292 & ** \\
var(warmth) & 0.010 & \\
var(\_cons) & 2.257 & \\
cov(warmth,\_cons) & 0.147 & \\
var(e) & 35.006 & \\
Number of observations & 3000 & \\
\end{longtable}

** p\textless.01, * p\textless.05

Results are mostly similar to those above. However, here, we are asking
additionally for information about the possible \emph{correlation} of
country specific initial levels of the outcome and the slope of the
country specific regression line for parental warmth. Results indicate
that there is no reason to be believe that these two parameters are
correlated. Put more intuitively, it does not appear that parental
warmth is any more or less correlated with the outcome in countries
where initial levels of the outcome are higher. Again, had this
correlation been statistically significant and positive, it would have
indicated that higher initial, or average levels of the outcome were
associated with a greater association of warmth with the outcome.

\section{Within and Between}\label{sec-withinbetween}

Coefficients in models can be divided into within and between. A
substantive example may be helpful here. When we consider the variable
of parental \texttt{warmth}, we can imagine the parental warmth
expressed in each family, \(\text{warmth}_{ij}\), representing family
\emph{i} in country \emph{j}. We can also think about the \emph{grand
mean} of warmth across the entire sample,
\(\overline{\text{warmth}}_{..}\). We can then also think about the mean
expression of parental warmth in each country,
\(\overline{\text{warmth}}_{.j}\), i.e.~the mean level of parental
warmth in country \emph{j}. \index{within and between}

\begin{figure}

\centering{

\includegraphics{cross-sectional_files/figure-pdf/fig-distributionwarmth-1.pdf}

}

\caption{\label{fig-distributionwarmth}Distribution of Parental Warmth
Across Countries}

\end{figure}%

Bearing this in mind, one can then think about the \emph{difference}
between each individual expression of parental warmth and the overall,
or grand mean: \(\text{warmth}_{ij} - \overline{\text{warmth}}_{..}\).
This value can then be decomposed into two values:

\[\text{warmth}_{ij} - \overline{\text{warmth}}_{..} = \text{warmth}_{ij} - \overline{\text{warmth}}_{.j} + \overline{\text{warmth}}_{.j} - \overline{\text{warmth}}_{..}\]
Put into words, this equation says that the difference in parental
warmth displayed by family i in country j from the overall or grand mean
of parental warmth is composed of two components:

\begin{itemize}
\tightlist
\item
  \emph{Within Country Component}: How is the level of warmth expressed
  by family \emph{i} in country \emph{j} different from the \emph{mean}
  level of warmth in country \emph{j}. Is family \emph{i} different from
  the \emph{average} family in country \emph{j}? For this particular
  country, is this a family that is higher, or lower, than average in
  parental warmth?
\item
  \emph{Between Country Component}: How is the \emph{mean} level of
  warmth in country \emph{j} different from the overall or \emph{grand
  mean} level of warmth in the sample as a whole? To what degree is
  country \emph{j} different from \emph{all countries} in the sample? Is
  this country a country where parents tend to be higher, or lower, in
  parental warmth?
\end{itemize}

Theoretically, or conceptually, one might imagine that it would be
useful to decompose a particular behavior into within country and
between country components. The within country component could be
theorized as \emph{how an individual family differs from their context},
and the between country component could be theorized as \emph{how a
particular context differs from the average context}.
\index{variation and diversity}

\begin{figure}

\centering{

\includegraphics{cross-sectional_files/figure-pdf/fig-withinbetween-1.pdf}

}

\caption{\label{fig-withinbetween}Decomposing a Variable into Within and
Between Differences}

\end{figure}%

In terms of using statistical software, we need to follow a few steps.

\begin{enumerate}
\def\labelenumi{\arabic{enumi}.}
\item
  Calculate the \emph{grand mean} of the variable.
\item
  Calculate \emph{country specific means} of the variable.
\item
  Calculate:

  \begin{itemize}
  \tightlist
  \item
    individual scores - country specific means
  \item
    country specific means - grand mean
  \end{itemize}
\item
  Estimate the model with within and between.
\end{enumerate}

\begin{longtable}[]{@{}lll@{}}
\toprule\noalign{}
& 1 & \\
\midrule\noalign{}
\endhead
\bottomrule\noalign{}
\endlastfoot
dev\_warmth & 0.834 & ** \\
cdev\_warmth & 1.196 & \\
physical\_punishment & -0.992 & ** \\
identity & & \\
1 & -0.300 & \\
intervention & & \\
1 & 0.640 & ** \\
HDI & -0.004 & \\
\_cons & 54.981 & ** \\
var(warmth) & 0.023 & \\
var(\_cons) & 2.964 & \\
var(e) & 34.975 & \\
Number of observations & 3000 & \\
\end{longtable}

** p\textless.01, * p\textless.05

Estimates suggest that both the difference in an individual family's
expression of parental warmth from the country level mean, \emph{but
not} the difference in the country level mean from the grand mean are
statistically significant predictors of the outcome.

\section{Summary of Advantages Of The Multilevel
Model}\label{summary-of-advantages-of-the-multilevel-model}

The discussion so far gives an idea of the advantages of the multilevel
model for studying intrinsically multilevel data: children in classrooms
or schools; individuals or families in neighborhoods; individuals or
families in countries. These advantages can be summarized below:
\index{advantages of the multilevel model}

\begin{enumerate}
\def\labelenumi{\arabic{enumi}.}
\tightlist
\item
  Standard errors are estimated correctly as is statistical
  significance. This means that p values are correctly estimated
  accounting for the clustered or nested nature of the data. More
  colloquially, this most often means that we do not make the mistake of
  attributing statistical significance to a given risk or protective
  factor, when such a statistical significance is not warranted. Put
  even more straightforwardly correct estimation of standard errors and
  statistical significance prevents us from seeing results that are
  simply not present in the data, whether those concern risk factors or
  protective factors. \index{p values}
\item
  Regression coefficients are estimated correctly accounting for the
  clustered or nested structure of the data. If one does not account for
  the clustered or nested structure of the data, regression slopes can
  be estimated as negative when they are more correctly estimated as
  positive, or as null, or conversely estimated as positive when there
  are more correctly seen as negative (or null). Again, to phrase things
  in a more colloquial fashion, this means that we do not judge
  something to be a risk factor when it is in fact a protective factor
  or a null effect; or a protective factor when it is in fact a risk
  factor, or a null effect. \index{sign of coefficients}
\end{enumerate}

\section{Some Wrong (or Partially Wrong)
Approaches}\label{sec-wrongapproaches}

When data are clustered--e.g.~residents in neighborhoods, children in
schools, families in countries--it is worth discussing the fact that we
have several choices statistically as how to proceed, other than using a
multilevel model. Given the discussion so far, we can see the advantages
of a multilevel model over these other approaches:

\begin{enumerate}
\def\labelenumi{\arabic{enumi}.}
\tightlist
\item
  First, we could simply ignore the clustering, and treat the data as
  though it were composed of statistically independent individuals,
  i.e.~statistically independent \(e_i\). As we have discussed above,
  however, this approach has at least two disadvantages. First, as
  discussed in Section~\ref{sec-pvalues}, this approach will
  mis-estimate standard errors, most often underestimating them,
  resulting in underestimated p values and false positives. Second, as
  discussed in Section~\ref{sec-multilevelstructure} ignoring clustering
  runs the risk of estimating regression \(\beta\)'s that are not
  estimated with information about the multilevel structure of the data,
  with the possibility that \(\beta\) coefficients may not only have
  incorrect statistical significance, but also incorrect magnitude, and
  even incorrect sign. \index{p values} \index{sign of coefficients}
\item
  A second approach would be to \emph{aggregate} the data to the level
  of the higher social unit, e.g.~aggregating the data at the level of
  the neighborhood. Here we run into an idea similar to that discussed
  in Section~\ref{sec-multilevelstructure}, the ``ecological fallacy'':
  the idea that group level and individual level relationships are
  necessarily the same (Firebaugh, 2001). \index{ecological fallacy}
\item
  Lastly, we could adopt a statistical strategy of \emph{clustering} the
  standard errors. Clustering the standard errors means that standard
  errors are corrected for the non-independence of the \(e_i\) within
  clusters. Thus, \emph{p} values are estimated correctly. However,
  clustering still does not account for the multilevel structure of the
  data (Section~\ref{sec-multilevelstructure}), and thus when
  relationships between \emph{x}'s and \emph{y} at different levels of
  the data are very different, simply clustering the standard errors may
  not give correct estimates of the \(\beta\)'s. \index{p values}
\end{enumerate}

\section{Variation}\label{sec-studyvariation2}

Above, in Section~\ref{sec-studyvariation}, I have referred to
multilevel models as the study and exploration of variation. Now that I
have provided some discussion of the multilevel model, more statistical
``unpacking'' of ideas about variation is warranted.
\index{variation and diversity}

I provide again, for pedagogical purposes, the example substantive
equation (Equation~\ref{eq-MLMsubstantive}) that I have been using in
this book.

\begin{equation}\phantomsection\label{eq-MLMsubstantive2}{\text{outcome}_{ij} = \beta_0 + \beta_1 \text{parental warmth}_{ij} + \beta_2 \text{physical punishment}_{ij} +}\end{equation}

\[\beta_3 \text{identity}_{ij} + \beta_4 \text{intervention}_{ij} + \beta_5 \text{HDI}_{ij} + \]

\[u_{0j} + u_{1j} \times \text{parental warmth} + e_{ij}\]

\subsection{Measured and Unmeasured
Variation}\label{sec-measured-and-unmeasured-variation}

An equation for a multilevel model can be divided into measured and
unmeasured variation. Below I use a simplified form of
Equation~\ref{eq-MLMsubstantive2}, focusing in particular--for sake of
illustration--on the identity variable.

\begin{figure}[H]

{\centering \includegraphics[width=0.75\textwidth,height=\textheight]{measured-and-unmeasured-variation.png}

}

\caption{Measured and Unmeasured Variation}

\end{figure}%

\begin{equation}\phantomsection\label{eq-measured-and-unmeasured-variation}{\!}\end{equation}

I have already introduced the idea of an unconditional model
(Section~\ref{sec-unconditional}), in which there are no independent
variables, and all of the variation is unmeasured. The unconditional
intraclcass correlation coefficient (ICC) (Section~\ref{sec-ICC}) is a
measure of the amount of variation that could potentially be
attributable to the Level 2 units, in this case, different countries.

\begin{figure}

\centering{

\includegraphics{cross-sectional_files/figure-pdf/fig-variationsources-1.pdf}

}

\caption{\label{fig-variationsources}Sources of Variation in a
Multilevel Model}

\end{figure}%

\subsection{Variation In Intercepts or
Outcomes}\label{variation-in-intercepts-or-outcomes}

In Equation~\ref{eq-MLMsubstantive2}, \(var(u_{0j})\) is the model
estimated amount of variation in the \emph{outcome}, \(y_{ij}\).
\index{variation in intercepts or outcomes}

In the regression in Section~\ref{sec-regression}, there is discernible
between country variation, but more of the variation is between
individuals within the same country. Put another way, there is a
moderate tendency for children in families in the same country to have
similar outcomes, but two children in families in the same country may
also have very different outcomes. Children from families in different
countries may be as similar as children from families in the same
country.

\subsection{Variation In Predictors}\label{variation-in-predictors}

Equally important, I think, but much less frequently explored than
variation in \emph{outcomes}, is the possibility of variation in
\emph{predictors}, \(var(x_{ij})\). In the substantive example that we
have employed so far, the \emph{predictors} are different
\emph{parenting behaviors}, so considering variation in
\emph{predictors} allows us to consider variation in \emph{parenting
behaviors}, as well as variation in the \emph{outcomes} of those
behaviors. \index{variation in predictors}

We would estimate variation in behaviors attributable to country in much
the same way that we would estimate variation in outcomes, estimating an
unconditional model, but substituting \(x\) for \(y\).\footnote{Here for
  the sake of clarity, I use \(w_{0j}\) as a random effect to think
  about country specific variation in \(x\).}

\begin{equation}\phantomsection\label{eq-unconditionalx}{x_{ij} = \beta_0 + w_{0j} + e_{ij}}\end{equation}

Then, similarly, the variation in a predictor attributable to the
clustered nature of the data--in this case the clustering of individuals
in countries--is given by:

\begin{equation}\phantomsection\label{eq-ICCx}{\text{ICC}_x = \frac{var(w_{0j})}{var(w_{0j}) + var(e_{ij})}}\end{equation}

\subsection{Variation in Slopes}\label{variation-in-slopes}

Another possible type of variation to investigate is variation in the
relationship of \(x\) and \(y\), which is represented in the multilevel
model by examining variation in the \(\beta\)'s, i.e.~\(var(u_{1j})\).
\index{variation in slopes}

\subsection{Summary}\label{summary}

Thus, we can consider a number of sources of possible variation.

\begin{longtable}[]{@{}
  >{\raggedright\arraybackslash}p{(\columnwidth - 2\tabcolsep) * \real{0.5750}}
  >{\raggedright\arraybackslash}p{(\columnwidth - 2\tabcolsep) * \real{0.4250}}@{}}
\caption{Some Possible Sources of Variation To Consider in A Multilevel
Model}\label{tbl-sourcesvariation}\tabularnewline
\toprule\noalign{}
\begin{minipage}[b]{\linewidth}\raggedright
Model Parameter
\end{minipage} & \begin{minipage}[b]{\linewidth}\raggedright
Meaning
\end{minipage} \\
\midrule\noalign{}
\endfirsthead
\toprule\noalign{}
\begin{minipage}[b]{\linewidth}\raggedright
Model Parameter
\end{minipage} & \begin{minipage}[b]{\linewidth}\raggedright
Meaning
\end{minipage} \\
\midrule\noalign{}
\endhead
\bottomrule\noalign{}
\endlastfoot
Independent Variables & \\
\(var(x_{ij})\) & What is the variation in x? \\
\(range(x_{ij})\) & What are the maximum and minimum of x? \\
\(var(w_{0j}) \text{ if } x = \beta_0 + w_{0j} + e_{ij}\) & What is the
country specific variation in the value of x? \\
Dependent Variable & \\
\(var(y_{ij})\) & What is the variation in y? \\
\(range(y_{ij})\) & What are the maximum and minimum of y? \\
\(var(u_{0j})\) & What is the country specific variation in the
intercept of y? \\
Regression Coefficients for Slopes & \\
\(\beta_{x} x\) & What is the relationship of x and y? \\
\(\beta_{xz} z \times x\) & What is the effect of z on the relationship
of x and y? \\
\(var(u_{1j}) \text{ from } u_{1j} \times x\) & What is the country
specific variation in the relationship of x and y? \\
\(cov(u_{0j}, u_{1j})\) & What is the covariance of the country specific
intercept and country specific slope. Is the country specific intercept
related to the country specific slope? \\
\end{longtable}

\subsection{Variation As An Outcome}\label{variation-as-an-outcome}

Even less common is to examine \emph{variation} itself as an outcome
(Burkner, 2018). \index{variation as an outcome}

\begin{equation}\phantomsection\label{eq-distributional}{\sigma_{yij} = \beta_0 + \beta_1 x_1 + u_{0j} + e_{ij}}\end{equation}

Here, the variation in the outcome, \(\sigma_{yij}\), rather than the
mean level of the outcome, \(y_{ij}\), is the focus of interest. My
notation for Equation~\ref{eq-distributional} draws upon Burkner
(2018)'s notation, but is modified in order to be consistent with the
rest of this book.

Why might such models be of conceptual interest? Imagine for example,
that the \emph{variation} in psychological well-being is higher in
countries with higher levels of poverty, or higher levels of income
inequality. The use of such models as this, discussed in more detail by
Burkner (2018), would allow us to explore such a question.

Of note, while I do not explore in detail differences between Bayesian
and frequentist approaches to multilevel modeling in this book, these
models are likely to be only estimable with Bayesian software rather
than with frequentist software (Burkner, 2018).

\subsection{Maximal Models}\label{maximal-models}

Hypothetically, one might imagine that there could be group level
unobserved factors which affect regression slopes: i.e.~the relationship
between a predictor x and outcome variable y. Arguably, were one to
ignore these unobserved factors in statistical estimation, they would
show up either in an error term, or in the regression coefficients
themselves. Were they to show up in the regression coefficients this
would represent statistical bias and a substantive mis-estimation of
important effects. thus, there is a conceptual argument for including as
many random effects---i.e.~random slopes---in a statistical model as
possible. \index{maximal models}

Models with all possible random effects are termed \emph{maximal models}
(Barr et al., 2013; Frank, 2018). Such models include a large number of
random slopes,
e.g.~\(u_1 \times x_1, u_2 \times x_2, u_3 \times x_3, ..., \text{etc.}\)
even when some of those estimated slopes are close to 0. Such models may
be more easily estimable when using Bayesian estimation (Frank, 2018), a
topic which I do not cover in detail in this book.

It should be noted that Matuschek et al. (2017) argue that such a
\emph{maximal} approach may lead to a loss of statistical power and
further argue that one should adhere to ``a random effect structure that
is supported by the data.'' In contrast, Nalborczyk et al. (2019) argue
that maximal models are supported under the Bayesian approach. Oberauer
(2022) also argues for including multiple random slopes. Schielzeth \&
Forstmeier (2009) make a similar argument from a frequentist
perspective.

\bookmarksetup{startatroot}

\chapter{The Longitudinal Multilevel Model}\label{sec-longitudinal}

\begin{quote}
``Mathematics is the art of giving the same name to different things.''
(Poincare, 1908) \index{Henri Poincaré}
\end{quote}

Counter-intuitively, and surprisingly, the mathematics of estimating
models with cross-sectional clustered data easily generalizes to
longitudinal data. In cross sectional clustered data, we imagine
\emph{individuals or families clustered in neighborhoods, schools, or
countries}. \index{longitudinal models}

\begin{longtable}[]{@{}ll@{}}
\caption{Levels in Cross-Sectional
Data}\label{tbl-levelscrosssectional}\tabularnewline
\toprule\noalign{}
Level & Example(s) \\
\midrule\noalign{}
\endfirsthead
\toprule\noalign{}
Level & Example(s) \\
\midrule\noalign{}
\endhead
\bottomrule\noalign{}
\endlastfoot
1 & Individuals or Families \\
2 & Schools \\
& Neighborhoods \\
& Countries \\
\end{longtable}

In longitudinal data, we consider the \emph{first level} to be that of
\emph{time points}, or \emph{study waves}, which we sometimes call the
\emph{person-observation}.\footnote{When we are studying families,
  e.g.~a parent-child pair, it might be more appropriate to call each
  row of data a \emph{family-observation}, but the term
  \emph{person-observation} is more commonly used.}
\index{person-observation} The \emph{second level} is then the
individual or family.

\begin{longtable}[]{@{}ll@{}}
\caption{Levels in Longitudinal
Data}\label{tbl-levelslongitudinal}\tabularnewline
\toprule\noalign{}
Level & Example(s) \\
\midrule\noalign{}
\endfirsthead
\toprule\noalign{}
Level & Example(s) \\
\midrule\noalign{}
\endhead
\bottomrule\noalign{}
\endlastfoot
1 & Timepoints \\
2 & Individuals or Families \\
\end{longtable}

While it is less common, we could then easily add additional clustering
to this longitudinal model, for example, clustering of individuals or
families inside social units.

\begin{longtable}[]{@{}ll@{}}
\caption{Multiple Levels in Longitudinal
Data}\label{tbl-levelslongitudinal2}\tabularnewline
\toprule\noalign{}
Level & Example(s) \\
\midrule\noalign{}
\endfirsthead
\toprule\noalign{}
Level & Example(s) \\
\midrule\noalign{}
\endhead
\bottomrule\noalign{}
\endlastfoot
1 & Timepoints \\
2 & Individuals or Families \\
3 & Schools \\
& Neighborhoods \\
& Countries \\
\end{longtable}

\section{Use Data With Multiple Observations Per
Individual}\label{use-data-with-multiple-observations-per-individual}

Multilevel data suitable for longitudinal analysis has \emph{multiple
rows of data per individual or family}. Put another way, \emph{every row
of data is a person-timepoint}.

\begin{quote}
This method of organizing data is known as the \emph{long} format.
Another way of organizing longitudinal data--which I do not discuss in
detail here--is the \emph{wide} format in which every individual or
family has only a single row of data. In \emph{wide} data, the different
timepoints are in \emph{different columns} of data. I do discuss
\emph{reshaping} data from \emph{wide} to \emph{long}, and vice versa,
in the Appendix. \index{wide and long data}
\end{quote}

\begin{longtable}[]{@{}lll@{}}
\caption{Data in Long Format}\label{tbl-datalong}\tabularnewline
\toprule\noalign{}
id & t & x \\
\midrule\noalign{}
\endfirsthead
\toprule\noalign{}
id & t & x \\
\midrule\noalign{}
\endhead
\bottomrule\noalign{}
\endlastfoot
1 & 1 & 10 \\
1 & 2 & 20 \\
1 & 3 & 30 \\
2 & 1 & 20 \\
2 & 2 & 30 \\
2 & 3 & 40 \\
\end{longtable}

\begin{longtable}[]{@{}llll@{}}
\caption{Data in Wide Format}\label{tbl-datawide}\tabularnewline
\toprule\noalign{}
id & x1 & x2 & x3 \\
\midrule\noalign{}
\endfirsthead
\toprule\noalign{}
id & x1 & x2 & x3 \\
\midrule\noalign{}
\endhead
\bottomrule\noalign{}
\endlastfoot
1 & 10 & 20 & 30 \\
2 & 20 & 30 & 40 \\
\end{longtable}

\section{Simulated Multilevel Longitudinal
Data}\label{simulated-multilevel-longitudinal-data}

For the discussion below, I use a longitudinal version of the simulated
data that has multiple rows of data per family.

\begin{longtable}[]{@{}
  >{\centering\arraybackslash}p{(\columnwidth - 12\tabcolsep) * \real{0.1389}}
  >{\centering\arraybackslash}p{(\columnwidth - 12\tabcolsep) * \real{0.0833}}
  >{\centering\arraybackslash}p{(\columnwidth - 12\tabcolsep) * \real{0.1250}}
  >{\centering\arraybackslash}p{(\columnwidth - 12\tabcolsep) * \real{0.0833}}
  >{\centering\arraybackslash}p{(\columnwidth - 12\tabcolsep) * \real{0.1528}}
  >{\centering\arraybackslash}p{(\columnwidth - 12\tabcolsep) * \real{0.2083}}
  >{\centering\arraybackslash}p{(\columnwidth - 12\tabcolsep) * \real{0.0556}}@{}}

\caption{\label{tbl-simulatedlongitudinaldata}Simulated Longitudinal
Multilevel Data}

\tabularnewline

\caption{Table continues below}\tabularnewline
\toprule\noalign{}
\begin{minipage}[b]{\linewidth}\centering
country
\end{minipage} & \begin{minipage}[b]{\linewidth}\centering
HDI
\end{minipage} & \begin{minipage}[b]{\linewidth}\centering
family
\end{minipage} & \begin{minipage}[b]{\linewidth}\centering
id
\end{minipage} & \begin{minipage}[b]{\linewidth}\centering
identity
\end{minipage} & \begin{minipage}[b]{\linewidth}\centering
intervention
\end{minipage} & \begin{minipage}[b]{\linewidth}\centering
t
\end{minipage} \\
\midrule\noalign{}
\endfirsthead
\toprule\noalign{}
\begin{minipage}[b]{\linewidth}\centering
country
\end{minipage} & \begin{minipage}[b]{\linewidth}\centering
HDI
\end{minipage} & \begin{minipage}[b]{\linewidth}\centering
family
\end{minipage} & \begin{minipage}[b]{\linewidth}\centering
id
\end{minipage} & \begin{minipage}[b]{\linewidth}\centering
identity
\end{minipage} & \begin{minipage}[b]{\linewidth}\centering
intervention
\end{minipage} & \begin{minipage}[b]{\linewidth}\centering
t
\end{minipage} \\
\midrule\noalign{}
\endhead
\bottomrule\noalign{}
\endlastfoot
1 & 69 & 1 & 1.1 & 1 & 0 & 1 \\
1 & 69 & 1 & 1.1 & 1 & 0 & 2 \\
1 & 69 & 1 & 1.1 & 1 & 0 & 3 \\
1 & 69 & 2 & 1.2 & 1 & 1 & 1 \\
1 & 69 & 2 & 1.2 & 1 & 1 & 2 \\
1 & 69 & 2 & 1.2 & 1 & 1 & 3 \\

\end{longtable}

\begin{longtable}[]{@{}
  >{\centering\arraybackslash}p{(\columnwidth - 4\tabcolsep) * \real{0.3056}}
  >{\centering\arraybackslash}p{(\columnwidth - 4\tabcolsep) * \real{0.1250}}
  >{\centering\arraybackslash}p{(\columnwidth - 4\tabcolsep) * \real{0.1389}}@{}}

\caption{\label{tbl-simulatedlongitudinaldata}Simulated Longitudinal
Multilevel Data}

\tabularnewline

\toprule\noalign{}
\begin{minipage}[b]{\linewidth}\centering
physical\_punishment
\end{minipage} & \begin{minipage}[b]{\linewidth}\centering
warmth
\end{minipage} & \begin{minipage}[b]{\linewidth}\centering
outcome
\end{minipage} \\
\midrule\noalign{}
\endhead
\bottomrule\noalign{}
\endlastfoot
3 & 3 & 57.47 \\
3 & 4 & 55.06 \\
1 & 2 & 58.77 \\
2 & 1 & 50.1 \\
3 & 0 & 53.31 \\
3 & 1 & 49.79 \\

\end{longtable}

Since I will be discussing the estimation of a \emph{longitudinal}
model, it is often useful to graph the outcome variable against time.

\begin{figure}

\centering{

\includegraphics{longitudinal_files/figure-pdf/fig-data2-1.pdf}

}

\caption{\label{fig-data2}Graph of Simulated Longitudinal Data}

\end{figure}%

\section{The Equation}\label{the-equation-1}

When data are in \emph{long} format, the following equation is
applicable. Observe that the model below is a \emph{three level} model
where \emph{timepoints} are nested inside \emph{families} which in turn
are nested inside \emph{countries}. A simpler two level model with
\emph{timepoints} nested inside \emph{families} would also be possible
to estimate.

\begin{equation}\phantomsection\label{eq-MLM-longitudinal}{\text{outcome}_{itj} = \beta_0 + \beta_1 \text{parental warmth}_{itj} + \beta_2 \text{physical punishment}_{itj} + \beta_3 \text{time}_{itj} \ + }\end{equation}

\[\beta_4 \text{identity}_{itj} + \beta_5 \text{intervention}_{itj} + \beta_6 \text{HDI}_{itj} +\]

\[u_{0j} + u_{1j} \times \text{parental warmth}_{itj} \ + \]

\[v_{0i} + v_{1i} \times \text{time}_{itj} + e_{itj}\] Here I include a
random slope (\(u_{1j}\)) at the country level for parental warmth, as
well as a random slope (\(v_{1i}\)) at the family level for time.

As before, the random slope for parental warmth,
\(u_{1j} \times \text{parental warmth}_{ij}\) suggests allows us to
estimate whether the relationship between parental warmth and the
outcome varies across countries. The random slope for time,
\(v_{1i} \times t\), allows us to estimate whether time trajectories
(the slope for time) vary across families.

\section{Growth Trajectories}\label{sec-growthtrajectories}

In longitudinal multilevel models, the variable for \emph{time} assumes
a special role as we are often visualizing a \emph{growth trajectory}
over the course of time. \index{growth trajectories}

Imagine a model as follows where \emph{identity} is a (1/0) variable for
membership in one of two groups:

\[\text{outcome} = \beta_0 + \beta_t \text{time} + \beta_\text{identity} \text{identity} + \beta_\text{interaction} \text{identity} \times \text{time} + u_{0i} + e_{it}\]
Then, each identity group has its own intercept and time trajectory:

\begin{longtable}[]{@{}
  >{\raggedright\arraybackslash}p{(\columnwidth - 4\tabcolsep) * \real{0.1000}}
  >{\raggedright\arraybackslash}p{(\columnwidth - 4\tabcolsep) * \real{0.5000}}
  >{\raggedright\arraybackslash}p{(\columnwidth - 4\tabcolsep) * \real{0.4000}}@{}}
\caption{Slope and Intercept for Each
Group}\label{tbl-trajectory}\tabularnewline
\toprule\noalign{}
\begin{minipage}[b]{\linewidth}\raggedright
Group
\end{minipage} & \begin{minipage}[b]{\linewidth}\raggedright
Intercept
\end{minipage} & \begin{minipage}[b]{\linewidth}\raggedright
Slope (Time Trajectory)
\end{minipage} \\
\midrule\noalign{}
\endfirsthead
\toprule\noalign{}
\begin{minipage}[b]{\linewidth}\raggedright
Group
\end{minipage} & \begin{minipage}[b]{\linewidth}\raggedright
Intercept
\end{minipage} & \begin{minipage}[b]{\linewidth}\raggedright
Slope (Time Trajectory)
\end{minipage} \\
\midrule\noalign{}
\endhead
\bottomrule\noalign{}
\endlastfoot
0 & \(\beta_0\) & \(\beta_t\) \\
1 & \(\beta_0 + \beta_\text{identity}\) &
\(\beta_t + \beta_\text{interaction}\) \\
\end{longtable}

\begin{quote}
Thus, in longitudinal multilevel models, \emph{main effects} modify the
\emph{intercept} of the time trajectory, while \emph{interactions with
time}, modify the \emph{slope} of the time trajectory.
\end{quote}

\begin{figure}

\centering{

\includegraphics{longitudinal_files/figure-pdf/fig-trajectory-1.pdf}

}

\caption{\label{fig-trajectory}Hypothetical Growth Trajectory}

\end{figure}%

\section{Regression With Simulated Multi-Country Longitudinal
Data}\label{sec-regressionlongitudinal}

\subsection{Unconditional Model}\label{sec-unconditional-longitudinal}

As we did earlier (Section~\ref{sec-unconditional}), we estimate an
unconditional model with no independent variables.
\index{unconditional model}

\begin{longtable}[]{@{}lll@{}}
\toprule\noalign{}
& 1 & \\
\midrule\noalign{}
\endhead
\bottomrule\noalign{}
\endlastfoot
\_cons & 53.378 & ** \\
var(\_cons) & 3.232 & \\
var(\_cons) & 11.724 & \\
var(e) & 28.234 & \\
Number of observations & 9000 & \\
\end{longtable}

** p\textless.01, * p\textless.05

In a three level model, there are two intraclass correlation
coefficients (StataCorp, 2023b). \index{ICC} The formulas for the
Intraclass Correlation Coefficient (ICC) are given by (StataCorp,
2023b):

\begin{equation}\phantomsection\label{eq-ICCunconditional3A}{\text{ICC} = \frac{\text{var}(u_{0j})}{\text{var}(u_{0j}) + \text{var}(v_{0i}) + \text{var}(e_{ij})}}\end{equation}

Following StataCorp (2023b), Equation~\ref{eq-ICCunconditional3A} is the
correlation of responses for person-timepoints from the same country but
different persons.

\begin{equation}\phantomsection\label{eq-ICCunconditional3B}{\text{ICC} = \frac{\text{var}(u_{0j}) + \text{var}(v_{0i})}{\text{var}(u_{0j}) + \text{var}(v_{0i}) + \text{var}(e_{ij})}}\end{equation}

Again, closely following StataCorp (2023b),
Equation~\ref{eq-ICCunconditional3B} is the correlation of responses for
person-timepoints from the same country and same person.

The ICCs suggest that almost 8\% of the variation in the outcome is
within time points for different individuals within the same country,
while almost 35\% of the variation in the outcome is within time points
for the same individual within the same country.

\subsection{Main Effects Only}\label{main-effects-only}

\begin{longtable}[]{@{}lll@{}}
\toprule\noalign{}
& 1 & \\
\midrule\noalign{}
\endhead
\bottomrule\noalign{}
\endlastfoot
t & 0.944 & ** \\
warmth & 0.913 & ** \\
physical\_punishment & -1.008 & ** \\
identity & & \\
1 & -0.128 & \\
intervention & & \\
1 & 0.859 & ** \\
HDI & -0.001 & \\
\_cons & 50.467 & ** \\
var(warmth) & 0.011 & \\
var(\_cons) & 3.167 & \\
var(\_cons) & 8.387 & \\
var(t) & 0.000 & \\
var(e) & 26.027 & \\
Number of observations & 9000 & \\
\end{longtable}

** p\textless.01, * p\textless.05

Examining the regression results, the results of the model suggest that
child outcomes improve over time. Better child outcomes are again
associated with parental \texttt{warmth}, and parental use of
\texttt{physical\_punishment} is associated with reduced child outcomes.
\texttt{identity} is not associated with the outcome. However, the
\texttt{intervention} is associated with increases in the outcome.
\texttt{HDI} is again not associated with outcomes.

\subsection{Interactions With Time}\label{sec-interactions-with-time}

As discussed in Section~\ref{sec-growthtrajectories}, we will likely
wish to model not only associations of other independent variables with
the intercept of the time trajectory, but also associations of other
independent variables with the slope of the time trajectory.
Accordingly, we modify Equation~\ref{eq-MLM-longitudinal} so that it
includes these interactions. Below, I add the letter \(B\) to some
\(\beta\) coefficients to denote that they are a second coefficient
estimating the \emph{interaction} of that variable with time.

\begin{equation}\phantomsection\label{eq-MLM-longitudinal}{\text{outcome}_{itj} = \beta_0 + \beta_1 \text{parental warmth}_{itj} + \beta_2 \text{physical punishment}_{itj} + \beta_3 \text{time}_{itj} \ + }\end{equation}

\[\beta_{1B} \text{parental warmth}_{itj} \times \text{time}_{itj} + \beta_{2B} \text{physical punishment}_{itj} \times \text{time}_{itj} +\]

\[\beta_4 \text{identity}_{itj} + \beta_5 \text{intervention}_{itj} + \beta_6 \text{HDI}_{itj} +\]

\[\beta_{4B} \text{identity}_{itj} \times \text{time}_{itj} + \beta_{5B} \text{intervention}_{itj} \times \text{time}_{itj} + \beta_{6B} \text{HDI}_{itj} \times \text{time}_{itj} +\]

\[u_{0j} + u_{1j} \times \text{parental warmth}_{itj} \ + \]

\[v_{0i} + v_{1i} \times \text{time}_{itj} + e_{itj}\]

\begin{longtable}[]{@{}lll@{}}
\toprule\noalign{}
& 1 & \\
\midrule\noalign{}
\endhead
\bottomrule\noalign{}
\endlastfoot
t & 0.758 & * \\
warmth & 0.817 & ** \\
physical\_punishment & -1.009 & ** \\
identity & & \\
1 & -0.239 & \\
intervention & & \\
1 & 0.661 & * \\
HDI & 0.001 & \\
t \# warmth & 0.048 & \\
t \# physical\_punishment & 0.001 & \\
identity \# t & & \\
1 & 0.055 & \\
intervention \# t & & \\
1 & 0.099 & \\
t \# HDI & -0.001 & \\
\_cons & 50.836 & ** \\
var(warmth) & 0.011 & \\
var(\_cons) & 3.170 & \\
var(\_cons) & 8.392 & \\
var(t) & 0.000 & \\
var(e) & 26.016 & \\
Number of observations & 9000 & \\
\end{longtable}

** p\textless.01, * p\textless.05

Examining the regression results, the results of the model again suggest
that child outcomes improve over time. Better child outcomes are again
associated with parental \texttt{warmth}, and parental use of
\texttt{physical\_punishment} is associated with reduced child outcomes.
\texttt{identity} is again not associated with outcomes, while
participation in the \texttt{intervention} is associated with
improvements in outcomes. \texttt{HDI} is again not associated with
outcomes.

Examining the interaction terms, we find that none of these variables
modify the time trajectory of the outcome.

\begin{tcolorbox}[enhanced jigsaw, breakable, colbacktitle=quarto-callout-tip-color!10!white, leftrule=.75mm, bottomrule=.15mm, opacityback=0, colframe=quarto-callout-tip-color-frame, coltitle=black, rightrule=.15mm, title=\textcolor{quarto-callout-tip-color}{\faLightbulb}\hspace{0.5em}{Interactions And Random Slopes in Longitudinal Models}, colback=white, opacitybacktitle=0.6, toprule=.15mm, titlerule=0mm, bottomtitle=1mm, left=2mm, toptitle=1mm, arc=.35mm]

Having now discussed both \emph{random slopes}
(Section~\ref{sec-conditional-model}), and \emph{interaction terms}
(Section~\ref{sec-interactions-with-time}) in longitudinal models--and
having seen that both \emph{random slopes} and \emph{interaction terms}
involve changes in slope--one might again ask what is the difference
between including a variable as a random slope or as an interaction
term. When a variable is included as an \emph{interaction term}, it
indicates that we are attempting to estimate the change in slope
associated with a \emph{measured} variable. When we include a variable
as a \emph{random slope}, it indicates that we are attempting to
estimate the change in slope associated with \emph{unmeasured}
variables. See in this regard
Section~\ref{sec-measured-and-unmeasured-variation}

\end{tcolorbox}

\section{Autocorrelation}\label{autocorrelation}

When data are ordered by a time variable \(t\), it is possible that
observations that are closer together in time will have a higher
correlation than observations that are distant in time. In the simplest
example, \(e_{i, t=k}\) may be correlated with \(e_{i, t=k-1}\). This
phenomenon is known as \emph{autocorrelation}. As Hooper (2022) would
suggest, it may make sense to assume that the correlation between
observations ``decays with increasing separation in time''.

Most software programs for multilevel modeling allow one to incorporate
measures of autocorrelation so that, e.g., \(e_{i,t=3}\) is allowed to
be correlated with \(e_{i,t=2}\), which in turn can be correlated with
\(e_{i,t=1}\). More complex autocorrelation structures are usually also
possible (StataCorp, 2023a).

\section{Causal Inference}\label{sec-causality}

\subsection{The Importance of Causal
Reasoning}\label{the-importance-of-causal-reasoning}

Causal reasoning is sometimes considered to be a statistical--or even
overly technical--concern. Arguably, however, whenever one is using
research to make recommendations about \emph{interventions}, or
\emph{treatments}, or \emph{policies}, one is engaging in some form of
causal reasoning (Duncan \& Gibson-Davis, 2006). \index{causality}

\begin{tcolorbox}[enhanced jigsaw, breakable, colbacktitle=quarto-callout-tip-color!10!white, leftrule=.75mm, bottomrule=.15mm, opacityback=0, colframe=quarto-callout-tip-color-frame, coltitle=black, rightrule=.15mm, title=\textcolor{quarto-callout-tip-color}{\faLightbulb}\hspace{0.5em}{The Ubiquity of Causal Reasoning}, colback=white, opacitybacktitle=0.6, toprule=.15mm, titlerule=0mm, bottomtitle=1mm, left=2mm, toptitle=1mm, arc=.35mm]

If one is saying that implementing \emph{x} would result in beneficial
changes in \emph{y}, one is arguing--at least implicitly--that \emph{x}
is one of the \emph{causes} of \emph{y}.

\end{tcolorbox}

It then behooves one to be explicit about this chain of causal
reasoning. For example, to continue one of the substantive examples of
this book, if one is going to argue for programs, interventions, or
treatments that promote \emph{parental warmth}, or that discourage
parental use of \emph{physical punishment} with the aim of improving
children's \emph{mental health}, one must be at least reasonably sure
that \emph{parental warmth} and \emph{physical punishment} are
\emph{causes} of children's mental health.

In a statement salient for social research, Duncan \& Gibson-Davis
(2006) point out the logical inconsistency of writing that does not
rigorously address causal processes, but then goes on to suggest
interventions or treatment or policies:

\begin{quote}
``Developmental studies are usually careful to point out when their data
do not come from a randomized experiment. As with much of the
nonexperimental literature in developmental psychology, most of the
articles then go on to assert that, as a consequence, it is impossible
to draw causal inferences from the analysis. Indeed, much of their
language describing results is couched in terms of `associations'
between child care quality and child outcomes. It is not uncommon,
however, to see these papers make explicit statements about effects, and
others draw explicit policy conclusions. For instance, NICHD (1997, 876)
stated, `The interaction analyses provided evidence that high-quality
child care served a compensatory function for children whose maternal
care was lacking.' On the policy side, NICHD (2002c, 199) asserted,
`These findings provide empirical support for policies that improve
state regulations for caregiver training and child-staff ratios.'\,''
(Duncan \& Gibson-Davis, 2006)
\end{quote}

\begin{quote}
``One cannot have it both ways. Studies that do not aspire to causal
analysis should make no claim whatsoever about effects and draw no
policy conclusions. At the same time, it would be a terrible waste of
resources to conduct expensive longitudinal studies without attempting
to use them for causal modeling.'' (Duncan \& Gibson-Davis, 2006)
\end{quote}

\subsection{Randomized Controlled
Trials}\label{randomized-controlled-trials}

Randomized studies provide the best evidence about \emph{internal
validity} and causal relationships. \index{randomized controlled trials}
However, randomized studies have certain important limitations (Diener
et al., 2022). First of all--especially in a study with a smaller
sample--randomization may not always be perfect, and the control and
treatment groups may not be statistically equivalent. Secondly, because
randomized studies are costly to conduct, they may have small samples
and may be statistically underpowered. Smaller samples and underpowered
studies are more likely to generate false positive results than larger
samples (Button et al., 2013) \footnote{See
  \url{https://agrogan.shinyapps.io/Thinking-Through-Bayes/} for a
  demonstration of this idea from a Bayesian perspective.}. Further, and
importantly, because of ethical concerns some studies can not be
conducted with randomization (Diener et al., 2022). For example, in the
study of parenting and child development, children cannot ethically be
assigned to parents with different styles of parenting and followed over
the long term (Heilmann et al., 2021). Finally, and crucially, because
of their often small samples, and their often rigorous exclusion
criteria, randomized studies may have high internal validity, but much
lower external validity, or generalization to larger populations (Diener
et al., 2022). This issue of generalizability becomes increasingly
salient, when we are reminded of the fact that so little social and
psychological research has been conducted outside of North American
contexts (Draper et al., 2022; Henrich et al., 2010). Thus, methods that
provide rigorous causal estimation with observational methods are
necessary (Diener et al., 2022).

\subsection{Observational Studies and
Causality}\label{observational-studies-and-causality}

Because of the assumed superiority of studies that employ randomization,
it is sometimes maintained that \emph{correlation is not causation} and
that studies that do not make use of randomization are \emph{only
observational} and \emph{correlational}, and that results from
observational studies cannot be used to support causal conclusions.
\index{observational studies and causality} However, in important
reviews Waddington et al. (2022) and Dahabreh \& Bibbins-Domingo (2024)
suggest that studies using appropriately quantitative methods can
provide causally robust conclusions. Heilmann et al. (2021) make a
similar assertion with specific regard to studies of physical punishment
and child outcomes, arguing that observational studies that make use of
appropriately advanced quantitative methods can make causally robust
conclusions about the effects of physical discipline.

It is necessary to make use of broadly representative observational data
sets, and appropriately sophisticated quantitative methods, to make
causally robust conclusions from observational data that are applicable
across diverse populations.

\subsection{Formal Criteria of
Causality}\label{formal-criteria-of-causality}

For x to be a cause of y, one needs the following 3 things to be true
(Holland, 1986).

\begin{enumerate}
\def\labelenumi{\arabic{enumi}.}
\tightlist
\item
  \emph{x} is (are) associated with (correlated with) \emph{y}.
\item
  \emph{x} come(s) before \emph{y} in time.
\item
  \emph{z}--or other factors--cannot explain the association of
  (correlation of) \emph{x} and \emph{y}.
\end{enumerate}

\begin{figure}

\centering{

\includegraphics[width=3.46in,height=\textheight]{fig-causality.png}

}

\caption{\label{fig-causality}Formal Criteria of Causality}

\end{figure}%

\begin{quote}
If \emph{z} is omitted from the regression model, then the estimates for
\(x \rightarrow y\) (i.e.~\(\beta_{x \rightarrow y}\)) will be biased.
In a common scenario, \(\beta_{x \rightarrow y}\) may be an
over-estimate of the effect, and statistical significance of
\(\beta_{x \rightarrow y}\) may represent a false positive.
\end{quote}

It is likely useful to restate the above abstract statements in terms of
the substantive issues that I have been considering so far in this book.

For \emph{parenting} to be a cause of \emph{child outcomes}, one needs
the following 3 things to be true (Holland, 1986).

\begin{enumerate}
\def\labelenumi{\arabic{enumi}.}
\tightlist
\item
  \emph{parenting} is (are) associated with (correlated with)
  \emph{child outcomes}.
\item
  \emph{parenting} come(s) before \emph{child outcomes} in time.
\item
  \emph{SES}, \emph{community characteristics}--or other factors--cannot
  explain the association of (correlation of) \emph{parenting} and
  \emph{child outcomes}.
\end{enumerate}

\begin{figure}

\centering{

\includegraphics[width=3.46in,height=\textheight]{fig-causalitysubstantive.png}

}

\caption{\label{fig-causalitysubstantive}Formal Criteria of Causality: A
Substantive Example}

\end{figure}%

If \emph{other factors} are omitted from the regression model, then the
estimates for \(\text{parenting} \rightarrow \text{child outcome}\)
(i.e.~\(\beta_{\text{parenting} \rightarrow \text{child outcome}}\))
will be biased. In a common scenario,
\(\beta_{\text{parenting} \rightarrow \text{child outcome}}\) may be an
over-estimate of the effect, and statistical significance of
\(\beta_{\text{parenting} \rightarrow \text{child outcome}}\) may
represent a false positive.

\subsection{Simpson's Paradox}\label{sec-Simpsons}

Earlier, in Section~\ref{sec-multilevelstructure}, I referred to the
idea of \emph{multilevel structure} wherein failure to account for the
clustering of data--omission of \(u_0\) from the equation being
estimated--may lead to incorrect conclusions. A closely related
phenomenon is that of \emph{Simpson's Paradox} (Simpson, 1951) wherein
omission of a relevant \emph{covariate} (e.g.~\(z_{it}\) such as SES,
community characteristics, country level characteristics) may also lead
to dramatically incorrect results. The issue of omitted variables is a
crucially important--and sometimes underappreciated--issue that pervades
all statistical work.

Statistically, we imagine a situation where the true model is:

\[\text{child outcome}_{it} = \beta_0 + \beta_1 \text{parenting}_{it} +\]

\[\beta_2 \text{individual or family or community or country characteristic}_{it} + \]
\[u_{0i} + e_{it}\]

If \emph{individual or family or community or country characteristics}
in fact influence \emph{outcome}, but are not included in the
statistical model, perhaps because they are not measured in the data,
then the estimate of \(\beta_1\) for \emph{parenting} will be biased.
See Figure~\ref{fig-Simpson} for an illustration. When possible
confounders are \emph{measured}, we can include those variables in the
statistical model. When possible confounders are \emph{unmeasured}, we
need to try to use methods that capture those \emph{unmeasured}
confounders.

\begin{figure}

\centering{

\includegraphics{longitudinal_files/figure-pdf/fig-Simpson-1.pdf}

}

\caption{\label{fig-Simpson}An Illustration of Simpson's Paradox}

\end{figure}%

\subsection{A Simpler Multilevel Model To Explore
Causality}\label{a-simpler-multilevel-model-to-explore-causality}

For purposes of explication of ideas about causal estimation, in this
section, I imagine a simpler equation where I am only considering the
clustering of \emph{person timepoints} within \emph{individual people},
and ignoring for the moment--again for the sake of exposition--the
clustering of \emph{individuals} within \emph{countries}.

After explication and comprehension of this model, however, it is a
simple matter to add back in the random effects for country level
clustering.

The appropriate multilevel model is below.

\begin{equation}\phantomsection\label{eq-MLM-simpler}{\text{outcome}_{it} = \beta_0 + \beta_1 \text{parental warmth}_{it} + \beta_2 \text{physical punishment}_{it} + \beta_3 \text{time}_{it} \ + }\end{equation}

\[\beta_4 \text{identity}_{it} + \beta_5 \text{intervention}_{it} + \beta_6 \text{HDI}_{it} \ +\]

\[v_{0i} + e_{it}\]

Note that in Equation~\ref{eq-MLM-simpler}, if one were estimating a
\emph{multilevel model}, one would consider the \(v_{0i}\) to be a
randomly varying parameter with a mean of 0, and a variance of
\(\sigma^2(v_{0i})\).

\subsection{Fixed Effects Regression}\label{fixed-effects-regression}

I can use the same equation:

\begin{equation}\phantomsection\label{eq-FE}{\text{outcome}_{it} = \beta_0 + \beta_1 \text{parental warmth}_{it} + \beta_2 \text{physical punishment}_{it} + \beta_3 \text{time}_{it} \ + }\end{equation}

\[\beta_4 \text{identity}_{it} + \beta_5 \text{intervention}_{it} + \beta_6 \text{HDI}_{it} \ +\]

\[v_{0i} + e_{it}\]

However, in Equation~\ref{eq-FE}, I now consider the \(v_{0i}\) to be
\emph{estimable} for each individual \(i\) in the data. In effect, the
\(v_{0i}\) become a unique indicator variable for each individual in the
data set. This is known as a \emph{fixed effects regression model}.
\index{fixed effects regression}

Recall the discussion in Section~\ref{sec-withinbetween}. In essence, in
the fixed effects regression model, I am only making use of the
variation within individuals, and not making use of the variation
between individuals.

Details are provided in Allison (2009) and Wooldridge (2010). StataCorp
(2023c) provides an exceptionally clear explication of the core idea of
fixed effects regression. The essential idea is that the fixed effects
model provides statistical control for all time invariant
characteristics of study participants, such as--as is often the case in
many data sets--their racial or ethnic identity, their neighborhood of
residence, or other characteristics which by definition are time
invariant, such as the region of the country or city in which a
respondent was born. Importantly, (Ma et al., 2018) note that:

\begin{quote}
``Another potential omitted variable is that of genetic predisposition,
in that observed neighborhood effects on child outcomes are possibly
attributable to a genetic heritage shared by parents and their child
(Caspi et al., 2000).''
\end{quote}

Such genetic heritage could be considered to be a time invariant
variable that, while unobserved, would be controlled for by a fixed
effects regression.

Thus, by ruling out many potential confounds, fixed effects regression
methods provide much more causally robust analyses, specifically because
they control for many more possible confounding variables than do
standard regression methods, including multilevel models, which are only
able to control for the variables that are measured in the study
\emph{and} that are included within the regression model.

However, a disadvantage of the fixed effects approach is that this
approach can not provide estimates for any time invariant characteristic
of study participants. Indeed, if one includes time invariant variables
into a fixed effects regression, they are automatically dropped from the
regression results as can be seen in the regression table below.

\begin{longtable}[]{@{}lllll@{}}
\toprule\noalign{}
& MLM & & FE & \\
\midrule\noalign{}
\endhead
\bottomrule\noalign{}
\endlastfoot
t & 0.943 & ** & 0.944 & ** \\
warmth & 0.913 & ** & 0.916 & ** \\
physical\_punishment & -0.982 & ** & -1.094 & ** \\
identity & & & & \\
1 & -0.116 & & & \\
intervention & & & & \\
1 & 0.886 & ** & & \\
HDI & 0.001 & & & \\
\_cons & 50.298 & ** & 50.988 & ** \\
var(\_cons) & 11.831 & & & \\
var(e) & 26.033 & & & \\
Number of observations & 9000 & & 9000 & \\
\end{longtable}

** p\textless.01, * p\textless.05

In comparing the multilevel model and the fixed effects regression, we
note a few salient difference. First, the fixed effects are similar to
the multilevel model coefficients. (Most often, the fixed effect
regression coefficients are attenuated versions of the multilevel model
coefficients, but not always.) The fixed effects regression coefficients
for variables that have some variation over time, provide estimates that
control for all time invariant variables in the model.

Second, estimates for any quantities that do not vary over time, in this
case, \texttt{identity} group membership, participation in the
\texttt{intervention}, and \texttt{HDI}, are not available from the
fixed effects regression.

\subsection{The Correlated Random Effects
Model}\label{the-correlated-random-effects-model}

The \emph{correlated random effects} model is based upon ideas first
developed by Mundlak (1978) and later explicated in Wooldridge (2010).
Antonakis et al. (2021) and Schunck (2013) provide very intuitive
explanations of this model. \index{correlated random effects model}

The central idea is that one can obtain estimates of both the time
invariant variables, and estimates for time varying variables. The key
idea is that for time varying variables, I include the \emph{individual}
level mean for that variable in the model. Thus, in the example below, I
include \(\beta_{1a}\overline{\text{parental warmth}}_{i}\) and
\(\beta_{2a}\overline{\text{physical punishment}}_{i}\). \footnote{The
  correlated random effects model can also be applied cross-sectionally,
  but the model is much easier to explicate in the longitudinal context.}
This is similar in approach to what is described in
Section~\ref{sec-withinbetween}, however, here I am simply adding the
group level mean to the equation instead of decomposing independent
variables into within and between components.

\begin{equation}\phantomsection\label{eq-CRE}{\text{outcome}_{it} = \beta_0 + \beta_1 \text{parental warmth}_{it} + \beta_{1a}\overline{\text{parental warmth}}_{i} \ + }\end{equation}

\[\beta_2 \text{physical punishment}_{it} + \beta_{2a}\overline{\text{physical punishment}}_{i} + \]

\[\beta_3 \text{time}_{it} \ + \]

\[\beta_4 \text{identity}_{it} + \beta_5 \text{intervention}_{it} + \beta_6 \text{HDI}_{it} \ +\]

\[v_{0i} + e_{ij}\]

By including these parameters, I obtain estimates for the time varying
variables that are \emph{equivalent} to what I would obtain from a fixed
effects regression (Schunck, 2013).

\begin{longtable}[]{@{}lllllll@{}}
\toprule\noalign{}
& MLM & & FE & & CRE & \\
\midrule\noalign{}
\endhead
\bottomrule\noalign{}
\endlastfoot
t & 0.943 & ** & 0.944 & ** & 0.944 & ** \\
warmth & 0.913 & ** & 0.916 & ** & 0.916 & ** \\
physical\_punishment & -0.982 & ** & -1.094 & ** & -1.094 & ** \\
identity & & & & & & \\
1 & -0.116 & & & & -0.116 & \\
intervention & & & & & & \\
1 & 0.886 & ** & & & 0.890 & ** \\
HDI & 0.001 & & & & 0.001 & \\
mean\_warmth & & & & & -0.005 & \\
mean\_physicalpunishment & & & & & 0.192 & \\
\_cons & 50.298 & ** & 50.988 & ** & 50.086 & ** \\
var(\_cons) & 11.831 & & & & & \\
var(e) & 26.033 & & & & 26.024 & \\
var(\_cons) & & & & & 11.824 & \\
Number of observations & 9000 & & 9000 & & 9000 & \\
\end{longtable}

** p\textless.01, * p\textless.05

Note a couple of things from this table. First, results from the
correlated random effects model, and the fixed effects regression model
are exactly the same for \emph{time varying} variables, \texttt{t},
\texttt{warmth}, and \texttt{physical\_punishment}. Again, these
coefficients for \emph{time varying} variables are estimated with
statistical control for all time invariant characteristics of study
subjects, whether those characteristics are observed, or unobserved.
Secondly, unlike the fixed effects regression, coefficients for
\emph{time invariant} variables, e.g.~\texttt{identity} group,
participation in the \texttt{intervention}, \texttt{HDI}, mean levels of
\texttt{warmth}, and mean levels of \texttt{physical\_punishment}, are
provided, while they would not not provided in the fixed effects model.

\bookmarksetup{startatroot}

\chapter{Multilevel Logistic Regression}\label{sec-logistic}

\begin{quote}
``We have peered into a new world and have seen that it is more
mysterious and more complex than we had imagined.'' (Rubin, 1997)
\index{Vera Rubin}
\end{quote}

\section{Introduction}\label{introduction-2}

New forms of the multilevel model are required when we have outcomes
that are not continuous. Let us imagine, for example, that we have a
situation in which our continuous outcome is now categorized into two
groups. For example, we might imagine that there is some sort of
diagnostic cutoff. Scores greater than this cutoff are assigned to one
group (\texttt{1}), while scores lower than the cutoff are assigned to
another group (\texttt{0}).

No doubt this scenario is more likely when we consider an
\emph{undesirable} outcome like depression or anxiety. Higher levels of
depression or anxiety might be greater than some diagnostic cutoff,
meriting a score of \texttt{1}, \emph{meets criteria for a diagnosis},
while scores lower than that diagnostic cutoff might receive a score of
\texttt{0}, \emph{does not meet diagnostic criteria}. However, in
keeping with our characterization of the outcome in the simulated data
employed in this book as desirable or beneficial, we may also imagine a
situation in which \texttt{1} is assigned to sufficiently high levels of
some desirable or beneficial outcome that exceed some threshold value,
while scores below that value are assigned to be \texttt{0}.

\begin{figure}

\centering{

\includegraphics{logistic_files/figure-pdf/fig-dichotomized-1.pdf}

}

\caption{\label{fig-dichotomized}A Continuous Outcome Dichotomized}

\end{figure}%

Pedagogically, this new categorical outcome satisfies our need to have a
dichotomous outcome to explain a new group of models--logistic
regression models--that are suitable for such outcomes. It is sometimes
considered to be a statistical rule of thumb that we should never
dichotomize a continuous outcome. It is definitely true that when we
dichotomize an outcome we lose a certain amount of information in the
data about the variation, heterogeneity, or diversity in the outcome.
Such a loss of information may reduce our ability to obtain
statistically significant results. At the same time, certain numerical
cutoffs may have important substantive meanings, clinical meanings, or
policy meanings. Additionally dichotomous outcomes may prove more
intuitive for readers of our work to understand than are continuous
outcomes. While there are sometimes quite strongly held statistical
opinions that continuous variables should never be dichotomized (Senn,
2005), my own personal belief is that sometimes variables dichotomized
into categories are more easily communicated to practitioners,
clinicians and community members, and in these cases, the importance of
clear communication outweighs the statistical arguments against
dichotomization.

For example, I have already considered higher levels of anxiety or
depression that may exceed clinically important cutoffs implying a
diagnosis \texttt{1} versus \texttt{0} of depression or anxiety. Levels
of income below a certain threshold--whether a country specific
threshold or some globally relevant threshold--are considered to be in
poverty\footnote{For example the World Bank (World Bank, 2024) considers
  \$2.15 per person per day to be a globally relevant indicator of
  extreme poverty.}, while individuals above that income threshold are
considered to be not in poverty (\texttt{1} versus \texttt{0}). In the
example considered in detail in this chapter, families with a child
having a beneficial outcome at a certain level might be considered to
have a child satisfying a certain minimal level of psychological
well-being.

In contrast, some outcomes are naturally dichotomous and do not arise
from dichotomizing a continuous variable: born versus not born; married
or partnered versus single; alive versus dead; entered a program; exited
a program; conflict or protest occurred versus conflict or protest did
not occur.

\section{Two Strategies}\label{two-strategies}

I now consider two possible strategies from modeling dichotomous
outcomes: the \emph{linear multilevel model} that we have been
considering so far in this book; and a \emph{multilevel logistic
regression} model designed specifically for dichotomous outcomes. The
easiest way to consider and compare these models is to begin with visual
presentation.

\begin{figure}

\centering{

\includegraphics{logistic_files/figure-pdf/fig-linear-logistic-1.pdf}

}

\caption{\label{fig-linear-logistic}Linear and Logistic Regression}

\end{figure}%

In each model, the equivalent pattern of dots represents some
dichotomous outcome (e.g.~birth, death, satisfies diagnostic criteria)
that becomes more likely as some independent variable increases in
value. At higher levels of the independent variable, the outcome is
almost exclusively \texttt{1}. At lower levels of the independent
variable, the outcome is almost exclusively \texttt{0}. In the middle
range of the independent variable, there is a mixture of \texttt{1} and
\texttt{0}. One could try to estimate this dichotomous outcome with a
straight line, or linear model, as depicted in the left hand panel of
Figure~\ref{fig-linear-logistic}. This might be roughly plausible, and
this procedure is termed a \emph{linear probability model} (Agresti \&
Finlay, 1997; Long \& Freese, 2014) which I will not discuss in more
detail here. Suffice it to say that several problems emerge for a linear
probability model: the model predicts values of the outcome greater than
\texttt{1} and less than \texttt{0}; and a linear model estimates a
constant association between changes in \emph{x} and changes in \emph{y}
when this is not appropriate (Figure~\ref{fig-marginal-changes}).

\begin{figure}

\centering{

\includegraphics{logistic_files/figure-pdf/fig-marginal-changes-1.pdf}

}

\caption{\label{fig-marginal-changes}Marginal Changes in \emph{y} in
Logistic Regression}

\end{figure}%

\section{Probabilities and Odds}\label{probabilities-and-odds}

To being to develop a \emph{multilevel logistic regression} model, it is
useful to think about ideas of \emph{probabilities} and \emph{odds}. One
possible definition of a probability is that a probability is the long
run fraction of times that an event occurs in series of events (Freedman
et al., 1991). \index{probability}

For example, if there are 100 possible occasions to experience an
event--such as to experience happiness--and happiness only occurs 10
times out of a possible 100, then we might say that the
\emph{probability} or \emph{risk} of happiness is .10. If happiness
occurs 90 times out of a possible 100 occurrences, then we would say
that the \emph{probability} or \emph{risk} of happiness is .90.
\index{risk}

In contrast, the \emph{odds} of an event are equivalently the
\emph{count} of times that an event occurs weighed against the
\emph{count} of times that the event does not occur, or the
\emph{probability} that an event occurs weighed against the
\emph{probability} that an event does not occur (Viera, 2008).

\begin{equation}\phantomsection\label{eq-odds-definition}{\text{odds} = \frac{\text{count of occurrences}}{\text{count of occurrences}} = \frac{p(\text{occurrence})}{p(\text{non-occurrence})} =}\end{equation}

\[\frac{p(\text{occurrence})}{1-p(\text{occurrence})}\]

Thus, if happiness occurs 10 times out of a possible 100, then the
\emph{odds} of happiness occurring are \texttt{10/90\ =\ 1/9\ =\ .11}.
Similarly if happiness occurs 90 times, then the \emph{odds} of
happiness occurring are \texttt{90/10\ =\ 9/1\ =\ 9} (Viera, 2008).
\index{odds}

\begin{longtable}[]{@{}
  >{\centering\arraybackslash}p{(\columnwidth - 8\tabcolsep) * \real{0.2500}}
  >{\centering\arraybackslash}p{(\columnwidth - 8\tabcolsep) * \real{0.2361}}
  >{\centering\arraybackslash}p{(\columnwidth - 8\tabcolsep) * \real{0.3056}}
  >{\centering\arraybackslash}p{(\columnwidth - 8\tabcolsep) * \real{0.0972}}
  >{\centering\arraybackslash}p{(\columnwidth - 8\tabcolsep) * \real{0.0972}}@{}}

\caption{\label{tbl-probabilities-odds}Probabilities and Odds}

\tabularnewline

\toprule\noalign{}
\begin{minipage}[b]{\linewidth}\centering
total occasions
\end{minipage} & \begin{minipage}[b]{\linewidth}\centering
event occurred
\end{minipage} & \begin{minipage}[b]{\linewidth}\centering
event did not occur
\end{minipage} & \begin{minipage}[b]{\linewidth}\centering
risk
\end{minipage} & \begin{minipage}[b]{\linewidth}\centering
odds
\end{minipage} \\
\midrule\noalign{}
\endhead
\bottomrule\noalign{}
\endlastfoot
100 & 10 & 90 & 0.1 & 0.11 \\
100 & 20 & 80 & 0.2 & 0.25 \\
100 & 30 & 70 & 0.3 & 0.43 \\
100 & 40 & 60 & 0.4 & 0.67 \\
100 & 50 & 50 & 0.5 & 1 \\
100 & 60 & 40 & 0.6 & 1.5 \\
100 & 70 & 30 & 0.7 & 2.33 \\
100 & 80 & 20 & 0.8 & 4 \\
100 & 90 & 10 & 0.9 & 9 \\
100 & 100 & 0 & 1 & Inf \\

\end{longtable}

A few things are worth noting about risks and odds. First, one might
think of the risks and the odds as simply different ways of thinking
about and talking about the chances that something will happen.

A classic diagram to more fully explicate some of these ideas appears in
many sources. (e.g. Viera, 2008)

\begin{equation}\phantomsection\label{eq-event-matrix}{\begin{matrix}
a & b \\
c & d 
\end{matrix}}\end{equation}

\begin{itemize}
\item
  Then a \emph{risk} is defined as \(a/(a + b)\).
\item
  \(c/(c + d)\) is also a \emph{risk}.
\item
  \(\frac{a/(a + b)}{c/(c + d)}\) is then a ratio of risks or \emph{risk
  ratio}.
\item
  \(a/b\) is the \emph{odds} of an event in the first row, while \(c/d\)
  is the \emph{odds} of an event in the second row.
\item
  \(\frac{a/b}{c/d}\) is then a ratio of odds, or \emph{odds ratio}.
\end{itemize}

\section{Logistic Regression}\label{logistic-regression}

Logistic regression begins with the idea that we are predicting the odds
of an event. A common way to write the odds is \(\frac{p(y)}{1-p(y)}\),
where \(p(y)\) is the probability of an outcome of interest, while
\(1-p(y)\) is the probability that the outcome did not occur.

We are estimating a function of these odds as a linear function of a set
of predictors, \(\beta_0 + \beta_1 x_1 + \beta_2 x_2 + u_{0j}\). Here
\(u_{0j}\) is the usual level 2 random intercept that I have been
including all along in multilevel models (Equation~\ref{eq-MLM}).
\index{logistic regression}

The expression \(\beta_0 + \beta_1 x_1 + \beta_2 x_2 + u_{0j}\) is a
linear function of the independent variables. In order make a linear
prediction appropriate, we take the logarithm of the odds, which
provides the equation for multilevel logistic regression (Long \&
Freese, 2014; Rabe-Hesketh \& Skrondal, 2022; Stock \& Watson, 2003).

\begin{equation}\phantomsection\label{eq-logistic-MLM}{\ln\Big(\frac{p(y)}{1-p(y)}\Big) = \beta_0 + \beta_1 x_1 + \beta_2 x_2 + u_{0j}}\end{equation}

Notice that while there is a level 2 error term, or random intercept
\(u_{0j}\), there is no level 1 error term (\(e_{ij}\)) as we saw in
Equation~\ref{eq-MLM}. This is because the use of probabilities \(p(y)\)
encompasses the idea of error in the logistic regression model.

\section{Odds Ratios}\label{odds-ratios}

In a logistic regression the dependent variable is the \emph{log-odds}
of the results (Equation~\ref{eq-logistic-MLM}). The log-odds is likely
to be a less than intuitive metric. We can obtain more intuitive results
if we \emph{exponentiate} both sides of Equation~\ref{eq-logistic-MLM}
(i.e.~raise both sides to the power of \(e\)).

\begin{equation}\phantomsection\label{eq-logistic-MLM-exponentiated}{\frac{p(y)}{1-p(y)} = e^{\beta_0 + \beta_1 x_1 + \beta_2 x_2 + u_{0j}}}\end{equation}

Through simple algebraic rules around exponents, this then becomes:

\begin{equation}\phantomsection\label{eq-logistic-MLM-exponentiated-multiplied}{\frac{p(y)}{1-p(y)}= e^{\beta_0} e^{\beta_1 x_1} e^{\beta_2 x_2} e^{u_{0j}}}\end{equation}

The quantity \(e^{\beta}\) is termed an \emph{odds ratio} and represents
the association of a 1 unit change in the independent variable with the
\emph{odds} of the outcome.

\section{\texorpdfstring{\(\beta\) Coefficients and Odds
Ratios}{\textbackslash beta Coefficients and Odds Ratios}}\label{beta-coefficients-and-odds-ratios}

It is worth thinking about the relationship between \(\beta\)
coefficients in a logistic regression, and the associated odds ratio.
Many times software only reports the odds ratios. There are simple rules
connecting \(\beta\) coefficients from regressions with the odds ratio.

\begin{longtable}[]{@{}
  >{\raggedright\arraybackslash}p{(\columnwidth - 4\tabcolsep) * \real{0.2361}}
  >{\raggedright\arraybackslash}p{(\columnwidth - 4\tabcolsep) * \real{0.1528}}
  >{\raggedleft\arraybackslash}p{(\columnwidth - 4\tabcolsep) * \real{0.1528}}@{}}
\caption{Logistic Regression Coefficients and Odds
Ratios}\label{tbl-coefficients-odds-ratios}\tabularnewline
\toprule\noalign{}
\begin{minipage}[b]{\linewidth}\raggedright
Substantively
\end{minipage} & \begin{minipage}[b]{\linewidth}\raggedright
\(\beta\)
\end{minipage} & \begin{minipage}[b]{\linewidth}\raggedleft
Odds Ratio
\end{minipage} \\
\midrule\noalign{}
\endfirsthead
\toprule\noalign{}
\begin{minipage}[b]{\linewidth}\raggedright
Substantively
\end{minipage} & \begin{minipage}[b]{\linewidth}\raggedright
\(\beta\)
\end{minipage} & \begin{minipage}[b]{\linewidth}\raggedleft
Odds Ratio
\end{minipage} \\
\midrule\noalign{}
\endhead
\bottomrule\noalign{}
\endlastfoot
x is associated with an increase in y & \(>0.0\) & \(>1.0\) \\
no association of x with y & \(0.0\) & \(1.0\) \\
x is associated with a decrease in y & \(<0.0\) & \(<1.0\) \\
\end{longtable}

In turn, thinking about odds ratios of different magnitudes can become
complicated.

\begin{longtable}[]{@{}
  >{\raggedright\arraybackslash}p{(\columnwidth - 2\tabcolsep) * \real{0.3472}}
  >{\raggedright\arraybackslash}p{(\columnwidth - 2\tabcolsep) * \real{0.5139}}@{}}
\caption{Describing Odds Ratios of Different
Magnitudes}\label{tbl-odds-ratios-different-magnitudes}\tabularnewline
\toprule\noalign{}
\begin{minipage}[b]{\linewidth}\raggedright
Odds Ratio for Group A Compared to Group B
\end{minipage} & \begin{minipage}[b]{\linewidth}\raggedright
Interpretations
\end{minipage} \\
\midrule\noalign{}
\endfirsthead
\toprule\noalign{}
\begin{minipage}[b]{\linewidth}\raggedright
Odds Ratio for Group A Compared to Group B
\end{minipage} & \begin{minipage}[b]{\linewidth}\raggedright
Interpretations
\end{minipage} \\
\midrule\noalign{}
\endhead
\bottomrule\noalign{}
\endlastfoot
.25 & The odds of the outcome for Group A were 25\% of the odds of Group
B. \\
& The odds of the outcome for Group A were 75\% lower than for the Group
B. \\
1.0 & The odds of the outcome for Group A were equivalent to (the same
as) the odds for Group B. \\
& There was no difference between Group A and Group B. \\
1.5 & The odds of the outcome for Group A were 1.5 times those for Group
B. \\
& The odds of the outcome for Group A were 50\% higher than for Group
B. \\
2.0 & The odds of the outcome for Group A were 100\% higher than those
for the Group B. (technically correct, but at the very least
confusing.) \\
& The odds of the outcome for Group A were 2.0 (twice) those for Group
B. \\
2.5 & The odds of the outcome for Group A were 2.5 times those for the
Group B. \\
& The odds of the outcome for Group A were 150\% higher than for Group
B. (technically correct, but at the very least confusing.) \\
\end{longtable}

\section{Regression With Simulated Multi-Country
Data}\label{sec-logistic-regression}

\subsection{Unconditional Model}\label{unconditional-model}

As I have done earlier (Section~\ref{sec-unconditional}), I first
estimate an unconditional model:

\begin{longtable}[]{@{}ll@{}}
\toprule\noalign{}
& 1 \\
\midrule\noalign{}
\endhead
\bottomrule\noalign{}
\endlastfoot
\_cons & 1.101 \\
var(\_cons{[}country{]}) & 0.231 \\
Number of observations & 3000 \\
\end{longtable}

** p\textless.01, * p\textless.05

This unconditional model provides us with the variance of the random
intercept (\(u_{0j}\)) which I use below to calculate the intra-class
correlation coefficient.

\subsection{Intra-Class Correlation
Coefficient}\label{intra-class-correlation-coefficient}

In logistic regression, as noted in Equation~\ref{eq-logistic-MLM},
there is no \(e_{ij}\) error term. In order to calculate the intra-class
correlation coefficient, we use the fact that a standard logistic
distribution has a variance of \(\frac{\pi^2}{3}\), or roughly 3.29
(Long \& Freese, 2014).

Hence, in a logistic regression, the equation for the intra-class
correlation coefficient becomes:

\begin{equation}\phantomsection\label{eq-ICC-logistic}{\text{ICC} = \frac{var(u_{0j})}{var(u_{0j}) + \frac{\pi^2}{3}}}\end{equation}

Substituting in the appropriate quantities suggests that approximately
6.6\% of the variation in the dichotomous outcome is potentially
explainable by the clustering of observations in countries.

\subsection{Conditional Model}\label{sec-conditional-logistic}

I now estimate a \emph{conditional} multilevel logistic regression model
\emph{with} independent variables.

\begin{longtable}[]{@{}lll@{}}
\toprule\noalign{}
& 1 & \\
\midrule\noalign{}
\endhead
\bottomrule\noalign{}
\endlastfoot
warmth & 1.293 & ** \\
physical\_punishment & 0.752 & ** \\
identity & & \\
1 & 0.952 & \\
intervention & & \\
1 & 1.192 & * \\
HDI & 0.999 & \\
\_cons & 0.912 & \\
var(\_cons{[}country{]}) & 0.290 & \\
Number of observations & 3000 & \\
\end{longtable}

** p\textless.01, * p\textless.05

The model demonstrates a number of results that are substantively
similar to those in Section~\ref{sec-conditional-model}. Here I use the
rubric described in Table~\ref{tbl-odds-ratios-different-magnitudes} to
first describe each result more generally, and then more
precisely\footnote{In describing these odds ratios, it is important to
  use the words odds, because, as demonstrated in
  Table~\ref{tbl-probabilities-odds}, odds are not the same thing as
  risks or probabilities!}. Increased levels of parental warmth are
associated with increased odds of being in the higher outcome category.
Specifically, the odds of the outcome increase 29\% with each 1 unit
increase in parental warmth. In contrast, increased levels of physical
punishment are associated with decreased odds of being in the higher
outcome category. Specifically, a 1 unit increase in the use of physical
punishment is associated with a 25\% decrease in the odds of the
outcome. The identity category in these data is not associated with
higher or lower odds of being in the higher outcome category.
Participation in the intervention is associated with higher odds of
being in the higher outcome category. Specifically, participation in the
intervention is associated with a 19\% increase in the odds of the
outcome. The Human Development Index is not associated with different
odds of being in the higher outcome category.

\section{Predicted Probabilities}\label{predicted-probabilities}

I have above discussed the ideas of estimating the association of
independent variables with the odds of an outcome. However, while odds
ratios are perhaps the most common way of reporting the results of
logistic regressions in the empirical literature, there are at least
several issues with odds ratios:

\begin{enumerate}
\def\labelenumi{\arabic{enumi}.}
\tightlist
\item
  Odds ratios provide a \emph{convenient} way of talking about the
  association of independent variables with categorical outcomes. At the
  same time, as shown in Table~\ref{tbl-probabilities-odds} the odds
  will overstate the risk, especially as the odds become larger.
\item
  Odds ratios provide one metric of association for the entire range of
  an independent variable \(x\). In contrast, as shown in
  Figure~\ref{fig-marginal-changes-2}, marginal changes in the outcome
  may be very small for very small values of the independent variable,
  and very large values of the independent variable, while marginal
  changes in the outcome are larger for intermediate ranges of the
  independent variable. Thus, odds ratios, while characterizing the
  overall logistic regression curve, do not provide a finer grained
  picture of changes in the dependent variable at different values of
  the independent variable.
\end{enumerate}

\begin{figure}

\centering{

\includegraphics{logistic_files/figure-pdf/fig-marginal-changes-2-1.pdf}

}

\caption{\label{fig-marginal-changes-2}Constant Odds Ratio But Different
Changes in Predicted Probability}

\end{figure}%

For these reasons it is often convenient to calculate predicted
probabilities at different values of the independent variables (CF Long
\& Freese, 2014). \index{predicted probabilities}

\begin{tcolorbox}[enhanced jigsaw, breakable, colbacktitle=quarto-callout-tip-color!10!white, leftrule=.75mm, bottomrule=.15mm, opacityback=0, colframe=quarto-callout-tip-color-frame, coltitle=black, rightrule=.15mm, title=\textcolor{quarto-callout-tip-color}{\faLightbulb}\hspace{0.5em}{The Algebra May Be Helpful}, colback=white, opacitybacktitle=0.6, toprule=.15mm, titlerule=0mm, bottomtitle=1mm, left=2mm, toptitle=1mm, arc=.35mm]

Below, I provide algebra for calculating predicted probabilities. This
algebra sometimes proves difficult for readers to follow but is provided
for the sake of completeness. Note that software often automates the
process of calculating these predicted probabilities.

\end{tcolorbox}

\subsection{Calculating Predicted
Probabilities}\label{calculating-predicted-probabilities}

To make the algebra easier, I first substitute \(Z\) for
\(\beta_0 + \beta_1 x_1 + \beta_2 x_2 + u_{0j}\):

\[\ln\Big(\frac{p(y)}{1-p(y)}\Big) = Z\] Then

\[\Big(\frac{p(y)}{1-p(y)}\Big) = e^Z\] \[p(y) = e^Z(1-p(y))\]
\[p(y) + e^Z(p(y)) = e^Z\] \[p(y) = \frac{e^Z}{1+e^Z}\] Finally,
substituting back in the original expression for \(Z\):

\[p(y) = \frac{e^Z}{1+e^Z}\]

\subsection{A Substantive Example}\label{a-substantive-example}

I now provide a substantive example. Recall the model estimated in
Section~\ref{sec-conditional-logistic}. This model suggests, as an
example, that participation in the intervention is associated with an
odds ratio of 1.192. Put another way, participation in the intervention
is associated with a .19 or 19\% increase in the odds of a beneficial
outcome. Bearing in mind the ideas presented in
Figure~\ref{fig-marginal-changes-2}, it is worth thinking about the
values of the predicted probability of a beneficial outcome at different
levels participation in the intervention.

According to the formulas presented above, which are again, usually
automatically calculated by statistical software, I obtain
Table~\ref{tbl-predicted-probabilities}.

\begin{longtable}[]{@{}
  >{\centering\arraybackslash}p{(\columnwidth - 6\tabcolsep) * \real{0.2000}}
  >{\centering\arraybackslash}p{(\columnwidth - 6\tabcolsep) * \real{0.3200}}
  >{\centering\arraybackslash}p{(\columnwidth - 6\tabcolsep) * \real{0.2400}}
  >{\centering\arraybackslash}p{(\columnwidth - 6\tabcolsep) * \real{0.2400}}@{}}

\caption{\label{tbl-predicted-probabilities}Predicted Probabilities Of A
Beneficial Outcome}

\tabularnewline

\toprule\noalign{}
\begin{minipage}[b]{\linewidth}\centering
intervention
\end{minipage} & \begin{minipage}[b]{\linewidth}\centering
predicted probability
\end{minipage} & \begin{minipage}[b]{\linewidth}\centering
absolute change
\end{minipage} & \begin{minipage}[b]{\linewidth}\centering
relative change
\end{minipage} \\
\midrule\noalign{}
\endhead
\bottomrule\noalign{}
\endlastfoot
0 & 0.51 & NA & NA \\
1 & 0.54 & 0.03 & 1.06 \\

\end{longtable}

Calculations in Table~\ref{tbl-predicted-probabilities} suggest that
when considered in terms of the predicted probability of a beneficial
outcome, participation in the intervention is associated with a 3.2\%
\emph{absolute} increase in the probability of a beneficial outcome, and
6.5\% \emph{relative} increase in the probability of a beneficial
outcome. Whether one considers \emph{absolute} or \emph{relative}
changes, these differences are far smaller than those suggested by the
odds ratios. Bear in mind that odds ratios often overstate the changes
in risk (Table~\ref{tbl-probabilities-odds}).

\bookmarksetup{startatroot}

\chapter{Models With More Complicated Structures}\label{sec-morelevels}

\begin{quote}
``The language we have in that world is not large enough for the
territory that we've already entered.'' (Whyte \& Tippett, 2016)
\index{David Whyte}
\end{quote}

\section{Introduction}\label{introduction-3}

In my experience, teaching about multilevel models, conducting research
using multilevel models, and reading and reviewing research articles by
others using multilevel models, the vast majority of multilevel modeling
is done with two level models. In this chapter, I discuss models with
more than two levels as well as models where the multiple levels are not
hierarchically nested, but are instead \emph{cross classified}.
\index{cross classified models}

In Chapter~\ref{sec-longitudinal}, I have already begun a discussion of
models with more than two levels. In that chapter, I discussed a
longitudinal model with three levels: time points nested inside
individuals inside countries. It is worth noting that in practice, many
longitudinal models are estimated as two level models with the time
points nested inside individuals. Indeed, had there been no
statistically significant geographic clustering by country in
Equation~\ref{eq-MLM-longitudinal}, this model could have been estimated
with a two level model.

Let us consider scenarios in which it is appropriate to estimate models
with four levels, or with levels that are cross-classified.

\section{Three Or More Levels}\label{sec-fourlevel}

\subsection{Data}\label{data}

Both the World Bank, and the United Nations divide the countries of the
world into a number of regions or sub-regions (Arel-Bundock et al.,
2018). Thus, I add five simulated regional United Nations groupings to
the \emph{longitudinal} data in order to illustrate the idea of a 4
level model.

\begin{tcolorbox}[enhanced jigsaw, breakable, colbacktitle=quarto-callout-caution-color!10!white, leftrule=.75mm, bottomrule=.15mm, opacityback=0, colframe=quarto-callout-caution-color-frame, coltitle=black, rightrule=.15mm, title=\textcolor{quarto-callout-caution-color}{\faFire}\hspace{0.5em}{Caution}, colback=white, opacitybacktitle=0.6, toprule=.15mm, titlerule=0mm, bottomtitle=1mm, left=2mm, toptitle=1mm, arc=.35mm]

Note that, to keep the example somewhat realistic with regard to the
regional groupings defined by the United Nations, I am adding 5
simulated regions to the data. Remember, however, that 5 groupings may
be too few categories to use as an additional random effect
(Section~\ref{sec-indicator-variables-random-effects}), and that it is
worth exploring whether it might be more appropriate to model these
regions as indicator variables, rather than as random effects
(Section~\ref{sec-indicator-variables-random-effects}).

\end{tcolorbox}

\subsection{Equation}\label{equation}

\begin{equation}\phantomsection\label{eq-MLM-fourlevel}{\text{outcome}_{itj} = \beta_0 + \beta_1 \text{parental warmth}_{itj} + \beta_2 \text{physical punishment}_{itj} + \beta_3 \text{time}_{itj} \ + }\end{equation}

\[\beta_4 \text{identity}_{itj} + \beta_5 \text{intervention}_{itj} + \beta_6 \text{HDI}_{itj} +\]

\[w_{0k} + u_{0j} + v_{0i} + e_{itjk}\]

All of the terms of this model are similar to those in
Equation~\ref{eq-MLM-longitudinal}, except that I now add a random
effect \(w_{0k}\) for United Nations sub-region.

\subsection{Unconditional Model}\label{unconditional-model-1}

\begin{longtable}[]{@{}lll@{}}
\toprule\noalign{}
& 1 & \\
\midrule\noalign{}
\endhead
\bottomrule\noalign{}
\endlastfoot
\_cons & 54.059 & ** \\
var(\_cons) & 4.173 & \\
var(\_cons) & 2.849 & \\
var(\_cons) & 11.724 & \\
var(e) & 28.234 & \\
Number of observations & 9000 & \\
\end{longtable}

** p\textless.01, * p\textless.05

\subsection{ICC}\label{icc}

Calculation of the ICC becomes conceptually and statistically more
complicated as we increase the number of levels. \index{ICC}

In order to think about this issue, it is useful to think first about
the conceptual issue. With a three level model, as noted in
Section~\ref{sec-unconditional-longitudinal}, we have two possible ICC's
to consider. One ICC measures the amount of clustering for observations
within the \emph{same} level unit but \emph{different} level two units,
while the second ICC measures the amount of clustering for observations
within the \emph{same} level three unit and the \emph{same} level two
unit.

The choices are illustrated in the first panel of Figure~\ref{fig-ICC}.
Extending this thinking, it then becomes more clear that the choices of
ICC for a four level model are therefore more complicated. One can
consider: observations that are only clustered within the same level 4
unit but different level 3 units and different level 2 units;
observations clustered within the same level 4 unit, the same level 3
unit and different level 2 units; and observations that are clustered
within the same level 4, 3 and 2 units.

\begin{figure}

\centering{

\includegraphics[width=0.75\textwidth,height=\textheight]{ICC.png}

}

\caption{\label{fig-ICC}ICCs in 3 Level And 4 Level Models}

\end{figure}%

In Section~\ref{sec-ICC} and
Section~\ref{sec-unconditional-longitudinal}, I used the notation
\(\text{var}\) to describe the variances of the different random
effects, which at that point in the discussion, I found to be intuitive
and therefore pedagogically useful. For the discussion below, I will use
the more compact but less intuitive notation \(\tau\) for the variance
components at level 2 and higher (e.g.~from the discussion in
Section~\ref{sec-unconditional-longitudinal},
\(\tau_3 = \text{var}(u_0)\) \(\tau_2 = \text{var}(v_0)\)) \footnote{The
  \(\tau\) notation for variance components is used in Lüdecke (2023)
  and Raudenbush \& Bryk (2002).}.

\begin{longtable}[]{@{}
  >{\centering\arraybackslash}p{(\columnwidth - 6\tabcolsep) * \real{0.1667}}
  >{\centering\arraybackslash}p{(\columnwidth - 6\tabcolsep) * \real{0.1250}}
  >{\centering\arraybackslash}p{(\columnwidth - 6\tabcolsep) * \real{0.1944}}
  >{\centering\arraybackslash}p{(\columnwidth - 6\tabcolsep) * \real{0.3472}}@{}}

\caption{\label{tbl-ICC-morelevels}ICCs in Models With More Than Two
Levels}

\tabularnewline

\toprule\noalign{}
\begin{minipage}[b]{\linewidth}\centering
Levels in Model
\end{minipage} & \begin{minipage}[b]{\linewidth}\centering
ICC at Level
\end{minipage} & \begin{minipage}[b]{\linewidth}\centering
Description
\end{minipage} & \begin{minipage}[b]{\linewidth}\centering
Formula
\end{minipage} \\
\midrule\noalign{}
\endhead
\bottomrule\noalign{}
\endlastfoot
3 & 3 & Same level 3 unit; different level 2 unit &
\(\frac{\tau_3}{\tau_3
+
\tau_2 +
e_{ij}}\) \\
3 & 2 & Same level 3 unit; same level 2 unit & \(\frac{\tau_3
+
\tau_2}{\tau_3
+
\tau_2 +
e_{ij}}\) \\
4 & 4 & Same level 4 unit; different level 3 and 2 unit &
\(\frac{\tau_4}{\tau_4
+
\tau_3 +
\tau_2 +
e_{ijk}}\) \\
4 & 3 & Same level 4 unit; same level 3 unit; different level 2 unit &
\(\frac{\tau_4
+
\tau_3}{\tau_4
+
\tau_3 +
\tau_2 +
e_{ijk}}\) \\
4 & 2 & Same level 4 unit; same level 3 unit; same level 2 unit &
\(\frac{\tau_4
+
\tau_3 +
\tau_2}{\tau_4
+\tau_3 +
\tau_2 +
e_{ijk}}\) \\

\end{longtable}

More generally Stata Corporation (StataCorp, 2023b) provides the
following formula for ICC's in multilevel models. \index{ICC} I adapt
the notation of this formula originally provided in the excellent Stata
documentation, for the intraclass correlation for models with many
levels. For a model with \emph{Q} total levels, the ICC for level
\emph{q} is defined as:

\begin{equation}\phantomsection\label{eq-ICC-morelevels}{\text{ICC}_q = \frac{\sum^Q_{q} \tau_q}{\text{var}(e) + \Sigma^Q_2 \tau_q}}\end{equation}

As a substantive example, the ICC for observations in the same level 4
unit, but different level 3 and level 2 units is:

\begin{equation}\phantomsection\label{eq-ICC-morelevels-substantive}{\text{ICC} = \frac{4.173}{4.173 + 2.849 + 11.724 + 28.234} \approx 0.089}\end{equation}

These results suggest that approximately 8.9\% of the variation in the
outcome is accounted for by the clustering of units in the same level 4
unit, but different level 3 and different level 2 units.

\subsection{Conditional Model}\label{conditional-model}

\begin{longtable}[]{@{}lll@{}}
\toprule\noalign{}
& 1 & \\
\midrule\noalign{}
\endhead
\bottomrule\noalign{}
\endlastfoot
t & 0.943 & ** \\
warmth & 0.914 & ** \\
physical\_punishment & -1.009 & ** \\
identity & & \\
1 & -0.133 & \\
intervention & & \\
1 & 0.859 & ** \\
HDI & 0.015 & \\
\_cons & 50.164 & ** \\
var(\_cons) & 4.722 & \\
var(\_cons) & 2.863 & \\
var(\_cons) & 8.421 & \\
var(e) & 26.029 & \\
Number of observations & 9000 & \\
\end{longtable}

** p\textless.01, * p\textless.05

\subsection{Interpretation}\label{interpretation}

Similar to the results in other chapters, the results suggest that time
is associated with an increase in the outcome. Parental warmth is
associated with an increase in the outcome, while physical punishment is
associated with a decrease in the outcome. Participation in the
intervention is associated with higher levels of the outcome, while the
identity group and HDI are not associated with the outcome.

\section{Cross-Classified Models}\label{cross-classified-models}

\subsection{Data}\label{data-1}

Ethnologue (2024) estimates that there are over 7,000 languages spoken
worldwide. For the sake of illustration, I add 100 hypothetical
languages to the \emph{cross-sectional} data.

\begin{tcolorbox}[enhanced jigsaw, breakable, colbacktitle=quarto-callout-tip-color!10!white, leftrule=.75mm, bottomrule=.15mm, opacityback=0, colframe=quarto-callout-tip-color-frame, coltitle=black, rightrule=.15mm, title=\textcolor{quarto-callout-tip-color}{\faLightbulb}\hspace{0.5em}{Tip}, colback=white, opacitybacktitle=0.6, toprule=.15mm, titlerule=0mm, bottomtitle=1mm, left=2mm, toptitle=1mm, arc=.35mm]

In contrast to my addition of simulated UN Regions to the
\emph{longitudinal} data earlier in Section~\ref{sec-fourlevel}, these
100 hypothetical languages are a large enough number of languages to
include as a random effect
(Section~\ref{sec-indicator-variables-random-effects}).

\end{tcolorbox}

\subsection{Equation}\label{equation-1}

The terms in the equation below are similar to those in
Equation~\ref{eq-MLM-fourlevel}. Note, however, that here, for purposes
of exposition, I am working with the cross sectional data.

\texttt{country}, and the new variable, \texttt{language} are
\emph{cross-classified} levels in the data in that neither is
hierarchically nested inside the other: residents of the same country
will not all speak the same language, and speakers of the same language
will not all live in the same country. I would use a random effect \(l\)
for language, except for the fact that it would create confusion because
of the similarity of \(l\) and the number \emph{1}. Therefore, I use the
next available letter of the alphabet for language, \(m\).
\index{cross classified models}

\begin{equation}\phantomsection\label{eq-crossclassified}{\text{outcome}_{itj} = \beta_0 + \beta_1 \text{parental warmth}_{itj} + \beta_2 \text{physical punishment}_{itj} + \beta_3 \text{time}_{itj} \ + }\end{equation}

\[\beta_4 \text{identity}_{itj} + \beta_5 \text{intervention}_{itj} + \beta_6 \text{HDI}_{itj} +\]

\[u_{0j} + m_{0m} + e_{ijm}\]

Again, terms are similar to the terms used in
Equation~\ref{eq-MLM-longitudinal} and Equation~\ref{eq-MLM-fourlevel},
with the exception that in Equation~\ref{eq-crossclassified}, we are
talking about family \emph{i} in country \emph{j}, speaking language
\emph{m}.

\subsection{Unconditional Model}\label{unconditional-model-2}

\begin{longtable}[]{@{}lll@{}}
\toprule\noalign{}
& 1 & \\
\midrule\noalign{}
\endhead
\bottomrule\noalign{}
\endlastfoot
\_cons & 52.435 & ** \\
var(R\_country) & 3.122 & \\
var(R\_language) & 1.426 & \\
var(e) & 39.438 & \\
Number of observations & 3000 & \\
\end{longtable}

** p\textless.01, * p\textless.05

\subsection{ICC}\label{icc-1}

Definitions for ICC's for cross classified models are provided in
Rabe-Hesketh \& Skrondal (2022). Following my procedure above, I adapt
this notation by using \(\tau\) for variance components. Because these
variance components do not refer to hierarchically nested levels,
instead of numbers, I refer to variance components \(\tau_A\) and
\(\tau_B\). \index{ICC}

\begin{longtable}[]{@{}
  >{\centering\arraybackslash}p{(\columnwidth - 4\tabcolsep) * \real{0.1667}}
  >{\centering\arraybackslash}p{(\columnwidth - 4\tabcolsep) * \real{0.1944}}
  >{\centering\arraybackslash}p{(\columnwidth - 4\tabcolsep) * \real{0.3472}}@{}}

\caption{\label{tbl-ICC-crossclassified}ICCs in Cross Classified Models}

\tabularnewline

\toprule\noalign{}
\begin{minipage}[b]{\linewidth}\centering
Groupings in Model
\end{minipage} & \begin{minipage}[b]{\linewidth}\centering
Description
\end{minipage} & \begin{minipage}[b]{\linewidth}\centering
Formula
\end{minipage} \\
\midrule\noalign{}
\endhead
\bottomrule\noalign{}
\endlastfoot
A & Same unit A & \(\frac{\tau_A}{\tau_A
+
\tau_B +
e_{ijm}}\) \\
B & Same unit B & \(\frac{\tau_B}{\tau_A
+
\tau_B +
e_{ijm}}\) \\

\end{longtable}

As a substantive example, the ICC for language is calculated as:

\begin{equation}\phantomsection\label{eq-ICC-crossclassified-substantive}{\text{ICC} = \frac{1.426}{3.122 + 1.426 + 39.438} \approx .032}\end{equation}

Results of Equation~\ref{eq-ICC-crossclassified-substantive} suggest
that approximately 3.2\% of the variation in the outcome is attributable
to language spoken.

\subsection{Conditional Model}\label{conditional-model-1}

\begin{longtable}[]{@{}lll@{}}
\toprule\noalign{}
& 1 & \\
\midrule\noalign{}
\endhead
\bottomrule\noalign{}
\endlastfoot
warmth & 0.822 & ** \\
physical\_punishment & -1.000 & ** \\
identity & & \\
1 & -0.288 & \\
intervention & & \\
1 & 0.611 & ** \\
HDI & 0.001 & \\
\_cons & 51.824 & ** \\
var(R\_country) & 3.332 & \\
var(R\_language) & 1.411 & \\
var(e) & 35.093 & \\
Number of observations & 3000 & \\
\end{longtable}

** p\textless.01, * p\textless.05

\subsection{Interpretation}\label{interpretation-1}

Again, similar to the results above, the results indicate that parental
warmth is associated with an increase in the outcome, while physical
punishment is associated with a decrease in the outcome. Participation
in the intervention is associated with higher levels of the outcome,
while the identity group and HDI are not associated with the outcome.

\bookmarksetup{startatroot}

\chapter{Conclusion}\label{conclusion}

\begin{quote}
``To take on a new perspective obviously does not mean throwing out all
of our knowledge; what it supposes, rather, is that we will relativize
that knowledge and critically revise it from the perspective of the
popular majorities. Only then will the theories and models show their
validity or deficiency, their utility or lack thereof, the universality
or provincialism. Only then will the techniques we have learned display
their liberating potential or their seeds of subjugation.''
(Martin-Baro, 1994b) \index{universalism} \index{Ignacio Martín-Baró}
\end{quote}

Many data sets relevant to the study of important social issues, or
social problems, are inherently multilevel. For example, data on diverse
children in schools, diverse individuals in neighborhoods, and
individuals or families in diverse and different countries all have
multilevel structures in which individuals are clustered in higher level
social structures. Data with repeated measures, sometimes termed panel
data, can also be thought of as multilevel data sets, wherein individual
timepoints are nested inside individuals, who may in turn be nested or
clustered in larger social units such as countries.

Failure to use appropriate basic multilevel models with such multilevel
data can lead to answers that are either biased, or demonstrably wrong.
Simple multilevel models allow the researcher to correctly estimate
statistical significance, and to correctly estimate regression
coefficients while accounting for multilevel structure.
\index{correct answers} More advanced applications of multilevel models
allow the researcher to explore the variation in both predictors and
outcomes--and the relationship of predictors to outcomes--and to
characterize the extent of this variation. Lastly, multilevel models
provide a foundation for thinking about closely related models--fixed
effects regression, and correlated random effects models--that provide
methods for estimation that afford stronger causal conclusions.

Thus, for applied researchers, interested in addressing a variety of
social problems and social issues with diverse samples of individuals,
multilevel models present a method to think clearly about variation, to
explore that variation, and to extend that thinking about variation to
estimate more causally robust models within the context of diversity and
variation. \index{variation and diversity}

\bookmarksetup{startatroot}

\chapter*{References}\label{references}
\addcontentsline{toc}{chapter}{References}

\markboth{References}{References}

\phantomsection\label{refs}
\begin{CSLReferences}{1}{0}
\bibitem[\citeproctext]{ref-Abelson1995}
Abelson, R. P. (1995). Statistics as principled argument. In
\emph{Statistics as principled argument.} (pp. 221, xv, 221--xv).
Lawrence Erlbaum Associates, Inc.

\bibitem[\citeproctext]{ref-Agresti1997}
Agresti, A., \& Finlay, B. (1997). \emph{Statistical methods for the
social sciences} (3rd ed.). Prentice Hall.

\bibitem[\citeproctext]{ref-Alhazen_2023}
Alhazen. (2023). \emph{The optics of {I}bn al-{H}aytham books {IV-V}: On
reflection and images seen by reflection} (J. P. Hogendijk, Ed.). The
Warburg Institute.

\bibitem[\citeproctext]{ref-Allison2009}
Allison, P. (2009). \emph{Fixed effects regression models}. Sage
Publishing.

\bibitem[\citeproctext]{ref-Antonakis2019}
Antonakis, J., Bastardoz, N., \& Ronko, M. (2021). On ignoring the
random effects assumption in multilevel models: Review, critique, and
recommendations. \emph{Organizational Research Methods}, \emph{24},
443--483. \url{https://doi.org/10.1177/1094428119877457}

\bibitem[\citeproctext]{ref-Antweiler2016}
Antweiler, C. (2016). \emph{Our common denominator: Human universals
revisited}. Berghahn.

\bibitem[\citeproctext]{ref-ArelBundock2018}
Arel-Bundock, V., Enevoldsen, N., \& Yetman, C. (2018). Countrycode: An
r package to convert country names and country codes. \emph{Journal of
Open Source Software}, \emph{3}(28), 848.
\url{https://doi.org/10.21105/joss.00848}

\bibitem[\citeproctext]{ref-Aron1994}
Aron, A., \& Corne, S. (1994). Introduction. In A. Aron \& S. Corne
(Eds.), \emph{Writings for a liberation psychology}. Harvard University
Press.

\bibitem[\citeproctext]{ref-Barr2013}
Barr, D. J., Levy, R., Scheepers, C., \& Tily, H. J. (2013). Random
effects structure for confirmatory hypothesis testing: Keep it maximal.
\emph{Journal of Memory and Language}, \emph{68}(3), 255--278.
\url{https://doi.org/10.1016/j.jml.2012.11.001}

\bibitem[\citeproctext]{ref-Barth2020}
Barth, R. P., \& Olsen, A. N. (2020). Are children oppressed? The timely
importance of answering this question. \emph{Children and Youth Services
Review}, \emph{110}, 104780.
\url{https://doi.org/10.1016/j.childyouth.2020.104780}

\bibitem[\citeproctext]{ref-Bland1994}
Bland, J. M., \& Altman, D. G. (1994). Statistics notes: Correlation,
regression, and repeated data. \emph{BMJ}, \emph{308}, 896.
\url{https://doi.org/10.1136/bmj.308.6933.896}

\bibitem[\citeproctext]{ref-Blasi2022}
Blasi, D. E., Henrich, J., Adamou, E., Kemmerer, D., \& Majid, A.
(2022). Over-reliance on {E}nglish hinders cognitive science.
\emph{Trends in Cognitive Sciences}.
\url{https://doi.org/10.1016/j.tics.2022.09.015}

\bibitem[\citeproctext]{ref-Burkner2018}
Burkner, P.-C. (2018). Advanced {B}ayesian multilevel modeling with the
{R} package brms. \emph{The R Journal}, \emph{10}(1), 395--411.

\bibitem[\citeproctext]{ref-Burton2005}
Burton, M., \& Kagan, C. (2005). {Liberation Social Psychology: Learning
From Latin America Psychology of liberation: Learning from Latin
America}. \emph{Journal of Community \& Applied Social Psychology},
\emph{15}. \url{https://doi.org/10.1002/casp.786}

\bibitem[\citeproctext]{ref-Button2013}
Button, K. S., Ioannidis, J. P. A., Mokrysz, C., Nosek, B. A., Flint,
J., Robinson, E. S. J., \& Munaf`o, M. R. (2013). Power failure: Why
small sample size undermines the reliability of neuroscience.
\emph{Nature Reviews Neuroscience}, \emph{14}, 365--376.
\url{https://doi.org/10.1038/nrn3475}

\bibitem[\citeproctext]{ref-Caspi2000}
Caspi, A., Taylor, A., Moffitt, T. E., \& Plomin, R. (2000).
Neighborhood deprivation affects children's mental health: Environmental
risks identified in a genetic design. \emph{Psychological Science},
\emph{11}(4), 338--342. \url{https://doi.org/10.1111/1467-9280.00267}

\bibitem[\citeproctext]{ref-Cesaire1956}
Cesaire, A. (1956). \emph{Letter to {M}aurice {T}horez}.

\bibitem[\citeproctext]{ref-Cokley2013}
Cokley, K., \& Awad, G. (2013). In defense of quantitative methods:
Using the {``master's tools''} to promote social justice. \emph{Journal
for Social Action in Counseling \& Psychology}, \emph{5}, 26.
\url{https://doi.org/10.33043/JSACP.5.2.26-41}

\bibitem[\citeproctext]{ref-Dahabreh2024}
Dahabreh, I. J., \& Bibbins-Domingo, K. (2024). Causal inference about
the effects of interventions from observational studies in medical
journals. \emph{JAMA}. \url{https://doi.org/10.1001/jama.2024.7741}

\bibitem[\citeproctext]{ref-Dalton2000}
Dalton, R. (2000). Como t{ú}. In M. Espada (Ed.), \emph{Poetry like
bread: Poets of the political imagination}. Curbstone Press.

\bibitem[\citeproctext]{ref-Deater-Deckard1996}
Deater-Deckard, K., Dodge, K. A., Bates, J. E., \& Pettit, G. S. (1996).
{Physical discipline among African American and European American
mothers: Links to children's externalizing behaviors.}
\emph{Developmental Psychology}, \emph{32}(6), 1065--1072.
\url{https://doi.org/10.1037/0012-1649.32.6.1065}

\bibitem[\citeproctext]{ref-Diener2022}
Diener, E., Northcott, R., Zyphur, M. J., \& West, S. G. (2022). Beyond
experiments. \emph{Perspectives on Psychological Science}, \emph{17},
1101--1119. \url{https://doi.org/10.1177/17456916211037670}

\bibitem[\citeproctext]{ref-DiezRoux2002}
Diez Roux, A. V. (2002). {A glossary for multilevel analysis}.
\emph{Journal of Epidemiology and Community Health}, \emph{56}(8), 588
LP--594. \url{https://doi.org/10.1136/jech.56.8.588}

\bibitem[\citeproctext]{ref-Dove1999}
Dove, R. (1999). The first book. In \emph{On the bus with {R}osa
{P}arks}. W.W. Norton.

\bibitem[\citeproctext]{ref-Draper2022}
Draper, C. E., Barnett, L. M., Cook, C. J., Cuartas, J. A., Howard, S.
J., McCoy, D. C., Merkley, R., Molano, A., no, C. M.-C., Obradovic, J.,
Scerif, G., Valentini, N. C., Venetsanou, F., \& Yousafzai, A. K.
(2022). Publishing child development research from around the world: An
unfair playing field resulting in most of the world's child population
under-represented in research. \emph{Infant and Child Development},
\emph{n/a}, e2375. \url{https://doi.org/10.1002/icd.2375}

\bibitem[\citeproctext]{ref-Duncan2006}
Duncan, G. J., \& Gibson-Davis, C. M. (2006). {Connecting Child Care
Quality to Child Outcomes: Drawing Policy Lessons from Nonexperimental
Data}. \emph{Evaluation Review}, \emph{30}(5), 611--630.
\url{https://doi.org/10.1177/0193841X06291530}

\bibitem[\citeproctext]{ref-Eamon2001}
Eamon, M. K. (2001). {Poverty, Parenting, Peer, and Neighborhood
Influences on Young Adolescent Antisocial Behavior}. \emph{Journal of
Social Service Research}, \emph{28}(1), 1--23.
\url{https://doi.org/10.1300/J079v28n01_01}

\bibitem[\citeproctext]{ref-Ellacuria2013}
Ellacuria, I. (2013). \emph{{I}gnacio {E}llacuria: Essays on history,
liberation, and salvation} (M. L. Lee, Ed.). Orbis Books.

\bibitem[\citeproctext]{ref-Ethnologue2024}
Ethnologue. (2024). \emph{Languages of the world}.
\url{https://www.ethnologue.com/}

\bibitem[\citeproctext]{ref-Felitti1998}
Felitti, V. J., Anda, R. F., Nordenberg, D., Williamson, D. F., Spitz,
A. M., Edwards, V., Koss, M. P., \& Marks, J. S. (1998). Relationship of
childhood abuse and household dysfunction to many of the leading causes
of death in adults: The adverse childhood experiences (ACE) study.
\emph{American Journal of Preventive Medicine}, \emph{14}(4), 245--258.
\url{https://doi.org/10.1016/S0749-3797(98)00017-8}

\bibitem[\citeproctext]{ref-FIREBAUGH20014023}
Firebaugh, G. (2001). Ecological fallacy, statistics of. In N. J.
Smelser \& P. B. Baltes (Eds.), \emph{International encyclopedia of the
social \& behavioral sciences} (pp. 4023--4026). Pergamon.
\url{https://doi.org/10.1016/B0-08-043076-7/00765-8}

\bibitem[\citeproctext]{ref-Frank2018}
Frank, M. (2018). \emph{Mixed effects models: Is it time to go
{B}ayesian by default?}
\url{http://babieslearninglanguage.blogspot.com/2018/02/mixed-effects-models-is-it-time-to-go.html}

\bibitem[\citeproctext]{ref-Freedman1991}
Freedman, D., Pisani, R., Purves, R., \& Adhikari, A. (1991).
\emph{Statistics} (2nd ed.). W.W. Norton \& Co.

\bibitem[\citeproctext]{ref-Gelman2007}
Gelman, A., Shor, B., Bafumi, J., \& Park, D. (2007). Rich state, poor
state, red state, blue state: What's the matter with {C}onnecticut?
\emph{Quarterly Journal of Political Science}, \emph{2}, 345--367.
\url{https://doi.org/10.2139/ssrn.1010426}

\bibitem[\citeproctext]{ref-Gershoff2016B}
Gershoff, E. T., \& Grogan-Kaylor, A. (2016a). {Race as a Moderator of
Associations Between Spanking and Child Outcomes}. \emph{Family
Relations}, \emph{65}(3), 490--501.
\url{https://doi.org/10.1111/fare.12205}

\bibitem[\citeproctext]{ref-Gershoff2016}
Gershoff, E. T., \& Grogan-Kaylor, A. (2016b). Spanking and child
outcomes: Old controversies and new meta-analyses. \emph{Journal of
Family Psychology}, \emph{30}, 453--469.
\url{https://doi.org/10.1037/fam0000191}

\bibitem[\citeproctext]{ref-Gershoff2010}
Gershoff, E. T., Grogan-Kaylor, A., Lansford, J. E., Chang, L., Zelli,
A., Deater-Deckard, K., \& Dodge, K. A. (2010). Parent discipline
practices in an international sample: Associations with child behaviors
and moderation by perceived normativeness. \emph{Child Development},
\emph{81}, 487--502.
\url{https://doi.org/10.1111/j.1467-8624.2009.01409.x}

\bibitem[\citeproctext]{ref-Goldacre2011}
Goldacre, B. (2011). {Foreward}. In \emph{Testing treatments} (2nd ed.).

\bibitem[\citeproctext]{ref-Gottlieb2002}
Gottlieb, A. (2002). New developments in the anthropology of childcare.
\emph{Anthropology News}, \emph{43}, 13.
\url{https://doi.org/10.1111/an.2002.43.7.13}

\bibitem[\citeproctext]{ref-Grogan-Kaylor2021}
Grogan-Kaylor, A., Castillo, B., Pace, G. T., Ward, K. P., Ma, J., Lee,
S. J., \& Knauer, H. (2021). {Global perspectives on physical and
nonphysical discipline: A {B}ayesian multilevel analysis}.
\emph{International Journal of Behavioral Development}.
\url{https://doi.org/10.1177/0165025420981642}

\bibitem[\citeproctext]{ref-GroganKaylor2018}
Grogan-Kaylor, A., Ma, J., Lee, S. J., Castillo, B., Ward, K. P., \&
Klein, S. (2018). Using {B}ayesian analysis to examine associations
between spanking and child externalizing behavior across race and ethnic
groups. \emph{Child Abuse and Neglect}, \emph{86}, 257--266.
\url{https://doi.org/10.1016/j.chiabu.2018.10.009}

\bibitem[\citeproctext]{ref-Hand2014}
Hand, D. J. (2014). {Wonderful Examples, but Let's not Close Our Eyes}.
\emph{Statistical Science}, \emph{29}(1), 98--100.
\url{https://doi.org/10.1214/13-STS446}

\bibitem[\citeproctext]{ref-Heilmann2021}
Heilmann, A., Mehay, A., Watt, R. G., Kelly, Y., Durrant, J. E., van
Turnhout, J., \& Gershoff, E. T. (2021). Physical punishment and child
outcomes: A narrative review of prospective studies. \emph{The Lancet}.
\url{https://doi.org/10.1016/S0140-6736(21)00582-1}

\bibitem[\citeproctext]{ref-Henrich2010}
Henrich, J., Heine, S. J., \& Norenzayan, A. (2010). {The weirdest
people in the world?} \emph{Behavioral and Brain Sciences}.
\url{https://doi.org/10.1017/S0140525X0999152X}

\bibitem[\citeproctext]{ref-Hines2022}
Hines, C. T., Kalil, A., \& Ryan, R. M. (2022). {Differences in Parents'
Attitudes Toward Spanking Across Socioeconomic Status and Region,
1986--2016}. \emph{Social Indicators Research}, \emph{160}(1), 133--158.
\url{https://doi.org/10.1007/s11205-021-02803-7}

\bibitem[\citeproctext]{ref-Holland1986}
Holland, P. W. (1986). {Statistics and Causal Inference}. \emph{Journal
of the American Statistical Association}, \emph{81}(396), 945--960.
\url{https://doi.org/10.1080/01621459.1986.10478354}

\bibitem[\citeproctext]{ref-Hooper2022}
Hooper, R. (2022). {Designing Stepped Wedge Trials with Continuous
Recruitment}. \emph{Methods: Mind the Gap (Webinar Series)}.

\bibitem[\citeproctext]{ref-Hox2018}
Hox, J. J., Moerbeek, M., \& van de Schoot, R. (2018). \emph{Multilevel
analysis: Techniques and applications} (Third edition.). Routledge,
Taylor \& Francis Group.

\bibitem[\citeproctext]{ref-Hurston1942}
Hurston, Z. N. (1942). \emph{Dust tracks on a road}. HarperPerennial.

\bibitem[\citeproctext]{ref-Khaleque2002}
Khaleque, A., \& Rohner, R. P. (2002). Perceived parental
acceptance-rejection and psychological adjustment: A meta-analysis of
cross-cultural and intracultural studies. \emph{Journal of Marriage and
Family}, \emph{64}, 54--64.
\url{https://doi.org/10.1111/j.1741-3737.2002.00054.x}

\bibitem[\citeproctext]{ref-Kottak2021}
Kottak, C. (2021). \emph{Anthropology: Appreciating human diversity}
(19th ed.). McGraw Hill.

\bibitem[\citeproctext]{ref-Kreft1998}
Kreft, I., \& de Leeuw, J. (1998). \emph{Introducing multilevel
modeling}. SAGE Publications.
\url{https://doi.org/10.4135/9781849209366}

\bibitem[\citeproctext]{ref-MR0554183}
Launer, R. L., \& Wilkinson, G. N. (1979). Robustness in statistics. In
R. L. Launer \& G. N. Wilkinson (Eds.), \emph{Proceedings of a
{W}orkshop held at the {A}rmy {R}esearch {O}ffice, {R}esearch {T}riangle
{P}ark, {N}.{C}., {A}pril 11--12, 1978} (p. xvi+296). Academic Press,
Inc. {[}Harcourt Brace Jovanovich, Publishers{]}, New York-London.

\bibitem[\citeproctext]{ref-Lee2022}
Lee, S. J., Ward, K. P., Grogan-Kaylor, A., \& Singh, V. (2022). Anxiety
and depression during {COVID-19}: Are adults in households with children
faring worse? \emph{Journal of General Internal Medicine}, \emph{37},
1328--1330. \url{https://doi.org/10.1007/s11606-021-07256-9}

\bibitem[\citeproctext]{ref-Livio2002}
Livio, M. (2002). The golden ratio: The story of phi, the world's most
astonishing number. In \emph{The golden ratio: the story of phi, the
world's most astonishing number} (1st ed.). Broadway Books.

\bibitem[\citeproctext]{ref-Long2014}
Long, J. S., \& Freese, J. (2014). \emph{Regression models for
categorical dependent variables using stata} (3rd ed.). Stata Press.

\bibitem[\citeproctext]{ref-Lovelace1992}
Lovelace, A. K. (1992). Ada: The enchantress of numbers: A selection
from the letters of {L}ord {B}yron's daughter and her description of the
first computer. In \emph{Ada: the enchantress of numbers: a selection
from the letters of {L}ord {B}yron's daughter and her description of the
first computer}. Strawberry Press.

\bibitem[\citeproctext]{ref-Ludecke2023}
Lüdecke, D. (2023). \emph{sjPlot: Data visualization for statistics in
social science}. \url{https://CRAN.R-project.org/package=sjPlot}

\bibitem[\citeproctext]{ref-Luke2004}
Luke, D. (2004). \emph{Multilevel modeling}. SAGE Publications, Inc.
\url{https://doi.org/10.4135/9781412985147}

\bibitem[\citeproctext]{ref-Ma2022}
Ma, J., Grogan-Kaylor, A. C., Pace, G. T., Ward, K. P., \& Lee, S. J.
(2022). {The association between spanking and physical abuse of young
children in 56 low- and middle-income countries}. \emph{Child Abuse \&
Neglect}, \emph{129}, 105662.
\url{https://doi.org/10.1016/j.chiabu.2022.105662}

\bibitem[\citeproctext]{ref-Ma2018}
Ma, J., Grogan-Kaylor, A., \& Lee, S. J. (2018). Associations of
neighborhood disorganization and maternal spanking with children's
aggression: A fixed-effects regression analysis. \emph{Child Abuse \&
Neglect}, \emph{76}, 106--116.
\url{https://doi.org/10.1016/j.chiabu.2017.10.013}

\bibitem[\citeproctext]{ref-Maas2004}
Maas, C. J. M., \& Hox, J. J. (2004). Robustness issues in multilevel
regression analysis. \emph{Statistica Neerlandica}, \emph{58}(2),
127--137.
https://doi.org/\url{https://doi.org/10.1046/j.0039-0402.2003.00252.x}

\bibitem[\citeproctext]{ref-Maas2005}
Maas, C., \& Hox, J. (2005). Sufficient sample sizes for multilevel
modeling. \emph{Methodology: European Journal of Research Methods for
the Behavioral and Social Sciences}, \emph{1}, 86--92.
\url{https://doi.org/10.1027/1614-2241.1.3.86}

\bibitem[\citeproctext]{ref-Mangharam2017}
Mangharam, M. L. (2017). \emph{Literatures of liberation: Non-{E}uropean
universalisms and democratic progress}. Ohio State University Press.
\url{https://doi.org/10.2307/j.ctv16gff5b}

\bibitem[\citeproctext]{ref-Martin-Baro1994}
Martin-Baro, I. (1994a). Public opinion research. In A. Aron \& S. Corne
(Eds.), \emph{Writings for a liberation psychology}. Harvard University
Press.

\bibitem[\citeproctext]{ref-Martin-Baro1994B}
Martin-Baro, I. (1994b). Toward a liberation psychology. In A. Aron \&
S. Corne (Eds.), \emph{Writings for a liberation psychology}. Harvard
University Press.

\bibitem[\citeproctext]{ref-Martin-Baro1998}
Martin-Baro, I. (1998). {Retos y perspectivas de la psicología
latinoamericana}. In A. Blanco (Ed.), \emph{Psicología de la
liberación}. Trotta.

\bibitem[\citeproctext]{ref-Matuschek2017}
Matuschek, H., Kliegl, R., Vasishth, S., Baayen, H., \& Bates, D.
(2017). Balancing type {I} error and power in linear mixed models.
\emph{Journal of Memory and Language}.
\url{https://doi.org/10.1016/j.jml.2017.01.001}

\bibitem[\citeproctext]{ref-Mbiti1970}
Mbiti, J. S. (1970). \emph{African religions and philosophy}. Anchor
Books.

\bibitem[\citeproctext]{ref-Montero2009}
Montero, M., \& Sonn, C. C. (2009). Psychology of liberation: Theory and
applications. In M. Montero \& C. C. Sonn (Eds.), \emph{Psychology of
Liberation: Theory and Applications} (1st ed. 2009.). Springer.

\bibitem[\citeproctext]{ref-Mandelbrot}
Moore, B., \& dos Reis, M. (2017). \emph{Mandelbrot: Generates views on
the mandelbrot set}. \url{https://CRAN.R-project.org/package=mandelbrot}

\bibitem[\citeproctext]{ref-Morton1972}
Morton, N. (1972). The rising of women's consciousness in a male
language structure. \emph{Andover Newton Quarterly}, \emph{12},
177--190.

\bibitem[\citeproctext]{ref-Mundlak1978}
Mundlak, Y. (1978). On the pooling of time series and cross section
data. \emph{Econometrica}, \emph{46}, 69--85.
\url{https://doi.org/10.2307/1913646}

\bibitem[\citeproctext]{ref-nalborczyk_batailler_loevenbruck_vilain_burkner_2019}
Nalborczyk, L., Batailler, C., Loevenbruck, H., Vilain, A., \& Burkner,
P.-C. (2019, May). \emph{An introduction to {B}ayesian multilevel models
using brms: A case study of gender effects on vowel variability in
standard {I}ndonesian}. PsyArXiv.
\url{https://doi.org/10.1044/2018_JSLHR-S-18-0006}

\bibitem[\citeproctext]{ref-Nieuwenhuis2015}
Nieuwenhuis, R. (2015). Association, aggregation, and paradoxes: On the
positive correlation between fertility and women's employment.
\emph{Demographic Research}, \emph{32}.
\url{https://www.demographic-research.org/volumes/vol32/23/}

\bibitem[\citeproctext]{ref-Oberauer2022}
Oberauer, K. (2022). {The Importance of Random Slopes in Mixed Models
for Bayesian Hypothesis Testing}. \emph{Psychological Science}.
\url{https://doi.org/doi:10.1177/09567976211046884}

\bibitem[\citeproctext]{ref-Oliver2015}
Oliver, M., \& Tippett, K. (2015). \emph{{M}ary {O}liver: {``{I} got
saved by the beauty of the world.''}} The On Being Project.
\url{https://onbeing.org/programs/mary-oliver-i-got-saved-by-the-beauty-of-the-world/}

\bibitem[\citeproctext]{ref-Pace2019}
Pace, G. T., Lee, S. J., \& Grogan-Kaylor, A. (2019). {Spanking and
young children's socioemotional development in low- and middle-income
countries}. \emph{Child Abuse and Neglect}, \emph{88}, 84--95.
\url{https://doi.org/10.1016/j.chiabu.2018.11.003}

\bibitem[\citeproctext]{ref-Poincare1908}
Poincare, H. (1908). \emph{Science et methode}. Flammarion.

\bibitem[\citeproctext]{ref-Proctor2012}
Proctor, R. N. (2012). {The history of the discovery of the
cigarette--lung cancer link: evidentiary traditions, corporate denial,
global toll}. \emph{Tobacco Control}, \emph{21}(2), 87 LP--91.
\url{https://doi.org/10.1136/tobaccocontrol-2011-050338}

\bibitem[\citeproctext]{ref-Pye2011}
Pye, E. (2011). {E}veline {P}ye: Poetry in numbers, by {J}ulian
{C}hampkin. \emph{Significance}, \emph{8}(3), 127--130.
\url{https://doi.org/10.1111/j.1740-9713.2011.00510.x}

\bibitem[\citeproctext]{ref-RabeHesketh2012}
Rabe-Hesketh, S., \& Skrondal, A. (2012). \emph{Multilevel and
longitudinal modeling using {S}tata - {V}olume {I}: Continuous
responses} (p. 974). Stata Press.

\bibitem[\citeproctext]{ref-RabeHesketh2022}
Rabe-Hesketh, S., \& Skrondal, A. (2022). \emph{Multilevel and
longitudinal modeling using {S}tata} (4th ed.). Stata Press.

\bibitem[\citeproctext]{ref-Raudenbush2002}
Raudenbush, S. W., \& Bryk, A. S. (2002). \emph{Hierarchical linear
models: Applications and data analysis methods}. Sage Publications.

\bibitem[\citeproctext]{ref-Rich1984}
Rich, A. (1984). Transcendental etude. In \emph{The fact of a doorframe:
Poems selected and new 1950-1984}. Norton.

\bibitem[\citeproctext]{ref-Rocha2020}
Rocha, Z. L., \& Aspinall, P. J. (2020). The palgrave international
handbook of mixed racial and ethnic classification. In Z. L. Rocha \& P.
J. Aspinall (Eds.), \emph{The Palgrave international handbook of mixed
racial and ethnic classification}. Palgrave Macmillan,.

\bibitem[\citeproctext]{ref-Rothenberg2022}
Rothenberg, W. A., Ali, S., Rohner, R. P., Lansford, J. E., Britner, P.
A., Giunta, L. D., Dodge, K. A., Malone, P. S., Oburu, P., Pastorelli,
C., Skinner, A. T., Sorbring, E., Steinberg, L., Tapanya, S., Tirado, L.
M. U., Yotanyamaneewong, S., Alampay, L. P. na, Al-Hassan, S. M.,
Bacchini, D., \ldots{} Deater-Deckard, K. (2022). Effects of parental
acceptance-rejection on children's internalizing and externalizing
behaviors: A longitudinal, multicultural study. \emph{Journal of Child
and Family Studies}, \emph{31}, 29--47.
\url{https://doi.org/10.1007/s10826-021-02072-5}

\bibitem[\citeproctext]{ref-Rubin1997}
Rubin, V. C. (1997). Bright galaxies, dark matters. In \emph{Bright
galaxies, dark matters}. American Institute of Physics.

\bibitem[\citeproctext]{ref-Sagan1995}
Sagan, C. (1995). \emph{The demon haunted world: Science as a candle in
the dark}. Ballantine Books.

\bibitem[\citeproctext]{ref-Scharrer2021}
Scharrer, E., \& Ramasubramanian, S. (2021). \emph{Quantitative research
methods in communication: The power of numbers for social justice}.
\url{https://doi.org/10.4324/9781003091653}

\bibitem[\citeproctext]{ref-Schielzeth2009}
Schielzeth, H., \& Forstmeier, W. (2009). Conclusions beyond support:
Overconfident estimates in mixed models. \emph{Behavioral Ecology},
\emph{20}, 416--420. \url{https://doi.org/10.1093/beheco/arn145}

\bibitem[\citeproctext]{ref-Schunck2013}
Schunck, R. (2013). Within and between estimates in random-effects
models: Advantages and drawbacks of correlated random effects and hybrid
models. \emph{The Stata Journal}, \emph{13}, 65--76.
\url{https://doi.org/10.1177/1536867X1301300105}

\bibitem[\citeproctext]{ref-Senn2005}
Senn, S. (2005). Dichotomania: An obsessive compulsive disorder that is
badly affecting the quality of analysis of pharmaceutical trials.
\emph{International Statistical Institute 55th Session}, 13.

\bibitem[\citeproctext]{ref-ShraderFrechette2014}
Shrader-Frechette, K. (2014). \emph{Tainted : How philosophy of science
can expose bad science}. Oxford University Press.

\bibitem[\citeproctext]{ref-Silverman1998}
Silverman, W. (1998). \emph{Where's the evidence}. Oxford University
Press.

\bibitem[\citeproctext]{ref-Simpson1951}
Simpson, E. H. (1951). The interpretation of interaction in contingency
tables. \emph{Journal of the Royal Statistical Society. Series B
(Methodological)}, \emph{13}, 238--241.
\url{http://www.jstor.org/stable/2984065}

\bibitem[\citeproctext]{ref-Singer2003}
Singer, J. D., \& Willett, J. B. (2003). Applied longitudinal data
analysis : Modeling change and event occurrence. In \emph{Applied
longitudinal data analysis : modeling change and event occurrence}.
Oxford University Press.

\bibitem[\citeproctext]{ref-Snijders2012}
Snijders, T. A. B., \& Bosker, R. J. (2012). Multilevel analysis: An
introduction to basic and advanced multilevel modeling. In
\emph{Multilevel analysis: an introduction to basic and advanced
multilevel modeling} (2nd ed.). Sage.

\bibitem[\citeproctext]{ref-Sontag2007}
Sontag, S. (2007). At the same time: The novelist and moral reasoning.
In P. Dilonardo \& A. Jump (Eds.), \emph{At the same time: The novelist
and moral reasoning}. Picador.

\bibitem[\citeproctext]{ref-Stage2014}
Stage, F. K., \& Wells, R. S. (2014). Critical quantitative inquiry in
context. \emph{New Directions for Institutional Research}, \emph{2013},
1--7. \url{https://doi.org/10.1002/ir.20041}

\bibitem[\citeproctext]{ref-StataCorp2021:1}
StataCorp. (2023a). \emph{Stata 18 longitudinal data/panel data
reference manual}. Stata Press.

\bibitem[\citeproctext]{ref-StataCorp2023}
StataCorp. (2023b). \emph{Stata 18 multilevel mixed effects reference
manual}. Stata Press.

\bibitem[\citeproctext]{ref-StataCorp2021}
StataCorp. (2023c). \emph{Stata statistical software: Release 18}.
StataCorp LLC.

\bibitem[\citeproctext]{ref-Stock2003}
Stock, J. H., \& Watson, M. W. (2003). \emph{Introduction to
econometrics}. Pearson.

\bibitem[\citeproctext]{ref-Strevens2020}
Strevens, M. (2020). \emph{The knowledge machine: How irrationality
created modern science}. Liveright.

\bibitem[\citeproctext]{ref-Takahashi2000}
Takahashi, S. (2000). \emph{Triumph of the sparrow : Zen poems} (L.
Stryk \& T. Ikemoto, Trans.). Grove Press.

\bibitem[\citeproctext]{ref-UNESCO1997}
UNESCO. (1997). {A}ime {C}esaire: The liberating power of words.
\emph{UNESCO Courier}.

\bibitem[\citeproctext]{ref-UNICEF2014}
UNICEF. (2014). \emph{Hidden in plain sight: A statistical analysis of
violence against children} (pp. 1--206). UNICEF.

\bibitem[\citeproctext]{ref-UNICEF2021}
UNICEF. (2024). \emph{Multiple indicator cluster surveys (MICS)}.
UNICEF. \url{https://mics.unicef.org/}

\bibitem[\citeproctext]{ref-UN2022}
United Nations. (2022). \emph{Sustainable development goals}.

\bibitem[\citeproctext]{ref-UNDPHDI}
United Nations Development Program. (2022). \emph{{Human Development
Index (HDI)}}.
\url{https://hdr.undp.org/data-center/human-development-index\#/indicies/HDI}

\bibitem[\citeproctext]{ref-UN1989}
United Nations General Assembly. (1989). \emph{Convention on the rights
of the child}.

\bibitem[\citeproctext]{ref-Census2022}
United States Census Bureau. (2022). About the topic of race. In
\emph{About the Topic of Race}.
\url{https://www.census.gov/topics/population/race/about.html}

\bibitem[\citeproctext]{ref-Viera2008}
Viera, A. J. (2008). Odds ratios and risk ratios: What's the difference
and why does it matter? \emph{Southern Medical Journal}.
\url{https://doi.org/10.1097/SMJ.0b013e31817a7ee4}

\bibitem[\citeproctext]{ref-Waddington2022}
Waddington, H. S., Villar, P. F., \& Valentine, J. C. (2022). Can
non-randomised studies of interventions provide unbiased effect
estimates? A systematic review of internal replication studies.
\emph{Evaluation Review}.
\url{https://doi.org/10.1177/0193841X221116721}

\bibitem[\citeproctext]{ref-Ward2023}
Ward, K. P., Grogan-Kaylor, A. C., Ma, J., Pace, G., \& Lee, S. J.
(2023). Associations between 11 parental discipline behaviors and child
outcomes across 60 countries. \emph{BMJ Open}.
\url{https://doi.org/10.1136/bmjopen-2021-058439}

\bibitem[\citeproctext]{ref-WardA}
Ward, K. P., Grogan-Kaylor, A. C., Pace, G. T., Cuartas, J., \& Lee, S.
J. (2021). {A Multilevel Ecological Analysis of the Predictors of
Spanking Across 65 Countries}. \emph{BMJ Open}, \emph{11}(e046075).
\url{https://doi.org/10.1136/bmjopen-2020-046075}

\bibitem[\citeproctext]{ref-ward_grogan-kaylor_ma_pace_lee_2021}
Ward, K. P., Grogan-Kaylor, A., Ma, J., Pace, G. T., \& Lee, S. J.
(2021). \emph{Associations between 11 parental discipline behaviors and
child outcomes across 60 countries}. PsyArXiv.
\url{https://doi.org/10.31234/osf.io/f5t8x}

\bibitem[\citeproctext]{ref-WardC}
Ward, K. P., Lee, S. J., Grogan-Kaylor, A. C., Ma, J., \& Pace, G. T.
(2022). {Patterns of Caregiver Aggressive and Nonaggressive Discipline
Toward Young Children in Low- and Middle-Income Countries: A Latent
Class Approach}. \emph{Child Abuse \& Neglect}, \emph{128}.
\url{https://doi.org/10.1016/j.chiabu.2022.105606}

\bibitem[\citeproctext]{ref-Watts2011}
Watts, D. J. (2011). \emph{Everything is obvious: *Once you know the
answer}. Crown Business.

\bibitem[\citeproctext]{ref-Whyte2016}
Whyte, D., \& Tippett, K. (2016). \emph{{D}avid {W}hyte: Seeking
language large enough}. The On Being Project.
\url{https://onbeing.org/programs/david-whyte-seeking-language-large-enough/}

\bibitem[\citeproctext]{ref-Wiest2007}
Wiest, L. R., Higgins, H. J., \& Frost, J. H. (2007). Quantitative
literacy for social justice. \emph{Equity \& Excellence in Education},
\emph{40}, 47--55. \url{https://doi.org/10.1080/10665680601079894}

\bibitem[\citeproctext]{ref-Wooldridge2010}
Wooldridge, J. M. (2010). \emph{Econometric analysis of cross section
and panel data}. The MIT Press.
\url{http://www.jstor.org/stable/j.ctt5hhcfr}

\bibitem[\citeproctext]{ref-WorldBankPovertyLine}
World Bank. (2024). \emph{Poverty and inequality}. Poverty and
Inequality.
\url{https://datatopics.worldbank.org/world-development-indicators/themes/poverty-and-inequality.html}

\end{CSLReferences}

\bookmarksetup{startatroot}

\chapter{About the Author}\label{about-the-author}

I am the Sandra K. Danziger Collegiate Professor of Social Work at the
University of Michigan School of Social Work.

My interests are in developing more knowledge to reduce violence against
children and Adverse Childhood Experiences (ACEs), with the aim of
improving child and family well-being. It is my hope that a better
understanding of how to reduce violence against children, and how to
reduce ACEs, will contribute to a better understanding of how to improve
mental health and well-being across the lifespan. In this research I try
to understand the family and community origins of aggression, antisocial
behavior, anxiety and depression across diverse communities and
contexts. My current research focuses on parenting and child development
using international data. I try to understand these issues within the
context of current conversations about children's rights.

A particular focus of my work has been to examine the outcomes of
physical punishment. Working closely with many colleagues, we have shown
that physical punishment is associated with a wide variety of negative
outcomes. This finding remains true even in contexts when physical
punishment is used minimally, or when used in ostensibly ``normative''
ways. We have investigated these associations across diverse communities
and countries. Lastly, we have worked to demonstrate more ``causally
robust'' associations between physical punishment and undesirable child
outcomes using a variety of quantitative methods.

A more recent stream of research examines a broader range of parenting
behaviors, with particular emphasis on ``positive parenting''
strategies.

I teach courses mostly in the area of statistics, quantitative methods
and data visualization.



\printindex


\end{document}
